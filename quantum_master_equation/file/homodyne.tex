
\section{平衡ホモダイン測定 (Balanced Homodyne detection)}
\subsection{Beam spliterの一般論}
今,被測定系Mとプローブ系Pを考える.また,被測定系の入力光,出力光を表す演算子をそれぞれ$\hat{a}_{\rm{M}}$, $\hat{a}_1$と表す.同様にして,プローブ系についても,$\hat{a}_{\rm{P}}$, $\hat{a}_2$と表す.

ビームスプリッターとは,入力光ビームが2つ$(\hat{a}_{\rm{P}},\ \hat{a}_{\rm{M}})$,出力光ビームが2つ$(\hat{a}_{1},\ \hat{a}_{2})$の2入力,2出力デバイスである.その出力関係は次の線形変換で記述される:
\begin{equation}
    \left(
        \begin{array}{c}
       \hat{a}_1\\[10pt]
       \hat{a}_2\\
        \end{array}
        \right)
        =B\left(
        \begin{array}{c}
       \hat{a}_{\rm{P}}\\[10pt]
       \hat{a}_{\rm{M}}\\
        \end{array}
        \right)
        =
        \left(
        \begin{array}{cc}
       B_{11}&B_{12}\\[10pt]
        B_{21}&B_{22} \\
        \end{array}
        \right)
        \left(
        \begin{array}{c}
       \hat{a}_{\rm{P}}\\[10pt]
       \hat{a}_{\rm{M}}\\
        \end{array}
        \right)
\end{equation}
$B$は$2\times2$行列であり,エネルギー保存の要請,つまり入出力の間で,光子数の変化はないことを示し,次式を満たす必要がある:
\begin{equation}\label{law_of_conservation}
    \hat{a}^{\dag}_1\hat{a}_1 + \hat{a}^{\dag}_2\hat{a}_2
    =\hat{a}^{\dag}_{\rm{P}}\hat{a}_{\rm{P}} + \hat{a}^{\dag}_{\rm{M}}\hat{a}_{\rm{M}}
\end{equation}

\begin{align}
    \hat{a}_1 &= B_{11}\hat{a}_{\rm{P}} + B_{12}\hat{a}_{\rm{M}}\\[10pt]
    \hat{a}_2 &= B_{21}\hat{a}_{\rm{P}} + B_{22}\hat{a}_{\rm{M}}
\end{align}

\begin{align}
    \hat{a}^{\dag}_1\hat{a}_1 + \hat{a}^{\dag}_2\hat{a}_2 
    &=
    (B_{11}^{\ast}\hat{a}^{\dag}_{\rm{P}} +B_{12}^{\ast}\hat{a}^{\dagger}_{\rm{M}})
    (B_{11}\hat{a}_{\rm{P}} + B_{12}\hat{a}_{\rm{M}})
    +(B_{21}^{\ast}\hat{a}^{\dag}_{\rm{P}} +B_{22}^{\ast}\hat{a}^{\dagger}_{\rm{M}})
    (B_{21}\hat{a}_{\rm{P}} + B_{22}\hat{a}_{\rm{M}})\nn[10pt]
    &=(|B_{11}|^2+|B_{21}|^2)\hat{a}^{\dagger}_{\rm{P}}\hat{a}_{\rm{P}}
    +(|B_{12}|^2+|B_{22}|^2)\hat{a}^{\dagger}_{\rm{M}}\hat{a}_{\rm{M}}\nn[10pt]
    &+(B^{\ast}_{11}B_{12}+B^{\ast}_{21}B_{22})\hat{a}^{\dagger}_{\rm{P}}\hat{a}_{\rm{M}}
    +(B_{11}B^{\ast}_{12}+B_{21}B^{\ast}_{22})\hat{a}^{\dagger}_{\rm{M}}\hat{a}_{\rm{P}}
\end{align}
よって,光子数の保存要請\eqref{law_of_conservation}から,行列$B$が満たすべき条件は次のようになる:
\begin{align}
    |B_{11}|^2+|B_{21}|^2 &= |B_{12}|^2+|B_{22}|^2 = 1\\[10pt]
    B^{\ast}_{11}B_{12}+B^{\ast}_{21}B_{22} &= B_{11}B^{\ast}_{12}+B_{21}B^{\ast}_{22}) = 0
\end{align}
この条件は,行列$B$がユニタリ行列$(B^{\dagger}B=I_2)$であることと等価である.これは損失のない過程を考えているので当然ともいえる.
任意の2次ユニタリ行列は,$\Lambda$, $\Psi$, $\Theta$, $\Phi$を任意の実数として,次のように書ける:
\begin{equation}
    B=e^{i\Lambda/2}
        \left(
        \begin{array}{cc}
       e^{i\Phi/2}&0\\[10pt]
        0&e^{-i\Phi/2} \\
        \end{array}
        \right)
        \left(
        \begin{array}{cc}
       \cos{\Theta/2}&\sin{\Theta/2}\\[10pt]
        -\sin{\Theta/2}&\cos{\Theta/2} \\
        \end{array}
        \right)
        \left(
        \begin{array}{cc}
       B_{11}&B_{12}\\[10pt]
        B_{21}&B_{22} \\
        \end{array}
        \right)
        \left(
        \begin{array}{cc}
       e^{i\Phi/2}&0\\[10pt]
        0&e^{-i\Phi/2} \\
        \end{array}
        \right)
\end{equation}
ここで,$\Lambda$, $\Psi$, $\Phi$は位相因子であるから,光学実験において適当に調整できるパラメータである.したがって,$\Lambda=\Psi=\Phi=0$とセットし,
\begin{equation}
    B=
        \left(
        \begin{array}{cc}
       \cos{\Theta/2}&\sin{\Theta/2}\\[10pt]
        -\sin{\Theta/2}&\cos{\Theta/2} \\
        \end{array}
        \right)
\end{equation}
とできる.透過率$T$, 反射率$R$をそれぞれ
\begin{align}
    T + R =1
\end{align}
て定義すると,行列$B$は
\begin{equation}
    B=
        \left(
        \begin{array}{cc}
       \sqrt{T}&-\sqrt{R}\\[10pt]
        \sqrt{R}&\sqrt{T}\\
        \end{array}
        \right)
\end{equation}

\subsection{バランス型ホモダイン検出}
ビームスプリッターとして,$50/50$ビームスプリッターを用いる.これは,$T=R=1/2$という意味である.つまり,ビームスプリッターによる変換を表す行列$B$は
\begin{equation}
    B=
        \left(
        \begin{array}{cc}
       \sqrt{1/2}&-\sqrt{1/2}\\[10pt]
        \sqrt{1/2}&\sqrt{1/2}\\
        \end{array}
        \right)
\end{equation}
となる.また,被測定系には単一モードの電磁場を利用し,$\hat{a}_{\rm{S}} = \hat{a}_{\rm{M}}$とする.(これは,シグナル光と呼ばれている.)プローブ系にも同様に単一モードの電磁場を利用し,$\hat{a}_{\rm{LO}} = \hat{a}_{\rm{P}}$とする.(こちらはLocal Oscillatorとよぶ.)このとき,入出力の関係は,
\begin{align}
    \hat{a}_1 &= \sqrt{T}\hat{a}_{\rm{LO}} - \sqrt{R}\hat{a}_{\rm{S}}\\[10pt]
    \hat{a}_2 &= \sqrt{T}\hat{a}_{\rm{S}} + \sqrt{R}\hat{a}_{\rm{LO}}
\end{align}
となる.ここで,測定するのは光の強さ(フォトン数)であり,測定の出力は光子検出器1,2でそれぞれ測定したフォトン数の差$\hat{R}\equiv\hat{n_2}-\hat{n_1}$で表される.ここで,$\hat{n}_1=\hat{a}^{\dag}_1\hat{a}_1$, $\hat{n}_2=\hat{a}^{\dag}_2\hat{a}_2$である.

\begin{align}
    \hat{R}&=\hat{n}_2-\hat{n}_1
    =\hat{a}^{\dag}_2\hat{a}_2-\hat{a}^{\dag}_1\hat{a}_1\nn[10pt]
    &=(\sqrt{T}\hat{a}^{\dagger}_{\rm{S}} + \sqrt{R}\hat{a}^{\dagger}_{\rm{LO}})
    (\sqrt{T}\hat{a}_{\rm{S}} + \sqrt{R}\hat{a}_{\rm{LO}})
    -(\sqrt{T}\hat{a}^{\dagger}_{\rm{LO}} - \sqrt{R}\hat{a}^{\dagger}_{\rm{S}})
    (\sqrt{T}\hat{a}_{\rm{LO}} - \sqrt{R}\hat{a}_{\rm{S}})\nn[10pt]
    &=(T-R)\hat{a}^{\dagger}_{\rm{S}}\hat{a}_{\rm{S}} 
    - (T-R)\hat{a}^{\dagger}_{\rm{LO}}\hat{a}_{\rm{LO}}
    +2\sqrt{TR}
    (\hat{a}^{\dagger}_{\rm{LO}}\hat{a}_{\rm{S}} + \hat{a}^{\dagger}_{\rm{S}}\hat{a}_{\rm{LO}})
\end{align}

ここで, $T=R=1/2$のとき,
\begin{align}
    \hat{R}=\hat{n}_2-\hat{n}_1
    =\hat{a}^{\dagger}_{\rm{LO}}\hat{a}_{\rm{S}} + \hat{a}^{\dagger}_{\rm{S}}\hat{a}_{\rm{LO}}
\end{align}

ビームスプリッターの入力状態,すなわち,全系の初期状態を$\ket{\Psi}\equiv\ket{\psi}_{\rm{LO}}\ket{\varphi}_{\rm{S}}$とすると,平均光子数は
\begin{align}
    \braket{\Psi|\hat{R}|\Psi}
    &=\bra{\varphi}_{\rm{S}}\bra{\psi}_{\rm{LO}}\hat{R}\ket{\varphi}_{\rm{S}}\ket{\psi}_{\rm{LO}}\nn[10pt]
    &=\bra{\varphi}_{\rm{S}}\bra{\psi}_{\rm{LO}}
    (\hat{a}^{\dagger}_{\rm{LO}}\hat{a}_{\rm{S}}
    +\hat{a}^{\dagger}_{\rm{S}}\hat{a}_{\rm{LO}})
    \ket{\varphi}_{\rm{S}}\ket{\psi}_{\rm{LO}}\nn[10pt]
    &=\braket{\psi|\hat{a}^{\dagger}_{\rm{LO}}|\psi}_{\rm{LO}}
    \cdot\braket{\varphi|\hat{a}_{\rm{S}}|\varphi}_{\rm{S}}
    +\braket{\psi|\hat{a}_{\rm{LO}}|\psi}_{\rm{LO}}
    \cdot\braket{\varphi|\hat{a}^{\dagger}_{\rm{S}}|\varphi}_{\rm{S}}
\end{align}
シグナル光$\ket{\varphi}_{\rm{S}}$では,スクイズド状態などさまざまな状態の光が入力として利用される.LOでは,コヒーレント状態$\ket{\alpha}$を用いる.すると,$\hat{a}_{\rm{LO}}\ket{\alpha}=\alpha\ket{\alpha}$より,
\begin{align}
    \braket{\Psi|\hat{R}|\Psi}
    &=\alpha^{\ast}\braket{\varphi|\hat{a}_{\rm{S}}|\varphi}_{\rm{S}}
    +\alpha\braket{\varphi|\hat{a}^{\dagger}_{\rm{S}}|\varphi}_{\rm{S}}
\end{align}
ここで,$\alpha=|\alpha|e^{i\theta_{\rm{LO}}}$とすると,$\braket{\hat{R}}$は,
\begin{align}
    \braket{\Psi|\hat{R}|\Psi}
    &=|\alpha|e^{-i\theta_{\rm{LO}}}\braket{\varphi|\hat{a}_{\rm{S}}|\varphi}_{\rm{S}}
    +|\alpha|e^{i\theta_{\rm{LO}}}\braket{\varphi|\hat{a}^{\dagger}_{\rm{S}}|\varphi}_{\rm{S}}\nn[10pt]
    &=2|\alpha|
    \Braket{\varphi|
    \frac{\hat{a}^{\dagger}_{\rm{S}}e^{i\theta_{\rm{LO}}}+\hat{a}_{\rm{S}}e^{-i\theta_{\rm{LO}}}}{2}
    |\varphi}_{\rm{S}}\nn[10pt]
    &=2|\alpha|\braket{\varphi|\hat{x}(\theta_{\rm{LO}})|\varphi}_{\rm{S}}
\end{align}
を得る.すなわち,$\hat{R}$の測定から,$\hat{x}(\theta_{\rm{LO}})$の期待値を知ることができる.例えば,$\hat{x}(\theta_{\rm{LO}}=0)$のとき,
