
\section{自由空間の電磁場}
 はじめは,$z$軸方向に一様な電場$\bm E_0=(0,0,E_0)$,$E_0=\bm E_0$があったとする.そこに帯電していない半径$a$の導体球を持ち込んだとき,電子は電場からの力によって,$z$軸の負の方向に移動し,$z$軸の正の方向は正に帯電する.したがって,導体表面上には電荷が誘導される.その誘導電荷によってつくられる静電場$\bm E_1$とはじめの静電場$\bm E_0$との重ね合わせによる静電場$\bm E$ができる.\\
 問題は,導体球によって,乱された静電ポテンシャル,すなわち電場$\bm E$による静電ポテンシャルを見出すことである.
\subsection{1次元の波動方程式}
$\psi=\psi(x,t)$とするとき,
\be\label{wd1}
\fbox{
$\left(
\dfrac{\partial^2}{\partial t^2}-v^2\dfrac{\partial^2}{\partial x^2}
\right)
\psi=0$
}
\ee
を1次元の波動方程式という.ただし,$v$は正の実定数である.\\
 (\ref{wd1})の一般解は,(2階の偏微分が可能で,偏微分して連続な)任意関数$f=f(x-vt)$,$g=g(x+vt)$を用いて,
\be
\psi=f(x-vt)+g(x+vt)
\ee
の形にかける.


















%
\begin{proof}
$t$と$x$の偏微分に対して,次の表記法を定義する:
\be
\partial_t\equiv\dfrac{\partial}{\partial t}
,\ \ \ \ 
\partial_x\equiv\dfrac{\partial}{\partial x}
\ee
すると,(\ref{wd1})は,
\be
(\partial_t^2-v^2\partial_x^2)\psi=0
\ee
と書ける.上式の左辺の()の中を因数分解すれば,
\be\label{wd12}
(\partial_t+v\partial_x)(\partial_t-v\partial_x)\psi=0
\ee
となる.\\
 次に変数変換$(x,t)\to[\xi,\eta]$,$\psi(x,t)=\psi[\xi,\eta]$
\begin{subnumcases}
  {}
  \xi=x-vt& \\
  \eta=x+vt&
\end{subnumcases}
を考える.合成関数の偏微分chain-ruleより
\begin{align}
\frac{\partial}{\partial t}
&=\frac{\partial\xi}{\partial t}\frac{\partial}{\partial \xi}+\frac{\partial\eta}{\partial t}\frac{\partial}{\partial\eta}
=-v\frac{\partial}{\partial \xi}+v\frac{\partial}{\partial\eta}
=v\left(-\frac{\partial}{\partial \xi}+\frac{\partial}{\partial\eta}\right)\\[10pt]
%
\frac{\partial}{\partial x}
&=\frac{\partial\xi}{\partial x}\frac{\partial}{\partial \xi}+\frac{\partial\eta}{\partial x}\frac{\partial}{\partial\eta}
=\frac{\partial}{\partial \xi}+\frac{\partial}{\partial\eta}
\end{align}
%%%%%%%%
したがって,
\begin{subnumcases}
  {}
  \partial_t+v\partial_x=2v\frac{\partial}{\partial \eta}& \\[10pt]
  \partial_t-v\partial_x=-2v\frac{\partial}{\partial \xi}&
\end{subnumcases}
となり,(\ref{wd12})は,
\be
(\partial_t+v\partial_x)(\partial_t-v\partial_x)\psi=-4v^2\frac{\partial}{\partial \eta}\frac{\partial}{\partial \xi}\psi=0
\ee
\be
\therefore\frac{\partial}{\partial \eta}\left(\frac{\partial\psi}{\partial \xi}\right)=0
\ee
となる.つまり,$\dfrac{\partial\psi}{\partial \xi}$は$\eta$に依存しないということがわかる.しかし,$\dfrac{\partial\psi}{\partial \xi}$は$\xi$にはいくらでも依存してもかまわないので,$\xi$のみの任意関数
\be
f=f[\xi,\eta]=f(\xi)
\ee
を用いて,
\be
\frac{\partial\psi}{\partial \xi}=\text{$\xi$のみの関数}=\frac{\partial f}{\partial \xi}
\ee
と書ける.ここで,右辺を$f(\xi)$ではなく$\dfrac{\partial\psi}{\partial \xi}$としたのは便宜的な理由による.\\
 よって,
\be
\frac{\partial}{\partial \xi}\left\{
\psi-f(\xi)
\right\}=0
\ee
となる.上式より,$\psi-f(\xi)$は$\xi$に依存しないということがわかる.
\footnote{%
ここで注意したいのは,$\psi$と$f(\xi)$の引き算が$\xi$に依存しないということである.
}
$\eta$にはいくらでも依存していいので,$\eta$のみの任意関数
\be
g=g[\xi,\eta]=g(\eta)
\ee
を用いて,
\be
\psi-f(\xi)=g(\eta)
\ee
と書ける.\\
 したがって,
\be
\psi[\xi,\eta]=f(\xi)+g(\eta)
\ee
となる.変数を$[\xi,\eta]\to(x,t)$に戻せば,
\be
\psi(x,t)=f(x-vt)+g(x+vt)
\ee
と書ける.
\end{proof}























%
\subsection{進行電磁波}
$z$軸方向に進む電磁場を考える.これは${\bm{E}}$,${\bm{B}}$が$z$と$t$のみに依存することを意味する.そこで以下では,
\be
\partial_x{\bm{E}}=\partial_x{\bm{E}}=\bm{0},\ \ \ \ \partial_x{\bm{B}}=\partial_x{\bm{B}}=\bm{0}
\ee
とする.\\
 自由空間でのMaxwell方程式は
\begin{align}
\label{mx1}
\nabla\times\bm{E}(\vr,t)+\dfrac{\partial\bm{B}(\vr,t)}{\partial t}=\bm 0
\end{align}

\begin{align}
\label{mx2}
\nabla\cdot\bm{B}(\vr,t)=0
\end{align}

\begin{align}
\label{mx3}
   \nabla\times\bm{B}(\vr,t)=\epsilon_0\mu_0\dfrac{\partial\bm{E}(\vr,t)}{\partial t}
\end{align}

\begin{align}
\label{mx4}
  \nabla\cdot{\bm{E}}(\vr,t)
      =0 
\end{align}
である.\\
 (\ref{mx2}),(\ref{mx4})より,
\be
\nabla\cdot{\bm{E}}={\partial_x} E_x+{\partial_y} E_y+{\partial_z} E_z=0,\ \ \ \therefore{\partial_z} E_z=0
\ee
%
\be
\nabla\cdot{\bm{B}}={\partial_x} B_x+{\partial_y} B_y+{\partial_z} B_z=0,\ \ \ \therefore{\partial_z} B_z=0
\ee
となる.\\
 (\ref{mx1}),(\ref{mx3})の$z$成分より,
\be
{\partial_x} E_y+{\partial_y} E_x=-{\partial_z} B_z,\ \ \ \therefore{\partial_z} B_z=0
\ee
%
\be
{\partial_x} B_y+{\partial_y} B_x=\epsilon_0\mu_0{\partial_z} E_z,\ \ \ \therefore{\partial_z} E_z=0
\ee
したがって,$E_z$,$B_z$は$x,y,t$に依存しない定数である.そこで,これをゼロとおく.(以下,静電場,静磁場部分は$0$とおく.):
\be
E_z=0,\ \ \ \ B_z=0
\ee
よって,${\bm{k}}=(0,0,1)$とすれば,
\be
{\bm{E}}\perp{\bm{k}},\ \ \ \ {\bm{B}}\perp{\bm{k}}
\ee
となる.これは進行電磁波が横波であることを表す.\\
%
%
 次に,${\bm{E}}$の方向を$x$軸方向にとる.つまり,$E_y=E_z=0$とする.(\ref{mx1})の$x$成分,(\ref{mx3})の$y$成分より,
\be
{\partial_y} E_z+{\partial_z} E_y=-{\partial_z} B_x,\ \ \ \therefore{\partial_z} B_x=0
\ee
%
\be
{\partial_z} B_x+{\partial_x} B_z=\epsilon_0\mu_0{\partial_z} E_y,\ \ \ \therefore{\partial_z} B_x=0
\ee
よって,$B_x$は$x,y,z,t$に依存しない定数であり,$B_x=0$とおく.したがって,
\be
{\bm{E}}=(E_x,0,0),\ \ \ {\bm{B}}=(0,B_y,0)
\ee
となり,${\bm{E}},{\bm{B}},{\bm{k}}$は互いに直交する.
%
%
以下,常に${\bm{E}}=(E_x,0,0)$であるとする.\\
${\bm{E}}$のみたす波動方程式は,$\epsilon_0\mu_0=1/c^2$として,
\be
(\partial_t^2-v^2\partial_x^2)\psi=0
\ee
であった,$0$でない$E_x$については,
\be
\partial_t^2E_x-c^2({\partial_x}^2E_x+{\partial_y}^2E_x+{\partial_z}^2E_x)=0
\ee
であるが,${\partial_x} E_x={\partial_y} E_x=0$なので,
\be\label{wex}
\dfrac{\partial^2E_x}{\partial t^2}-c^2\dfrac{\partial^2E_x}{\partial z^2}=0
\ee
となる.これは1次元の波動方程式なので,$E_x$は任意関数$f$と$g$を用いて,
\be\label{sol1}
E_x=f(z-ct)+g(z+ct)
\ee
(\ref{mx1})の$y$成分は,
\be
{\partial_z} E_x-{\partial_x} E_z=-{\partial_z} B_y
\ee
$E_z=0$であるから,したがって,
\begin{eqnarray*}
  \begin{split}
{\partial_z} B_y=-{\partial_z} E_x&=-\frac{\partial}{\partial z}
\left\{
f(z-ct)+g(z+ct)
\right\}\\[10pt]
%
&=-\left\{
\frac{\partial(z-ct)}{\partial z}\frac{\partial f(z-ct)}{\partial (z-ct)}
+\frac{\partial(z+ct)}{\partial z}\frac{\partial g(z+ct)}{\partial (z+ct)}
\right\}\\[10pt]
%
&=-\left\{
\frac{\partial f(z-ct)}{\partial (z-ct)}
+\frac{\partial g(z+ct)}{\partial (z+ct)}
\right\}\\[10pt]
%
&=\frac{1}{c}\left\{
\frac{\partial f(z-ct)}{\partial (z-ct)}\cdot(-c)
-\frac{\partial g(z+ct)}{\partial (z+ct)}\cdot(c)
\right\}\\[10pt]
%
&=\frac{1}{c}\left\{
\frac{\partial f(z-ct)}{\partial (z-ct)}\cdot(-c)
-\frac{\partial g(z+ct)}{\partial (z+ct)}\cdot(c)
\right\}\\[10pt]
  \end{split}
\end{eqnarray*}
ここで,$\mp c={\partial_z} (z\mp ct)$であるから,
\begin{eqnarray*}
  \begin{split}
{\partial_z} B_y=-{\partial_z} E_x
&=\frac{1}{c}\left\{
\frac{\partial f(z-ct)}{\partial (z-ct)}\frac{\partial (z-ct)}{\partial t}
-\frac{\partial g(z+ct)}{\partial (z+ct)}\frac{\partial (z+ct)}{\partial t}
\right\}\\[10pt]
%
&=\frac{1}{c}\left\{
\frac{\partial f(z-ct)}{\partial t}
-\frac{\partial g(z+ct)}{\partial t}
\right\}\\[10pt]
%
&=\frac{\partial }{\partial t}
\left[
\frac{1}{c}\left\{
f(z-ct)+g(z+ct)
\right\}\right]\\[10pt]
  \end{split}
\end{eqnarray*}
つまり,
\be
\frac{\partial }{\partial t}
\left[
B_y-
\frac{1}{c}\left\{
f(z-ct)+g(z+ct)
\right\}\right]=0
\ee
を得る.$B_y$は$x,y$に依存しないので,
\be
B_y=\frac{1}{c}\left\{
f(z-ct)+g(z+ct)
\right\}+D(z)
\ee
となる.ここで,$D(z)$は$z$のみに依存する任意関数である.\\
 (\ref{mx3})の$x$成分より,
\be
{\partial_y} B_z-{\partial_z} B_y=\epsilon_0\mu_0{\partial_z} E_x=\frac{1}{c^2}{\partial_z} E_x
\ee
$B_z=0$であるから,
\be
-{\partial_z} B_y=\frac{1}{c^2}{\partial_z} E_x
\ee
となる.すなわち,
\be
-\frac{\partial }{\partial z}
\left[
\frac{1}{c}\left\{
f(z-ct)+g(z+ct)
\right\}+D(z)\right]
%
=\frac{1}{c^2}\frac{\partial }{\partial t}
\left\{
f(z-ct)+g(z+ct)
\right\}\\[10pt]
\ee
%%
%%
\be
\therefore
-\frac{1}{c}\left\{
\frac{\partial f(z-ct)}{\partial (z-ct)}+\frac{\partial g(z+ct)}{\partial (z+ct)}
\right\}+\frac{\partial D(z)}{\partial z}
%
=\frac{1}{c^2}\left\{
\frac{\partial f(z-ct)}{\partial (z-ct)}\cdot(-c)+\frac{\partial g(z+ct)}{\partial (z+ct)}\cdot(c)
\right\}
\ee
したがって,${\partial_z} D(z)=0$.これは,$D(z)$が$x,y,z,t$に依存しない定数であることを表すので,$D(z)=0$とおく.\\
 以上から,
\be
E_x=f(z-ct)+g(z+ct)
\ee
\be
B_y=\frac{1}{c}
\left\{
f(z-ct)+g(z+ct)
\right\}
\ee
を得る.これが$z$軸の方向に進行する($x$軸の方向に偏った)電磁波の一般形である.




























%
\subsection{平面波}
$z$軸の方向に進む進行電磁波は,
\be
f_1=f_1(z-ct),\ \ \ f_2=f_2(z-ct)
\ee
\be
g_1=g_1(z+ct),\ \ \ g_2=g_2(z+ct)
\ee
を任意関数として,
\begin{subnumcases}
  {}
  E_x=f_1+g_1,\ \ \ E_y=f_2+g_2,\ \ \ E_z=0& \\[10pt]
  B_x=\frac{1}{c}(f_1+g_1),\ \ \ B_y=\frac{1}{c}(f_2+g_2),\ \ \ B_z=0&
\end{subnumcases}
と書けた.このとき,$f_1,f_2$を考えたもの(進行波)を,${\bm{E}}_{{\uparrow}}$,${\bm{B}}_{{\uparrow}}$.$g_1,g_2$を考えたもの(後退波)を,${\bm{E}}_{{\downarrow}}$,${\bm{B}}_{{\downarrow}}$としよう:
\be
{\bm{E}}_{{\uparrow}}=(f_1,f_2,0),\ \ \ {\bm{B}}_{{\uparrow}}=\left(-\frac{1}{c}f_2,-\frac{1}{c}f_1,0\right)
\ee
\be
{\bm{E}}_{{\downarrow}}=(g_1,g_2,0),\ \ \ {\bm{B}}_{{\downarrow}}=\left(\frac{1}{c}g_2,-\frac{1}{c}g_1,0\right)
\ee

















%
\subsubsection{直交性について}
\paragraph{右手系}
${\bm{E}}$と${\bm{B}}$は必ずしも直交しないが,${\bm{E}}_{{\uparrow}}\perp{\bm{B}}_{{\uparrow}}$($\therefore{\bm{E}}_{{\uparrow}}\cdot{\bm{B}}_{{\uparrow}}=0$),${\bm{E}}_{{\downarrow}}\perp{\bm{B}}_{{\downarrow}}$($\therefore{\bm{E}}_{{\downarrow}}\cdot{\bm{B}}_{{\downarrow}}=0$)であり,波の進む向きを${\bm{k}}$とすると,図\ref{},\ref{}のように右手系を作る.

















%
\paragraph{ポインティングベクトル}
ポインティングベクトル$\bm{S}=\dfrac{1}{\mu_0}{\bm{E}}\times{\bm{B}}$について,
\begin{align}
\bm{S}=\frac{1}{\mu_0}
\left(
    \begin{array}{c}
      f_1+g_1 \\[10pt]
      f_2+g_2 \\[10pt]
      0
    \end{array}
  \right)
  \times
  \left(
    \begin{array}{c}
      \dfrac{1}{c}(f_1+g_1) \\[10pt]
      \dfrac{1}{c}(f_2+g_2) \\[10pt]
      0
    \end{array}
  \right)
  =\frac{1}{\mu_0c}\left(0,0,f_1^2+f_2^2-g_1^2-g_2^2\right)
\end{align}
となるので,波の伝わる向き($z$軸方向)に,エネルギーが流れていることがわかる.



















%
\paragraph{電磁場のエネルギー密度}
電場と磁場のエネルギー密度をそれぞれ$w_e,w_m$とおく.電磁場のエネルギー密度
\be
w=w_e+w_m=\frac{\epsilon_0}{2}|{\bm{E}}|^2+\frac{1}{2\mu_0}|{\bm{B}}|^2
\ee
も${\bm{E}},{\bm{B}}$のままではきれいにならない.しかし,${\bm{E}}={\bm{E}}_{{\uparrow}}$,${\bm{B}}={\bm{B}}_{{\downarrow}}$のときは,
\be
w_e=\frac{\epsilon_0}{2}|{\bm{E}}_{{\uparrow}}|^2=\frac{\epsilon_0}{2}(f_1^2+f_2^2)
\ee
\be
w_m=\frac{1}{2\mu_0}|{\bm{B}}_{{\uparrow}}|^2=\frac{1}{2\mu_0c^2}(f_1^2+f_2^2)=\frac{\epsilon_0}{2}(f_1^2+f_2^2)
\ee
となり,$w_e=w_m$となる.



























\subsection{}
$f$のみ,$g$のみのように波の進む向きを決定したものが,通常は平面波と呼ばれるものである.以下,${\bm{E}}={\bm{E}}_{{\uparrow}}$,${\bm{B}}={\bm{B}}_{{\uparrow}}$として考える.
\be\label{v1}
z-ct=-c\left(
t-\frac{z}{c}
\right)
=-\frac{c}{\omega}\left(
\omega t-\frac{\omega z}{c}
\right)
=-\frac{c}{\omega}\left(
\omega t-kz
\right)
\ee
(ここで,$k=\omega/c$とおいた.)より,(正弦波を考えることにして)例えば
\be
f_1(z-ct)=E_0\sin{(\omega t-kz)}
\ee
とおくことができる.(\ref{v1})より,
\be
\sin(\omega t-kz)=\sin{\left\{
-\frac{\omega}{c}(z-ct)
\right\}}
\ee
さらに,$f_2=0$とすれば,
\begin{subnumcases}
  {}
  {\bm{E}}=E_0\sin{(\omega t-kz)}(1,0,0)& \\[10pt]
  {\bm{B}}=\frac{E_0}{c}\sin{(\omega t-kz)}(0,1,0)&
\end{subnumcases}
となり,このとき,
\be
\bm{S}=\frac{1}{\mu_0}{\bm{E}}\times{\bm{B}}=\frac{E_0^2}{\mu_0c^2}\sin^2(\omega t-kz)(0,0,1)
\ee
となる.




