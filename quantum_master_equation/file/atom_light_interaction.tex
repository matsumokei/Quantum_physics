
\part{原子と光の相互作用}
\section{原子-光相互作用の一般論}
原子系と電磁場の相互作用を考える.原子系,電磁場そして相互作用Hamiltonianをそれぞれ$\hat{H}_{\rm{A}}$,$\hat{H}_{\rm{F}}$,$\hat{H}^{'}$としたとき,全系のHamiltonianは,電気双極子近似のもとで,
\begin{align}
    \hat{H} &= \hat{H}_{\rm{A}}+\hat{H}_{\rm{F}}+\hat{H}^{'}\\[10pt]
    \hat{H}_{\rm{A}}&=\sum_{i}E_{i}\ket{i}\bra{i}\\[10pt]
    \hat{H}_{\rm{F}}&=\sum_{k}\hbar\omega_k\left(\hat{a}_k^{\dag}\hat{a}_k+\frac{1}{2}\right)\\[10pt]
    \hat{H}^{'}&=-e\hat{\bm{r}}\cdot\hat{\bm{E}}
\end{align}
と表される.ここで,$\ket{i}$は原子系の$i$番目のエネルギー固有値$E_i$に対応する固有状態,$\hat{\bm{r}}$はポテンシャルの中心を原点とした時の電子の位置座標演算子であり,電場演算子$\hat{\bm{E}}$は原子内で一様であるとする.\\

相互作用$\hat{H}^{'}$については,原子系に演算する部分$e\hat{\bm{r}}$と電磁場に演算する部分$\hat{\bm{E}}$に分けられる.まず,演算子$e\hat{\bm{r}}$を固有関数系$\{\ket{i}\}$で展開すると,
\begin{equation}
    e\hat{\bm{r}}=e\hat{I}\hat{\bm{r}}\hat{I}
    =\sum_{i,j}\ket{i}\braket{i|\hat{\bm{r}}|j}\bra{j}
    =\sum_{i,j}P_{i,j}\hsig_{i,j}
\end{equation}
となる.ここで,$P_{i,j}\equiv e\braket{i|\hat{\bm{r}}|j}$は電気双極子遷移の遷移行列要素であり,$\hsig_{i,j}=\ket{i}\bra{j}$は,原子系の$\ket{j}$から$\ket{i}$へ変換する状態変換演算子である.一方,電場$\hat{\bm{E}}$は,原子が固定されていることにより,(電磁場の量子化を参照)を使い,
\begin{equation}
    \hat{\bm{E}}=\sum_{k}{\Tilde{\bm{e}}}\mathcal{E}_k(\hat{a}_k+\hat{a}_k^{\dag})
\end{equation}
と表される.ここで簡単のために,電磁場は直線偏光(ここよくわからない)としてこれらをまとめると,
\begin{equation}
    \hat{H}^{'}=-e\hat{\bm{r}}\cdot\hat{\bm{E}}
    =-(\sum_{i,j}P_{i,j}\hsig_{i,j})(\sum_{k}{\Tilde{\bm{e}}}\mathcal{E}_k(\hat{a}_k+\hat{a}_k^{\dag}))
    =\hbar\sum_{i,j}\sum_{k}g_k^{ij}\hsig_{i,j}(\hat{a}_k+\hat{a}_k^{\dag})
\end{equation}
where $g_k^{ij}=-P_{i,j}\cdot\Tilde{\bm{e}}\mathcal{E}_k$.
ここまでが一般論である.



\section{Jaynes-Cummings model}
2準位系を考える.第一励起状態を$\ket{e}$, 基底状態を$\ket{g}$とし,次のベクトル表示をとる:
\begin{equation}
     \ket{e}=\left(
        \begin{array}{c}
       1\\[10pt]
       0\\
        \end{array}
        \right),\ \ \ 
        \ket{g}=\left(
        \begin{array}{c}
       0\\[10pt]
       1\\
        \end{array}
        \right)
\end{equation}
とする.

\begin{equation}
     \hat{H}_{\rm{A}}=\left(
        \begin{array}{cc}
       \frac{1}{2}\hbar\omega&0\\[15pt]
       0&-\frac{1}{2}\hbar\omega\\
        \end{array}
        \right)
        =\frac{1}{2}\hbar\omega\hsig_z
\end{equation}
上昇演算子,下降演算子はそれぞれ次のように表される:
\begin{equation}
     \hsig_+=\ket{e}\bra{g}=\left(
        \begin{array}{cc}
       0&1\\[10pt]
       0&0\\
        \end{array}
        \right),\ \ \ 
        \hsig_-=\ket{g}\bra{e}=\left(
        \begin{array}{cc}
       0&0\\[10pt]
       1&0\\
        \end{array}
        \right)
\end{equation}

またこれらは,
\begin{equation}
    \hsig_{\pm}=\frac{\hsig_x\pm i\hsig_y}{2}
\end{equation}

\subsection{問題設定}
2準位原子が,周波数$\omega_0$の単一モードの電磁場と相互作用する状況を考える.このとき,2準位原子と単一モードの電磁場との相互作用を記述するHamiltonianは
\begin{equation}
    \hat{H}_{\rm{JC}}
    =\frac{1}{2}\hbar\omega\hsig_z
    +\hbar\omega_0\hat{a}^\dagger\ha
    +\hbar g(\hsig_+\hat{a}+\hsig_-\hat{a}^\dagger)
\end{equation}
と書ける.これはJaynes-Cummings modelと呼ばれている.


\subsection{Jaynes-Cummings modelの固有値問題}
ここでは,Jaynes-Cummings modelの固有状態および固有値を求める.そこで次の固有方程式を考える:
\begin{equation}
    \hat{H}_{\rm{JC}}\ket{\Psi}=E\ket{\Psi}
\end{equation}

ここで,$\ket{u,n}=\ket{u}\otimes\ket{n}$, $\ket{g,n}=\ket{g}\otimes\ket{n}$の基底で展開すると
\begin{align}
    \hsig_z\otimes\hat{1}_n
    &=(\ket{e}\bra{e}-\ket{g}\bra{g})\otimes
    \sum_{n=0}^{\infty}\ket{n}\bra{n}\nn[10pt]
    &=\sum_{n=0}^{\infty}(\ket{e,n}\bra{e,n}-\ket{g,n}\bra{g,n})
    \nn[10pt]
    &=-\ket{g,0}\bra{g,0}+\ket{e,0}\bra{e,0}
    -\ket{g,1}\bra{g,1}+\ket{e,1}\bra{e,1}+\cdots
    \nn[10pt]
    &=\left(
        \begin{array}{cc}
       -1&0\\[10pt]
       0&1\\
        \end{array}
        \right)
        \otimes
        \left(
        \begin{array}{ccccc}
        1                                                \\
         & 1          &        & \text{\huge{0}}   \\
         &                 & \ddots                     \\
         & \text{\huge{0}} &        & \ddots            \\
         &                 &        &           & \ddots
        \end{array}
        \right)\nn[10pt]
    &=  \left(
        \begin{array}{cccccc}
        -1                                                \\
         & 1          &        & &\text{\huge{0}}   \\
         &                 &-1                     \\
         & &        & 1           \\
         &                  \text{\huge{0}}&        &           & \ddots\\
         &                 &        &           &        &\ddots
        \end{array}
        \right)\nn[10pt]
\end{align}



\begin{align}
    \hat{1}_2\otimes\hat{a}^\dag\hat{a}
    &=(\ket{g}\bra{g}+\ket{e}\bra{e})\otimes
    \sum_{n=0}^{\infty}n\ket{n}\bra{n}\nn[10pt]
    &=\sum_{n=0}^{\infty}n(\ket{g,n}\bra{g,n}+\ket{e,n}\bra{e,n})
    \nn[10pt]
    &=0(\ket{g,0}\bra{g,0}+\ket{e,0}\bra{e,0})
    +1(\ket{g,1}\bra{g,1}+\ket{e,1}\bra{e,1})+\cdots
    \nn[10pt]
    &=\left(
        \begin{array}{cc}
       1&0\\[10pt]
       0&1\\
        \end{array}
        \right)
        \otimes
        \left(
        \begin{array}{cccccc}
        0                                                \\
         & 1          &        & &\text{\huge{0}}   \\
         &                 &2                    \\
         & &        & 3           \\
         &                  \text{\huge{0}}&        &           & \ddots\\
         &                 &        &           &        &\ddots
        \end{array}
        \right)\nn[10pt]
    &=  \left(
        \begin{array}{cccccc}
        0                                               \\
         & 1          &        & &\text{\huge{0}}   \\
         &                 &2                     \\
         & &        & 3           \\
         &                  \text{\huge{0}}&        &           & \ddots\\
         &                 &        &           &        &\ddots
        \end{array}
        \right)\nn[10pt]
\end{align}

\begin{align}
    \hsig_+\hat{a}
    &=(\ket{g}\bra{g}+\ket{e}\bra{e})\otimes
    \sum_{n=0}^{\infty}n\ket{n}\bra{n}\nn[10pt]
    &=\sum_{n=0}^{\infty}n(\ket{g,n}\bra{g,n}+\ket{e,n}\bra{e,n})
    \nn[10pt]
    &=0(\ket{g,0}\bra{g,0}+\ket{e,0}\bra{e,0})
    +1(\ket{g,1}\bra{g,1}+\ket{e,1}\bra{e,1})+\cdots
    \nn[10pt]
    &=\left(
        \begin{array}{cc}
       0&1\\[10pt]
       0&0\\
        \end{array}
        \right)
        \otimes
    \left(
        \begin{array}{ccccccc}
       0& 0& 0&0&0& \dots  & \dots\\
      1 &0& 0&0&0& \dots  & \dots\\
      0&\sqrt{2}& 0&0&0& \dots  & \dots\\
      0&0&\sqrt{3}&0& \dots& \dots  & \dots\\
      \vdots & \vdots &\vdots & \vdots & \ddots& \dots & \vdots \\
      \vdots & \vdots &\vdots & \vdots & \dots& \ddots & \vdots \\
      \dots & \dots&\dots & \dots& \dots& \dots  & \ddots
        \end{array}
        \right)\nn[10pt]
    &=  \left(
        \begin{array}{cccccc}
        0                                               \\
         & 1          &        & &\text{\huge{0}}   \\
         &                 &2                     \\
         & &        & 3           \\
         &                  \text{\huge{0}}&        &           & \ddots\\
         &                 &        &           &        &\ddots
        \end{array}
        \right)\nn[10pt]
\end{align}


\begin{align}
    \hsig_-\hat{a}^\dagger
    &=(\ket{g}\bra{g}+\ket{e}\bra{e})\otimes
    \sum_{n=0}^{\infty}n\ket{n}\bra{n}\nn[10pt]
    &=\sum_{n=0}^{\infty}n(\ket{g,n}\bra{g,n}+\ket{e,n}\bra{e,n})
    \nn[10pt]
    &=0(\ket{g,0}\bra{g,0}+\ket{e,0}\bra{e,0})
    +1(\ket{g,1}\bra{g,1}+\ket{e,1}\bra{e,1})+\cdots
    \nn[10pt]
    &=\left(
        \begin{array}{ccccccc}
       0& 0& 0&0&0& \dots  & \dots\\
      1 &0& 0&0&0& \dots  & \dots\\
      0&\sqrt{2}& 0&0&0& \dots  & \dots\\
      0&0&\sqrt{3}&0& \dots& \dots  & \dots\\
      \vdots & \vdots &\vdots & \vdots & \ddots& \dots & \vdots \\
      \vdots & \vdots &\vdots & \vdots & \dots& \ddots & \vdots \\
      \dots & \dots&\dots & \dots& \dots& \dots  & \ddots
        \end{array}
        \right)\nn[10pt]
    &=  \left(
        \begin{array}{cccccc}
        0                                               \\
         & 1          &        & &\text{\huge{0}}   \\
         &                 &2                     \\
         & &        & 3           \\
         &                  \text{\huge{0}}&        &           & \ddots\\
         &                 &        &           &        &\ddots
        \end{array}
        \right)\nn[10pt]
\end{align}


したがって,Hamiltonian $\hat{H}_{\rm{JC}}$を行列表示すると,
\begin{align}
    \hat{H}_{\rm{JC}}
    &=\frac{1}{2}\hbar\omega(\hsig_z\otimes\hat{1}_N)
    +\hbar\omega_0(\hat{1}_2\otimes\hat{a}^\dagger\hat{a})
    +\hbar g(\hsig_+\hat{a}+\hsig_-\hat{a}^\dagger)\nn[10pt]
    &=  \left(
        \begin{array}{cccccc}
        - \frac{1}{2}\hbar\omega                                              \\
         & \frac{1}{2}\hbar\omega         &        & &\text{\huge{0}}   \\
         &                 &-\frac{1}{2}\hbar\omega                  \\
         & &        & \frac{1}{2}\hbar\omega         \\
         &                  \text{\huge{0}}&        &           & \ddots\\
         &                 &        &           &        &\ddots
        \end{array}
        \right)
    +  \left(
        \begin{array}{cccccc}
        0                                               \\
         & \hbar\omega_0          &        & &\text{\huge{0}}   \\
         &                 &2\hbar\omega_0                     \\
         & &        & 3\hbar\omega_0           \\
         &                  \text{\huge{0}}&        &           & \ddots\\
         &                 &        &           &        &\ddots
        \end{array}
        \right)\nn[10pt]
    &+  \left(
        \begin{array}{cccccccc}
       0& 0& 0&0&0&0& \dots  & \dots\\
      0&0& g&0&0&0& \dots  & \dots\\
      0&g& 0&0&0&0& \dots  & \dots\\
      0&0&0&0&\sqrt{2}g& \dots& \dots  & \dots\\
      0&0&0&\sqrt{2}g&0& \dots& \dots  & \dots\\
      \vdots & \vdots &\vdots&\vdots & \vdots & \ddots& \dots & \vdots \\
      \vdots & \vdots &\vdots&\vdots & \vdots & \dots& \ddots & \vdots \\
      \dots & \dots&\dots&\dots & \dots& \dots& \dots  & \ddots
        \end{array}
        \right)\nn[10pt]
    &=  \left(
        \begin{array}{cccccccc}
       - \frac{1}{2}\hbar\omega& 0& 0&0&0&0& \dots  & \dots\\
      0&\frac{1}{2}\hbar\omega& g&0&0&0& \dots  & \dots\\
      0&g& - \frac{1}{2}\hbar\omega+\hbar\omega_0&0&0&0& \dots  & \dots\\
      0&0&0&\frac{1}{2}\hbar\omega+\hbar\omega_0&\sqrt{2}g& \dots& \dots  & \dots\\
      0&0&0&\sqrt{2}g&- \frac{1}{2}\hbar\omega+2\hbar\omega_0& \dots& \dots  & \dots\\
      \vdots & \vdots &\vdots&\vdots & \vdots & \ddots& \dots & \vdots \\
      \vdots & \vdots &\vdots&\vdots & \vdots & \dots& \ddots & \vdots \\
      \dots & \dots&\dots&\dots & \dots& \dots& \dots  & \ddots
        \end{array}
        \right)
\end{align}
よって,状態をベクトル$\ket{e,n}$と$\ket{g,n+1}$で張られる部分空間のブロック行列を得る:
\begin{align}
    \hat{H}_{\rm{JC}}^{(n)}=\left(
        \begin{array}{cc}
      \frac{1}{2}\hbar\omega+n\hbar\omega_0& \hbar\sqrt{n+1}g\\[10pt]
      \hbar\sqrt{n+1}g& - \frac{1}{2}\hbar\omega+(n+1)\hbar\omega_0\\
        \end{array}
        \right)
\end{align}
ただし,$n=0,1,2,\ldots$である.
行列$\hat{H}_{\rm{JC}}^{(n)}$から得られる固有方程式は
\begin{align}
    \left(
        \begin{array}{cc}
      E-(\hbar\omega/2+n\hbar\omega_0)& -\hbar\sqrt{n+1}g\\[10pt]
      -\hbar\sqrt{n+1}g& E-(-\hbar\omega/2+(n+1)\hbar\omega_0)\\
        \end{array}
        \right)
        \left(
        \begin{array}{c}
      a\\[10pt]
      b\\
        \end{array}
        \right)=0
\end{align}

\begin{align}
    \left|
    E\hat{1}_2-\hat{H}_{\rm{JC}}^{(n)}
    \right|
    &=\left|\left(
        \begin{array}{cc}
      E-(\hbar\omega/2+n\hbar\omega_0)& -\hbar\sqrt{n+1}g\\[10pt]
      -\hbar\sqrt{n+1}g& E-(-\hbar\omega/2+(n+1)\hbar\omega_0)\\
        \end{array}
        \right)\right|\nn[10pt]
    &=\left[
    E-(\hbar\omega/2+n\hbar\omega_0)
    \right]
    \left[
    E-(-\hbar\omega/2+(n+1)\hbar\omega_0)
    \right]
    -(\hbar g)^2(n+1)=0
\end{align}


これを解くと
\begin{equation}
    E_{\pm}=\left(
    n+\frac{1}{2}
    \right)\hbar\omega_0\pm\frac{\hbar\Delta_n}{2}
\end{equation}
を得る.

$E=E_+$のとき
\begin{align}
    \left(
        \begin{array}{cc}
      \left(n+\frac{1}{2}
    \right)\hbar\omega_0+\frac{\hbar\Delta_n}{2}-(\hbar\omega/2+n\hbar\omega_0)& -\hbar\sqrt{n+1}g\\[10pt]
      -\hbar\sqrt{n+1}g& \left(n+\frac{1}{2}
    \right)\hbar\omega_0+\frac{\hbar\Delta_n}{2}-(-\hbar\omega/2+(n+1)\hbar\omega_0)\\
        \end{array}
        \right)
        \left(
        \begin{array}{c}
      a\\[10pt]
      b\\
        \end{array}
        \right)&=0\nn[10pt]
        \left(
        \begin{array}{cc}
      \frac{\hbar}{2}(\omega_0-\omega+\Delta_n)& -\hbar\sqrt{n+1}g\\[10pt]
      -\hbar\sqrt{n+1}g& -\frac{\hbar}{2}(\omega_0-\omega+\Delta_n)\\
        \end{array}
        \right)
        \left(
        \begin{array}{c}
      a\\[10pt]
      b\\
        \end{array}
        \right)&=0
\end{align}

この連立方程式を解くと
\begin{align}
    \frac{\hbar}{2}(\omega_0-\omega+\Delta_n)a-\hbar\sqrt{n+1}gb&=0\\[10pt]
    a&=\frac{\hbar\sqrt{n+1}g}{\frac{\hbar}{2}(\omega_0-\omega+\Delta_n)}b
\end{align}

規格化条件より$a^2+b^2=1$

\begin{align}
    1&=a^2+b^2=\left(
    1+\frac{\hbar^2(n+1)g^2}{\frac{\hbar^2}{4}(\omega_0-\omega+\Delta_n)^2}
    \right)b^2\nn[10pt]
    &=
    \frac{\frac{\hbar^2}{4}(\omega_0-\omega+\Delta_n)^2+\hbar^2(n+1)g^2}{\frac{\hbar^2}{4}(\omega_0-\omega+\Delta_n)^2}
    b^2
    =\frac{(\omega_0-\omega+\Delta_n)^2+4(n+1)g^2}{(\omega_0-\omega+\Delta_n)^2}
    b^2
    \end{align}
    
    \begin{align}
    \therefore 
    b^2&=\frac{(\Delta_n-(\omega-\omega_0))^2}{(\Delta_n-(\omega-\omega_0))^2+4(n+1)g^2}\nn[10pt]
    \therefore 
    b&=\frac{(\Delta_n-\delta)}{\sqrt{(\Delta_n-\delta)^2+4(n+1)g^2}}
    \equiv\cos\theta_n
\end{align}
ここで,$\delta=\omega-\omega_0$である.
\begin{align}
    a=\frac{2\sqrt{n+1}g}{(\omega_0-\omega+\Delta_n)}b
    =\frac{2\sqrt{n+1}g}
    {\sqrt{(\Delta_n-\delta)^2+4(n+1)g^2}}
    \equiv\sin\theta_n
\end{align}
ここで
\begin{equation}
    \tan\theta_n
    =\frac{\sin{\theta_n}}{\cos{\theta_n}}
    =\frac{2\sqrt{n+1}g}{(\Delta_n-\delta)}
\end{equation}
である.

よって,固有状態$\ket{\Psi_{\pm}}$についても
\begin{align}\label{JC_eigen_state1}
    \ket{\Psi_+}&=\sin{\theta_n}\ket{e,n}+\cos\theta_n\ket{g,n+1}\\[10pt]
    \label{JC_eigen_state2}
    \ket{\Psi_-}&=\cos{\theta_n}\ket{e,n}-\sin\theta_n\ket{g,n+1}
\end{align}

得られたエネルギー固有値から,電磁場と結合した原子のエネルギー準位が2つに分裂し,その差が
\begin{equation}
    \Delta E\equiv E_+-E_-=\hbar\Delta_n=\hbar\sqrt{4g^2(n+1)+\delta}
\end{equation}
であることがわかる.これをラビ分裂という.特に,電磁場が真空状態$(n=0)$の場合でさえも,原子のエネルギー準位が$\hbar\sqrt{4g^2+\delta}$だけ分裂を起こす.これを真空ラビ分裂(Vacume Labi Splitting)と呼ぶ.

\eqref{JC_eigen_state1},\eqref{JC_eigen_state2}のような電磁場と結合した状態にある原子は,あたかも電磁場のドレスをまとった原子のように見えるので,ドレストアトムと呼ばれる.




\subsection{状態の時間発展}
次にJC modelにおける状態の時間発展について論じる.今,初期時刻$t=0$では原子は第一励起状態$\ket{e}$にあり,電磁場は光子数状態$\ket{n}$にあるとする.このとき初期状態は
\begin{equation}
    \ket{\Psi(t=0)}=\ket{e}\ket{n}\equiv\ket{u,n}
\end{equation}
となる.系の時間発展は,Schr\"{o}dinger方程式
\begin{equation}\label{Sch.eq_JC}
    i\hbar\frac{\partial}{\partial t}\ket{\Psi(t)}=\hat{H}_{\rm{JC}}\ket{\Psi(t)}
\end{equation}
で記述される.

\eqref{Sch.eq_JC}の形式的な解は以下で与えられる:
\begin{equation}\label{Solve_Sch.eq_JC}
    \ket{\Psi(t)}=e^{-i\hat{H}_{\rm{JC}}t/\hbar}\ket{\Psi(0)}=e^{-i\hat{H}_{\rm{JC}}t/\hbar}\ket{e,n}
\end{equation}
ここで,\eqref{JC_eigen_state1},\eqref{JC_eigen_state2}を逆に解くことで,初期状態$\ket{e,n}$をJC-modelの固有状態$\ket{\Psi_+}$,$\ket{\Psi_-}$での展開した形を具体的に記述することができる:
\begin{align}
    \sin\theta_n\ket{\Psi_+}&=\sin^2{\theta_n}\ket{e,n}+\sin{\theta_n}\cos\theta_n\ket{g,n+1}\\[10pt]
    \cos{\theta_n}\ket{\Psi_-}&=\cos^2{\theta_n}\ket{e,n}-\sin\theta_n\cos\theta_n\ket{g,n+1}
\end{align}

\begin{equation}\label{initial_state}
    \ket{u,n} = \sin{\theta_n}\ket{\Psi_+}+\cos{\theta_n}\ket{\Psi_-}
\end{equation}

\eqref{initial_state}を\eqref{Solve_Sch.eq_JC}へ代入すると
\begin{align}\label{Solve_Sch.eq_JC2}
    \ket{\Psi(t)}&=e^{-i\hat{H}_{\rm{JC}}t/\hbar}
    (\sin{\theta_n}\ket{\Psi_+}+\cos{\theta_n}\ket{\Psi_-}\nn[10pt]
    &=
    \sin{\theta_n}e^{-iE_+t/\hbar}\ket{\Psi_+}
    +\cos{\theta_n}e^{-iE_-t/\hbar}\ket{\Psi_-}\nn[10pt]
    &=
    \sin{\theta_n}e^{-iE_+t/\hbar}\Bigl(\sin{\theta_n}\ket{e,n}+\cos\theta_n\ket{g,n+1}\Bigr)
    +\cos{\theta_n}e^{-iE_-t/\hbar}\Bigl(\cos{\theta_n}\ket{e,n}-\sin\theta_n\ket{g,n+1}\Bigr)
    \nn[10pt]
    %
    &=
    \Bigl(e^{-iE_+t/\hbar}\sin^2{\theta_n}
    +e^{-iE_-t/\hbar}\cos^2{\theta_n}\Bigr)\ket{e,n}
    +
    \sin{\theta_n}\Bigl(e^{-iE_+t/\hbar}-e^{-iE_-t/\hbar}\Bigr)\sin{\theta_n}\cos\theta_n\ket{g,n+1}
\end{align}
\eqref{Solve_Sch.eq_JC2}にJC modelのエネルギー固有値\eqref{},\eqref{}を代入すると,
\begin{align}
    e^{-iE_{\pm}t/\hbar}
    &=\exp{
    \left\{
    -i
    \left(
    \left(
    n+\frac{1}{2}
    \right)\hbar\omega_0\pm\frac{\hbar\Delta_n}{2}
    \right)
    t/\hbar
    \right\}
    }\nn[10pt]
    &=\exp{
    \left\{
    -i
    \left(
    n+\frac{1}{2}
    \right)\omega_0t
    \right\}
    }
    \exp{
    \left\{
    \mp i\frac{\hbar\Delta_n}{2}t
    \right\}
    }\nn[10pt]
    &=\exp{
    \left\{
    -i
    \left(
    n+\frac{1}{2}
    \right)\omega_0t
    \right\}
    }
    \left\{
    \cos\frac{\Delta_n t}{2}
    \mp i\sin\frac{\Delta_n t}{2}
    \right\}
\end{align}
より,まず
\begin{align}
    &e^{-iE_+t/\hbar}\sin^2{\theta_n}
    +e^{-iE_-t/\hbar}\cos^2{\theta_n}\nn[10pt]
    &=\exp{
    \left\{
    -i
    \left(
    n+\frac{1}{2}
    \right)\omega_0t
    \right\}
    }
    \left\{
    \left(
    \cos\frac{\Delta_n t}{2}
    -i\sin\frac{\Delta_n t}{2}
    \right)\sin^2{\theta_n}
    +\left(
    \cos\frac{\Delta_n t}{2}
    +i\sin\frac{\Delta_n t}{2}
    \right)\cos^2{\theta_n}
    \right\}\nn[10pt]
    %%%%%
    &=e^{
    -i
    \left(
    n+\frac{1}{2}
    \right)\omega_0t
    }
    \left\{
    \cos\frac{\Delta_n t}{2}
    \left(\sin^2{\theta_n}
    +\cos^2{\theta_n}
    \right)
    -\sin\frac{\Delta_n t}{2}
    \left(
    \cos^2{\theta_n}
    -i\sin^2{\theta_n}
    \right)
    \right\}
    \nn[10pt]
    %%%%%
    &=e^{
    -i
    \left(
    n+\frac{1}{2}
    \right)\omega_0t
    }
    \left\{
    \cos\frac{\Delta_n t}{2}
    -i\sin\frac{\Delta_n t}{2}
    \cos{2\theta_n}
    \right\}
\end{align}
次に
\begin{align}
    e^{-iE_+t/\hbar}
    -e^{-iE_-t/\hbar}
    &=\exp{
    \left\{
    -i
    \left(
    n+\frac{1}{2}
    \right)\omega_0t
    \right\}
    }
    \left(
    \cos\frac{\Delta_n t}{2}
    -i\sin\frac{\Delta_n t}{2}
    -
    \cos\frac{\Delta_n t}{2}
    -i\sin\frac{\Delta_n t}{2}
    \right)\nn[10pt]
    &=\exp{
    \left\{
    -i
    \left(
    n+\frac{1}{2}
    \right)\omega_0t
    \right\}
    }
    \left(
    -2i\sin\frac{\Delta_n t}{2}
    \right)
\end{align}
したがって,JC modelの解として次式を得る:
\begin{align}\label{Solve_Sch.eq_JC2}
    \ket{\Psi(t)}&=
    e^{
    -i
    \left(
    n+\frac{1}{2}
    \right)\omega_0t
    }
    \left[
    \left\{
    \cos\frac{\Delta_n t}{2}
    -i\sin\frac{\Delta_n t}{2}
    \cos{2\theta_n}
    \right\}\ket{e,n}
    +
    \left(
    -i\sin\frac{\Delta_n t}{2}
    \right)2\sin{\theta_n}\cos\theta_n\ket{g,n+1}
    \right]\nn[10pt]
    &=
    e^{
    -i
    \left(
    n+\frac{1}{2}
    \right)\omega_0t
    }
    \left[
    \left\{
    \cos\frac{\Delta_n t}{2}
    -i\sin\frac{\Delta_n t}{2}
    \cos{2\theta_n}
    \right\}\ket{e,n}
    -i\sin\frac{\Delta_n t}{2}
    \sin{2\theta_n}\ket{g,n+1}
    \right]
\end{align}
この解について考察を行う.まず,$\ket{g,n+1}$の展開係数に含まれる
\begin{equation}
    \sin{2\theta_n}=2\sin\theta_n\cos\theta_n
    =\frac{4g\sqrt{n+1}(\Delta_n-\delta)}{(\Delta_n-\delta)^2+4(n+1)g^2}
\end{equation}
に注目すると,Detuning$\delta$が十分大きいとき$(\delta\to\infty)$,
\begin{equation}
    \sin{2\theta_n}
    =\frac{4g\sqrt{n+1}(\Delta_n\delta-1)}{\delta(\Delta_n/\delta-1)^2+4(n+1)g^2/\delta}
    \to\mathcal{O}{(1/\delta)}
\end{equation}
となる.
つまり,$\delta=\omega-\omega_0$が大きくなると,原子は基底状態$\ket{g,n+1}$の状態へ遷移しにくくなることがわかる.


次にOn resonant$(\delta=0),\therefore\omega=\omega_0$の場合,
\begin{equation}
    \Delta_n=2g\sqrt{n+1},\ \ \ \sin{2\theta_n}=1,\ \ \ \cos{2\theta_n}=0
\end{equation}
となるので,系の時間発展は次のようになる:
\begin{align}\label{Solve_Sch.eq_JC2}
    \ket{\Psi(t)}
    &=
    e^{
    -i
    \left(
    n+\frac{1}{2}
    \right)\omega_0t
    }
    \left[
    \cos{(g\sqrt{n+1} t)}
    \ket{e,n}
    -i\sin{(g\sqrt{n+1} t)}\ket{g,n+1}
    \right]
\end{align}
ここから,時刻$t$に原子が第一励起状態に見いだされる確率$P_e(t)$および基底状態に見いだされる確率$P_g(t)$は次のように与えられる:
\begin{align}
    P_{e}(t)&=|\Braket{e,n|\Psi(t)}|^2
    =\cos^2{(g\sqrt{n+1} t)}\\[10pt]
    P_{g}(t)&=|\Braket{g,n+1|\Psi(t)}|^2
    =\sin^2{(g\sqrt{n+1} t)}.
\end{align}
また,時刻$t$における平均光子数は次のように与えられる:
\begin{align}
    \Braket{\hat{a}^\dagger\hat{a}}(t)
    &=\braket{\Psi(t)|\hat{a}^{\dagger}\hat{a}|\Psi(t)}\nn[10pt]
    &=
    (\cos{(g\sqrt{n+1} t)}
    \bra{e,n}
    +i\sin{(g\sqrt{n+1} t)}\bra{g,n+1})
    (\hat{a}^\dagger\hat{a})
    (\cos{(g\sqrt{n+1} t)}
    \ket{e,n}
    -i\sin{(g\sqrt{n+1} t)}\ket{g,n+1})\nn[10pt]
    &=
    (\cos^2{(g\sqrt{n+1} t)}
    \braket{e,n|\hat{a}^\dagger\hat{a}|e,n}
    -i\sin{(g\sqrt{n+1} t)}\cos{(g\sqrt{n+1} t)}
    \braket{e,n|\hat{a}^\dagger\hat{a}|g,n+1}\nn[10pt]
    %%%%%
    &-i\sin{(g\sqrt{n+1} t)}\cos{(g\sqrt{n+1} t)}
    \braket{g,n+1|\hat{a}^\dagger\hat{a}|e,n}
    +\sin^2{(g\sqrt{n+1} t)}\braket{g,n+1|\hat{a}^\dagger\hat{a}|g,n+1})\nn[10pt]
    &=
    n\cos^2{(g\sqrt{n+1} t)}
    +(n+1)\sin^2{(g\sqrt{n+1} t)}\nn[10pt]
    &=
    n
    +\sin^2{(g\sqrt{n+1} t)}
    =n+P_{g}(t)
\end{align}

このように,原子は決まった周波数
\begin{equation}
    \Omega_{\rm{Rabi}}=2g\sqrt{n+1}
\end{equation}
で光子の吸収と放出を繰り返す.これをRabi振動という.また,$\Delta_n$や$\Omega_{\rm{Rabi}}$はRabi周波数と呼ばれる.特に,電磁場の初期状態が真空状態の場合,すなわち,$n=0$の場合も原子の初期状態が第一励起状態であればRabi振動が起こることがわかる.
