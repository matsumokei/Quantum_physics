
\subsection{直交位相振幅(Quadrature phase amplitude)}



\section{スクイーズド状態 (Single-mode squeezed states)}
\subsection{スクイーズド Hamiltonianの対角化}
We discuss the eigenvalues and eigenvector of Hamiltonian
\begin{equation}
    \hH_{\rm{S}}=\hbar\omega\ha^\dag\ha + \hbar \left(\frac{E^\ast}{2}\ha^{2} + \frac{E}{2}\ha^{\dag2}\right).
\end{equation}
Bogoliubov transformation
\begin{align}
    \left( 
     \begin{array}{c}
     \hb\\[10pt]
     \hb^\dag
     \end{array}
    \right)
    =
    \left( 
     \begin{array}{cc}
     \xi_1&\xi_2 \\[10pt]
     \xi_2^{\ast}&\xi_1^{\ast}
     \end{array}
    \right)
    \left( 
     \begin{array}{c}
     \ha\\[10pt]
     \ha^\dag
     \end{array}
    \right)
\end{align}
この逆変換は
\begin{align}
    \left( 
     \begin{array}{c}
     \ha\\[10pt]
     \ha^\dag
     \end{array}
    \right)
    =
    \left( 
     \begin{array}{cc}
     \xi_1^{\ast}&-\xi_2 \\[10pt]
     -\xi_2^{\ast}&\xi_1
     \end{array}
    \right)
    \left( 
     \begin{array}{c}
     \hb\\[10pt]
     \hb^\dag
     \end{array}
    \right)
\end{align}

\begin{align}
    \ha&=\xi_1^{\ast}\hb-\xi_2\hb^\dag\\[10pt]
     \ha^\dag&=-\xi_2^{\ast}\hb+\xi_1\hb^\dag
\end{align}


\begin{align}
    \hH_{\rm{S}}&=\hbar\omega(-\xi_2^{\ast}\hb+\xi_1\hb^\dag)(\xi_1^{\ast}\hb-\xi_2\hb^\dag)\nn[10pt]
    &+ \hbar\frac{E^\ast}{2}(\xi_1^{\ast}\hb+-\xi_2\hb^\dag)(\xi_1^{\ast}\hb+-\xi_2\hb^\dag)\nn[10pt]
    &+ \hbar\frac{E}{2}(-\xi_2^{\ast}\hb+\xi_1\hb^\dag)(-\xi_2^{\ast}\hb+\xi_1\hb^\dag)
\end{align}

the first term
\begin{align}
    \hH_{\rm{S}1}/(\hbar\omega)&=(-\xi_2^{\ast}\hb+\xi_1\hb^\dag)(\xi_1^{\ast}\hb-\xi_2\hb^\dag)
    =-\xi_1^{\ast}\xi_2^{\ast}\hb^2 +|\xi_2|^2\hb\hb^\dag
    +|\xi_1|^2\hb^\dag\hb-\xi_1\xi_2\hb^{\dag2}\nn[10pt]
    &=|\xi_1|^2\hb^\dag\hb
    +|\xi_2|^2\hb^\dag\hb
    -\xi_1^{\ast}\xi_2^{\ast}\hb^2 -\xi_1\xi_2\hb^{\dag2}+|\xi_2|^2
\end{align}

\begin{align}
    \hH_{\rm{S}2}/(\hbar E^{\ast}/2)&=(\xi_1^{\ast}\hb-\xi_2\hb^\dag)(\xi_1^{\ast}\hb-\xi_2\hb^\dag)
    =-\xi_1^{\ast}\xi_2\hb^\dag\hb-\xi_1^{\ast}\xi_2\hb\hb^\dag
    +\xi_1^{\ast2}\hb^2+\xi_2^{2}\hb^{\dag2}\nn[10pt]
    &=-2\xi_1^{\ast}\xi_2\hb^\dag\hb
    +\xi_1^{\ast2}\hb^2+\xi_2^{2}\hb^{\dag2}-\xi_1^{\ast}\xi_2
\end{align}


\begin{align}
    \hH_{\rm{S}3}/(\hbar E/2)
    &=(-\xi_2^{\ast}\hb+\xi_1\hb^\dag)(-\xi_2^{\ast}\hb+\xi_1\hb^\dag)
    =-\xi_1\xi_2^{\ast}\hb^\dag\hb-\xi_1\xi_2^{\ast}\hb\hb^\dag
    +\xi_2^{\ast2}\hb^2+\xi_1^{2}\hb^{\dag2}\nn[10pt]
    &=-2\xi_1\xi_2^{\ast}\hb^\dag\hb
    +\xi_2^{\ast2}\hb^2+\xi_1^{2}\hb^{\dag2}-\xi_1\xi_2^{\ast}
\end{align}

\begin{align}
    \hH_{\rm{S}}&=\hbar\omega(|\xi_1|^2\hb^\dag\hb
    +|\xi_2|^2\hb^\dag\hb
    -\xi_1^{\ast}\xi_2^{\ast}\hb^2 -\xi_1\xi_2\hb^{\dag2}+|\xi_2|^2)\nn[10pt]
    &+ \hbar\frac{E^\ast}{2}(-2\xi_1^{\ast}\xi_2\hb^\dag\hb
    +\xi_1^{\ast2}\hb^2+\xi_2^{2}\hb^{\dag2}-\xi_1^{\ast}\xi_2)\nn[10pt]
    &+ \hbar\frac{E}{2}(-2\xi_1\xi_2^{\ast}\hb^\dag\hb
    +\xi_2^{\ast2}\hb^2+\xi_1^{2}\hb^{\dag2}-\xi_1\xi_2^{\ast})\nn[15pt]
    %
    &=\hbar\left\{\omega(|\xi_1|^2+|\xi_2|^2)\hb^\dag\hb
    +\frac{E^\ast}{2}(-2\xi_1^{\ast}\xi_2\hb^\dag\hb)
    +\frac{E}{2}(-2\xi_1\xi_2^{\ast}\hb^\dag\hb)
    \right\}
    \nn[10pt]
    &+\hbar\left\{\omega(-\xi_1^{\ast}\xi_2^{\ast}\hb^2)
    +\frac{E^\ast}{2}(\xi_1^{\ast2}\hb^2)
    +\frac{E}{2}(\xi_2^{\ast2}\hb^2)
    \right\}\nn[10pt]
    &+\hbar\left\{\omega(-\xi_1\xi_2\hb^{\dag2})
    +\frac{E^\ast}{2}(\xi_2^{2}\hb^{\dag2})
    +\frac{E}{2}(\xi_1^{2}\hb^{\dag2})
    \right\}\nn[10pt]
    &+\hbar\left\{\omega|\xi_2|^2
    +\frac{E^\ast}{2}(-\xi_1^{\ast}\xi_2)
    +\frac{E}{2}(-\xi_1\xi_2^{\ast})
    \right\}
\end{align}
このHaniltonianを対角化するためには,非対角項が
\begin{align}
\label{squeez_offdiag_1}
    -\omega\xi_1^{\ast}\xi_2^{\ast}
    +\frac{E^\ast}{2}\xi_1^{\ast2}
    +\frac{E}{2}\xi_2^{\ast2}&=0\\[10pt]
    %
\label{squeez_offdiag_2}
    -\omega\xi_1\xi_2
    +\frac{E^\ast}{2}\xi_2^{2}
    +\frac{E}{2}\xi_1^{2}&=0
\end{align}
を満たすような$\xi_1$,$\xi_2$を満たす必要がある.\eqref{squeez_offdiag_1}の両辺に$E^\ast/\xi_1^2$をかけると,$\xi_2/\xi_1$に関する2次方程式となる:
\begin{align}
    E^{\ast2}\left(\frac{\xi_2}{\xi_1}\right)^{2}-2\omega E^{\ast}\frac{\xi_2}{\xi_1}
    +|E|^2
    &=0.
\end{align}
これを解くと
\begin{align}
    \frac{\xi_2}{\xi_1}&=\frac{\omega E^{\ast}\pm\sqrt{\omega^2E^{\ast2}-E^{\ast2}|E|^2}}{E^{\ast2}}\nn[10pt]
    &=\frac{\omega \pm\sqrt{\omega^2-|E|^2}}{E^{\ast}}
    =\frac{\omega \pm\lambda}{E^{\ast}},
\end{align}
where $\lambda=\sqrt{\omega^2-|E|^2}$を得る.2次方程式の解の公式を用いた:
\begin{align}
    ax^2+bx+c=0,\\[10pt]
    x=\frac{-b\pm\sqrt{b^2-4ac}}{2a}.
\end{align}
同様にして,\eqref{squeez_offdiag_2}の両辺に$E/\xi_1^{\ast2}$をかけると
\begin{align}
    E^{2}\left(\frac{\xi_2^{\ast}}{\xi_1^{\ast}}\right)^{2}-2\omega E\frac{\xi_2^{\ast}}{\xi_1^{\ast}}
    +|E|^2
    &=0
\end{align}
\begin{align}
    \frac{\xi_2^{\ast}}{\xi_1^{\ast}}
    &=\frac{\omega \pm\sqrt{\omega^2-|E|^2}}{E}
    =\frac{\omega \pm\lambda}{E}
\end{align}
where $\lambda=\sqrt{\omega^2-|E|^2}$.よって,
\begin{align}
    \frac{|\xi_2|^2}{|\xi_1|^2}
    &=\frac{(\omega \pm\lambda)^2}{|E|^2}
    =\frac{-(\lambda\pm\omega)^2}{(\lambda^2-\omega^2)}
    =\frac{-(\lambda\pm\omega)}{(\lambda\mp\omega)}
\end{align}
\begin{align}
    -\frac{|\xi_2|^2}{|\xi_1|^2}
    =\frac{(\lambda\pm\omega)}{(\lambda\mp\omega)}
\end{align}
を得る.Using $|\xi_1|^2-|\xi_2|^2=1$, we obtain
\begin{align}
    \frac{1}{|\xi_1|^2}&=1+\left(-\frac{|\xi_2|^2}{|\xi_1|^2}\right)
    =1+\frac{(\lambda\pm\omega)}{(\lambda\mp\omega)}=\frac{2\lambda}{\lambda\mp\omega}
\end{align}
Therefore we have
\begin{align}
    |\xi_1|^2&=\frac{\lambda\mp\omega}{2\lambda}\\[10pt]
    |\xi_2|^2&=\frac{\lambda\mp\omega}{2\lambda}-1=\frac{\lambda\mp\omega-2\lambda}{2\lambda}
    =\frac{-\lambda\mp\omega}{2\lambda}.
\end{align}


ここで,各項は,
\begin{align}
    \hH_{\rm{S}}
    &=\hbar\left\{\omega(|\xi_1|^2+|\xi_2|^2)\hb^\dag\hb
    +\frac{E^\ast}{2}(-2\xi_1^{\ast}\xi_2\hb^\dag\hb)
    +\frac{E}{2}(-2\xi_1\xi_2^{\ast}\hb^\dag\hb)
    \right\}
\end{align}




\begin{align}
    {\mathrm{1st\ term}}
    &=|\xi_1|^2+|\xi_2|^2=\frac{\lambda\mp\omega}{2\lambda}+\frac{-\lambda\mp\omega}{2\lambda}
    =\frac{\mp2\omega}{2\lambda}=\mp\frac{\omega}{\lambda}
\end{align}






\begin{align}
    {\mathrm{2nd\ term}}
    %
    &=\frac{E^\ast}{2}(-2\xi_1^{\ast}\xi_2)
    =-E^\ast\xi_1^{\ast}\xi_2
    =-E^\ast|\xi_1|^2\frac{\xi_2}{\xi_1}
    =-E^\ast\frac{\lambda\mp\omega}{2\lambda}\frac{\omega\pm\lambda}{E^\ast}\nn[10pt]
    &=\frac{(\pm\omega-\lambda)(\omega\pm\lambda)}{2\lambda}
    =\frac{\omega^2-\lambda^2}{2\lambda}=\frac{|E|^2}{2\lambda}
\end{align}

\begin{align}
    {\mathrm{3rd\ term}}
    %
    &=\frac{E}{2}(-2\xi_1\xi_2^{\ast})
    =-E|\xi_1|^2\frac{\xi_2^{\ast}}{\xi_1^{\ast}}
    =-E \frac{\lambda\mp\omega}{2\lambda}\frac{\omega\pm\lambda}{E}\nn[10pt]
    &=\frac{(\pm\omega-\lambda)(\omega\pm\lambda)}{2\lambda}
    =\frac{\omega^2-\lambda^2}{2\lambda}=\frac{|E|^2}{2\lambda}
\end{align}
と計算できるので,Squeezed Hamiltonianは次のように対角化できる:
\begin{align}
    \hH_{\rm{S}}
    &=\hbar\left\{\mp\frac{\omega^2}{\lambda}
    \pm\frac{|E|^2}{2\lambda}
    \pm\frac{|E|^2}{2\lambda}
    \right\}\hb^\dag\hb
    %
    =\hbar\left\{\frac{\mp\omega^2\pm|E|^2}{\lambda}
    \right\}\hb^\dag\hb
    =\mp\hbar\lambda\hb^\dag\hb
\end{align}

\subsection{スクイーズド状態の演算子形式}
スクイーズ演算子を
\begin{equation}
    \hat{S}(\zeta)\equiv
    \exp{
    \left\{
    \frac{1}{2}(\zeta^\ast\hat{a}^2-\zeta\hat{a}^{\dagger 2})
    \right\}
    },\ \ \ \zeta = re^{i\varphi}
\end{equation}
により導入する.定義より
\begin{equation}
    \hat{S}^\dagger(\zeta)=\hat{S}(-\zeta)=\hat{S}^{-1}(\zeta)
\end{equation}
が確認できる.つまりスクイーズ演算子はユニタリ演算子であることがわかる.

一般スクイーズ状態はスクイーズ演算子$\hat{S}(\zeta)$を用いて,以下のように定義される:
\begin{equation}
    \ket{\zeta,\alpha}\equiv\hat{S}(\zeta)\ket{\alpha}
    =\hat{S}(\zeta)\hat{D}(\alpha)\ket{0}
\end{equation}


また,スクイーズ演算子による生成消滅演算子$\hat{a}^\dagger$, $\hat{a}$のユニタリ変換は
\begin{align}
    \hat{S}^\dagger(\zeta)\hat{a}\hat{S}(\zeta)&=
    \hat{a}\cosh{r} - \hat{a}^\dagger e^{i\varphi}\sinh{r}\\[10pt]
    \hat{S}^\dagger(\zeta)\hat{a}^{\dagger}\hat{S}(\zeta)&=
    \hat{a}^\dagger\cosh{r} - \hat{a} e^{-i\varphi}\sinh{r}
\end{align}
同様にして,
\begin{align}
    \hat{S}(\zeta)\hat{a}\hat{S}^\dagger(\zeta)&=
    \hat{a}\cosh{r} + \hat{a}^\dagger e^{i\varphi}\sinh{r}\\[10pt]
    \hat{S}(\zeta)\hat{a}^{\dagger}\hat{S}^\dagger(\zeta)&=
    \hat{a}^\dagger\cosh{r} + \hat{a} e^{-i\varphi}\sinh{r}
\end{align}


\begin{align}
    \hat{S}^\dagger(\zeta)\hat{a}\hat{S}(\zeta)
    &=e^{
    (\zeta\hat{a}^{\dagger 2}-\zeta^\ast\hat{a}^2)/2
    }\ 
    \hat{a}\ 
    e^{
    (\zeta^\ast\hat{a}^2-\zeta\hat{a}^{\dagger 2})/2
    }\nn[10pt]
    &=\hat{a}
    +\frac{1}{1!}
    \hat{C}^{(1)}
    +\frac{1}{2!}\hat{C}^{2}+\frac{1}{3!}\hat{C}^{(3)}+\cdots
\end{align}
ここで,入れ子になっている交換子を次のように定義した:
\begin{align}
    \hat{C}^{(1)}&\equiv[(\zeta\hat{a}^{\dagger 2}-\zeta^\ast\hat{a}^2)/2, \hat{a}]\\[10pt]
    \hat{C}^{(2)}&\equiv\Bigl[(\zeta\hat{a}^{\dagger 2}-\zeta^\ast\hat{a}^2)/2, 
    \hat{C}^{(1)}
    \Bigr]
    =\Bigl[(\zeta\hat{a}^{\dagger 2}-\zeta^\ast\hat{a}^2)/2, 
    [(\zeta\hat{a}^{\dagger 2}-\zeta^\ast\hat{a}^2)/2, \hat{a}]
    \Bigr]\\[10pt]
    %
    \hat{C}^{(3)}&\equiv\Bigl[(\zeta\hat{a}^{\dagger 2}-\zeta^\ast\hat{a}^2)/2, 
    \hat{C}^{(2)}
    \Bigr]
    =\biggl[(\zeta\hat{a}^{\dagger 2}-\zeta^\ast\hat{a}^2)/2, 
    \Bigl[(\zeta\hat{a}^{\dagger 2}-\zeta^\ast\hat{a}^2)/2, 
    [(\zeta\hat{a}^{\dagger 2}-\zeta^\ast\hat{a}^2)/2, \hat{a}]
    \Bigr]\biggr]\\[10pt]
    \vdots
\end{align}

次に交換子$\hat{C}^{(n)}$, $n=1,2,3,\ldots$の計算を実行する:
\begin{align}
    \hat{C}^{(1)}&\equiv[(\zeta\hat{a}^{\dagger 2}-\zeta^\ast\hat{a}^2)/2, \hat{a}]
    =\frac{1}{2}(
    \zeta\hat{a}^{\dagger 2}\hat{a}-\zeta^\ast\hat{a}^3
    -\zeta\hat{a}\hat{a}^{\dagger 2}+\zeta^\ast\hat{a}^3
    )
    =\frac{1}{2}(
    \zeta\hat{a}^{\dagger 2}\hat{a}-\zeta\hat{a}\hat{a}^{\dagger}\hat{a}^{\dagger}
    )\nn[10pt]
    &=\frac{\zeta}{2}(
    \hat{a}^{\dagger 2}\hat{a}-(\hat{a}^{\dagger}\hat{a}+1)\hat{a}^{\dagger}
    )
    =\frac{\zeta}{2}(
    \hat{a}^{\dagger 2}\hat{a}-\hat{a}^{\dagger}\hat{a}\hat{a}^{\dagger}-\hat{a}^{\dagger}
    )
    =\frac{\zeta}{2}(
    \hat{a}^{\dagger 2}\hat{a}-\hat{a}^{\dagger}(\hat{a}^{\dagger}\hat{a}+1)-\hat{a}^{\dagger}
    )
    \nn[10pt]
    &=\frac{\zeta}{2}(
    \hat{a}^{\dagger 2}\hat{a}-\hat{a}^{\dagger}\hat{a}^{\dagger}\hat{a}-2\hat{a}^{\dagger}
    )
    =-\zeta\hat{a}^{\dagger}
\end{align}

\begin{align}
    \hat{C}^{(2)}&\equiv\Bigl[(\zeta\hat{a}^{\dagger 2}-\zeta^\ast\hat{a}^2)/2, 
    \hat{C}^{(1)}
    \Bigr]
    =\Bigl[(\zeta\hat{a}^{\dagger 2}-\zeta^\ast\hat{a}^2)/2, 
    -\zeta\hat{a}^{\dagger}
    \Bigr]\nn[10pt]
    &=-\frac{1}{2}(
    \zeta^2\hat{a}^{\dagger 3}-|\zeta|^2\hat{a}^2\hat{a}^{\dagger}
    -\zeta^2\hat{a}^{\dagger 3}+|\zeta|^2\hat{a}^{\dagger}\hat{a}^2
    )
    =-\frac{|\zeta|^2}{2}(
    \hat{a}^{\dagger}\hat{a}^2-\hat{a}^2\hat{a}^{\dagger}
    )\nn[10pt]
    &=-\frac{|\zeta|^2}{2}(
    \hat{a}^{\dagger}\hat{a}^2-\hat{a}^2\hat{a}^{\dagger}
    )
    =-\frac{|\zeta|^2}{2}(
    \hat{a}^{\dagger}\hat{a}^2-\hat{a}^{\dagger}\hat{a}^2-2\hat{a}
    )
    \nn[10pt]
    &=|\zeta|^2\hat{a}
\end{align}


\begin{align}
    \hat{C}^{(3)}&\equiv\Bigl[(\zeta\hat{a}^{\dagger 2}-\zeta^\ast\hat{a}^2)/2, 
    \hat{C}^{(2)}
    \Bigr]
    =\Bigl[(\zeta\hat{a}^{\dagger 2}-\zeta^\ast\hat{a}^2)/2, 
    |\zeta|^2\hat{a}
    \Bigr]\nn[10pt]
    &=\frac{1}{2}(
    \zeta|\zeta|^2\hat{a}^{\dagger 2}\hat{a}-\zeta^\ast|\zeta|^2\hat{a}^3
    -\zeta|\zeta|^2\hat{a}\hat{a}^{\dagger 2}+\zeta^\ast|\zeta|^2\hat{a}^3
    )
    =\frac{\zeta|\zeta|^2}{2}(
    \hat{a}^{\dagger 2}\hat{a}-\hat{a}\hat{a}^{\dagger 2}
    )\nn[10pt]
    &=\frac{\zeta|\zeta|^2}{2}(
    \hat{a}^{\dagger 2}\hat{a}-\hat{a}^{\dagger 2}\hat{a}-2\hat{a}^\dagger
    )
    \nn[10pt]
    &=-\zeta|\zeta|^2\hat{a}^\dagger
\end{align}

\begin{align}
    \hat{C}^{(4)}&\equiv\Bigl[(\zeta\hat{a}^{\dagger 2}-\zeta^\ast\hat{a}^2)/2, 
    \hat{C}^{(3)}
    \Bigr]
    =\Bigl[(\zeta\hat{a}^{\dagger 2}-\zeta^\ast\hat{a}^2)/2, 
    -\zeta|\zeta|^2\hat{a}^\dagger
    \Bigr]\nn[10pt]
    &=-\frac{1}{2}(
    \zeta^2|\zeta|^2\hat{a}^{\dagger 2}\hat{a}^{\dagger}-|\zeta|^4\hat{a}^2\hat{a}^{\dagger}
    -\zeta^2|\zeta|^2\hat{a}^{\dagger}\hat{a}^{\dagger 2}+|\zeta|^4\hat{a}^{\dagger}\hat{a}^2
    )
    =-\frac{|\zeta|^4}{2}(
    \hat{a}^{\dagger}\hat{a}^2-\hat{a}^2\hat{a}^{\dagger}
    )\nn[10pt]
    &=-\frac{|\zeta|^4}{2}(
    \hat{a}^{\dagger}\hat{a}^2-\hat{a}^{\dagger}\hat{a}^2-2\hat{a}
    )
    \nn[10pt]
    &=|\zeta|^4\hat{a}
\end{align}


\begin{align}
    \hat{C}^{(5)}&\equiv\Bigl[(\zeta\hat{a}^{\dagger 2}-\zeta^\ast\hat{a}^2)/2, 
    \hat{C}^{(4)}
    \Bigr]
    =\Bigl[(\zeta\hat{a}^{\dagger 2}-\zeta^\ast\hat{a}^2)/2, 
    |\zeta|^4\hat{a}
    \Bigr]\nn[10pt]
    &=\frac{1}{2}(
    \zeta|\zeta|^4\hat{a}^{\dagger 2}\hat{a}-\zeta^\ast|\zeta|^4\hat{a}^3
    -\zeta|\zeta|^4\hat{a}\hat{a}^{\dagger 2}+\zeta^\ast|\zeta|^4\hat{a}^3
    )
    =\frac{\zeta|\zeta|^4}{2}(
    \hat{a}^{\dagger 2}\hat{a}-\hat{a}\hat{a}^{\dagger 2}
    )\nn[10pt]
    &=\frac{\zeta|\zeta|^4}{2}(
    \hat{a}^{\dagger 2}\hat{a}-\hat{a}^{\dagger 2}\hat{a}-2\hat{a}^\dagger
    )
    \nn[10pt]
    &=-\zeta|\zeta|^4\hat{a}^\dagger
\end{align}

したがって,一般形について
\begin{align}
    \hat{C}^{(n)}
    =\left\{
    \begin{array}{l}
    -\zeta|\zeta|^{n-1}\hat{a}^{\dagger},\ \ n=\rm{odd} \\[10pt]
    |\zeta|^{n}\hat{a},\ \ n=\rm{even}
    \end{array}
    \right.
\end{align}
が成り立つことがわかる.


\begin{align}
    \hat{S}^\dagger(\zeta)\hat{a}\hat{S}(\zeta)
    &=e^{
    (\zeta\hat{a}^{\dagger 2}-\zeta^\ast\hat{a}^2)/2
    }\ 
    \hat{a}\ 
    e^{
    (\zeta^\ast\hat{a}^2-\zeta\hat{a}^{\dagger 2})/2
    }\nn[10pt]
    &=\hat{a}
    -\frac{1}{1!}
    \zeta\hat{a}^{\dagger}
    +\frac{1}{2!}
    |\zeta|^{2}\hat{a}
    -\frac{1}{3!}
    \zeta|\zeta|^2\hat{a}^{\dagger}
    +\frac{1}{4!}
    |\zeta|^{4}\hat{a}
    -\frac{1}{5!}
    \zeta|\zeta|^4\hat{a}^{\dagger}
    +\cdots\nn[10pt]
    &=\hat{a}
    \left(
    1+\frac{1}{2!}
    |\zeta|^{2}+\frac{1}{4!}
    |\zeta|^{4}+\cdots
    \right)
    -\zeta\hat{a}^{\dagger}
    \left(
    \frac{1}{1!}
    |\zeta|^{0}
    +\frac{1}{3!}
    |\zeta|^2
    +\frac{1}{5!}
    |\zeta|^4
    +\cdots
    \right)\nn[10pt]
    %
    &=\hat{a}
    \left(
    1+\frac{1}{2!}r^2
    +\frac{1}{4!}
    r^{4}+\cdots
    \right)
    -\hat{a}^{\dagger}e^{i\varphi}
    \left(
    \frac{1}{1!}r
    +\frac{1}{3!}
    r^3
    +\frac{1}{5!}
    r^5
    +\cdots
    \right)\nn[10pt]
    &=\hat{a}\ 
    {\rm{cosh}}r
    -\hat{a}^{\dagger}e^{i\varphi}
    {\rm{sinh}}r
\end{align}

$\zeta=re^{i\varphi}$

\begin{align}
    {\rm{sinh}}x &= \frac{x}{1!}+\frac{x^3}{3!}+\frac{x^5}{5!}+\cdots\\[10pt]
    {\rm{cosh}}x &= 1+\frac{x^2}{2!}+\frac{x^4}{4!}+\cdots
\end{align}

