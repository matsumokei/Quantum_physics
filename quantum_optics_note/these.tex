\documentclass[dvipdfmx,autodetect-engine,10.5pt]{jsarticle}
\usepackage[utf8]{inputenc}

\usepackage{amsmath,amsfonts,amssymb}
\usepackage{graphicx}
\usepackage[dvipdfmx]{hyperref}
\usepackage{pxjahyper}%these two come together
\usepackage[dvipdfmx]{color}
\usepackage{braket}%dirac notation
\usepackage{wrapfig}
\usepackage{here}
\usepackage{tabularx, dcolumn}
\usepackage{subfigure}
\usepackage{cases}
\usepackage{bigints}%インテグラルで大きくする
\usepackage{mathtools} 
\hypersetup{hidelinks}
\interfootnotelinepenalty=10000 % this is to keep a footnote in a single page
\usepackage{bm}%ベクトル記号
\usepackage{ascmac} %囲い
%%%%%%
\usepackage{tikz}
\usepackage{amsmath}
\usepackage{cases}%連立方程式


%%%%%%newcomand
\newcommand{\be}{\begin{equation}}
\newcommand{\ee}{\end{equation}}
\newcommand{\nn}{\notag \\}

\usepackage{mdframed}%ページをまたぐ    
\newmdenv[skipabove=6mm, skipbelow=4mm]{kotak}


\newtheorem{definition}{定義}[section]
\newtheorem{theorem}[definition]{定理}
\newtheorem{proof}{証明}
\newtheorem{request}[definition]{要請}
\newtheorem{prop}[definition]{命題}
\newtheorem{these}[definition]{仮定}
\newtheorem{lemma}[definition]{補題}
\newtheorem{postlate}[definition]{公理}

%sectionの大きさを変更する
%\usepackage[explicit]{titlesec}

%operator
\newcommand{\hH}{{\hat{H}}}%ハミルトニアン
\newcommand{\hHt}{{\hat{\mathcal{H}}}}%ハミルトニアン
\newcommand{\hU}{{\hat{U}}}
\newcommand{\hM}{{\hat{M}}}
\newcommand{\hN}{{\hat{N}}}
\newcommand{\hA}{{\hat{A}}}
\newcommand{\hB}{{\hat{B}}}
\newcommand{\hO}{{\hat{O}}}
\newcommand{\hAd}{{\hat{A}^\dag}}
\newcommand{\ha}{{\hat{a}}}
\newcommand{\hb}{{\hat{b}}}
\newcommand{\had}{{\hat{a}^\dag}}
\newcommand{\hpsi}{{\hat{\psi}}}
\newcommand{\hpsid}{{\hat{\psi}^\dag}}
\newcommand{\hrho}{{\hat{\rho}}}
\newcommand{\hsig}{{\hat{\sigma}}}
\newcommand{\hx}{{\hat{x}}}
\newcommand{\hy}{{\hat{y}}}
\newcommand{\hz}{{\hat{z}}}
\newcommand{\hX}{{\hat{X}}}
\newcommand{\hY}{{\hat{Y}}}
\newcommand{\hZ}{{\hat{Z}}}
\newcommand{\hp}{{\hat{p}}}
\newcommand{\hvp}{{\hat{\bm p}}}
\newcommand{\dv}{{\nabla\cdot}} 
\newcommand{\rot}{{\nabla\times}}

%%%%%%%%%%%%%%%%%%%%%%%%%%%%%%%%%%%%%%
%%%%%%%%%%%%%%%%%%%%%%%%%%%%%%%%
% USER SPECIFIED COMMANDS
%%%%%%%%%%%%%%%%%%%%%%%%%%%%%%%%
\newcommand{\ie}{i.e.}
\newcommand{\eg}{e.g.}
\newcommand{\etal}{\textit{et al.}}
\newcommand{\e}{\textrm{e}} % Exponential
\newcommand{\hc}{\text{h.c.}} % Hermitian conjugate

\newcommand{\nbar}{\bar{n}}
\newcommand{\adag}{\hat{a}^\dagger}
\newcommand{\adagsq}{\hat{a}^{\dagger 2}}
\newcommand{\hata}{\hat{a}}

\newcommand{\jg}[1]{{\color{orange}#1}}
\newcommand{\dr}[1]{{\color{red}#1}}
\newcommand{\rg}[1]{{\color{MidnightBlue}#1}}
\newcommand{\filler}[1][1]{{\color{gray}\lipsum[1-#1]}}
%%%%%%%%%%%%%%%%%%%%%%%%%%%%%%%%%%%%%

%\vector
\newcommand{\vr}{{\bm{r}}} %vector r
\newcommand{\vP}{{\bm{P}}} %vector r
\newcommand{\vphi}{{\varphi(t,\bm{r})}}

%

\newcommand{\tr}{\mathrm{Tr}}
\newcommand{\diag}{\mathrm{diag}}
\newcommand{\rint}{\mathrm{int}}
\newcommand{\tot}{\mathrm{tot}}


\newcommand{\YM}[1]{\textcolor[rgb]{1, 0.1, 0.1}{#1}}
\newcommand{\YMdel}[1]{\sout{#1}}
%\newcommand{\YMdel}[1]{\textcolor[rgb]{1, 0.1, 0.1}{\sout{\textcolor{black}{#1}}}}
\newcommand{\KM}[1]{\textcolor[rgb]{0.1, 0.1, 1}{#1}}
\newcommand{\KMdel}[1]{\textcolor[rgb]{0.1, 0.1, 0.9}{\sout{\textcolor{black}{#1}}}}


\makeatletter
\title{Quantum Optics}



\begin{document}
\maketitle


\tableofcontents
%目の保護用
%\pagecolor{black}
%\color{white}
%%%%%%%%%%%%%%%%%%%%%

% \part{古典電磁気学}
\section{Maxwell方程式}
電場$\bm{E}$と磁場$\bm{B}$で記述されたMaxwellの方程式に対して,ベクトルポテンシャルとスカラーポテンシャルを導入し,Maxwellの方程式を書き直すのが本書の目的である.

電荷分布$\rho(\vr,t)$と電流分布${\bm{i}}(\vr,t)$とが与えられ,電磁場$\bm{E}(\vr,t)$,$\bm{B}(\vr,t)$が時間,空間的に変化する場合を考える.次に示す4組の方程式
\begin{align}
\label{mx1}
\nabla\times\bm{E}(\vr,t)+\dfrac{\partial\bm{B}(\vr,t)}{\partial t}=\bm 0
\end{align}

\begin{align}
\label{mx2}
\nabla\cdot\bm{B}(\vr,t)=0
\end{align}

\begin{align}
\label{mx3}
   \nabla\times\bm{B}(\vr,t)-\epsilon_0\mu_0\dfrac{\partial\bm{E}(\vr,t)}{\partial t}
      =\mu_0{\bm{i}}(\vr,t)
\end{align}

\begin{align}
\label{mx4}
   \nabla\cdot\bm{E}(\vr,t)
      =\frac{\rho(\vr,t)}{\epsilon_0}
\end{align}
をMaxwellの方程式という.これらは,電磁場$\bm{E}$,$\bm{B}$の時間と空間の変化を記述するもっとも基本的な
法則である.上式において,$\epsilon_0$は真空の誘電率,$\mu_0$は真空の透磁率である.物質中の電磁場を考えているわけではないので$\epsilon_0$,$\mu_0$を用いている.\\





%
\subsection{方程式と解の個数}
$\bm{E}$と$\bm{B}$の6個の未知数を求めるのに対して,Maxwellの方程式(\ref{mx1})$\sim$(\ref{mx4})の数は8個である.求める解に対して条件式が多い.そのため,解が存在しないのではないかという心配がある.その心配は無用であることをこの節では論じる.
(\ref{mx1})の発散をとると
\begin{align}
\dv\Bigl(\nabla\times\bm{E}(\vr,t)\Bigr)+\dv\dfrac{\partial\bm{B}(\vr,t)}{\partial t}=\bm 0
\end{align}
第1項は恒等的に$0$となるから,
\begin{align}
\dv\dfrac{\partial\bm{B}(\vr,t)}{\partial t}=\bm 0
\end{align}
が得られる.また,空間微分と時間微分の順序は
\begin{align}
\dfrac{\partial^2 }{\partial \alpha\partial t}=\dfrac{\partial^2 }{\partial t\partial \alpha},\ \ \ \ \ \alpha=x,y,z
\end{align}
と交換できるので,
\begin{align}\label{g1}
\dfrac{\partial}{\partial t}\dv\bm{B}(\vr,t)=\bm 0
\end{align}
となる.つまり,$\dv\bm{B}(\vr,t)$は時間的に一定である.したがって,初期時刻$t=0$において,
\begin{align}\label{g2}
\dv\bm{B}(\vr,t=0)=\bm 0
\end{align}
がみたされているならば,任意の時刻に対してみたされる.\\
%
%
 今度は,(\ref{mx3})に対して,発散をとると
\begin{align}
\dv\Bigl(\nabla\times\bm{B}(\vr,t)\Bigr)-\epsilon_0\mu_0\dv\dfrac{\partial\bm{E}(\vr,t)}{\partial t}
      =\mu_0\dv{\bm{i}}(\vr,t)
\end{align}
左辺第1項は恒等的に$0$になるので
\begin{align}
\label{g3}
\epsilon_0\mu_0\dv\dfrac{\partial\bm{E}(\vr,t)}{\partial t}+\mu_0\dv{\bm{i}}(\vr,t)=0
\end{align}
となる.(\ref{g3})左辺第2項に対して,電荷保存測
\begin{align}
\label{dc}
\dv{\bm{i}}(\vr,t)=-\dfrac{\partial\rho(\vr,t)}{\partial t}
\end{align}
を適用し,整理すると
\begin{align}
\label{g4}
\dfrac{\partial}{\partial t}\left(\dv\bm{E}(\vr,t)-\frac{\rho(\vr,t)}{\epsilon_0}\right)=0
\end{align}
が得られる.つまり,$\dv\bm{E}(\vr,t)-\dfrac{\rho(\vr,t)}{\epsilon_0}$は時間的に一定である.したがって,初期時刻$t=0$において,
\begin{align}\label{g5}
\dv\bm{E}(\vr,t=0)=\frac{\rho(\vr,t=0)}{\epsilon_0}
\end{align}
がみたされているならば,任意の時刻に対してみたされる.\\
 上で論じたことから,静電磁磁場であろうと時間変動する電磁場だろうと,必ず磁場は性質(\ref{mx2})を持ち,電場は性質(\ref{mx4})を持っているということである.つまり,(\ref{mx2})と(\ref{mx4})は電磁場に対する初期条件として要求されるにすぎないということである.このことから,電場$\bm{E}$と磁場$\bm{B}$の時間発展は(\ref{mx1})と(\ref{mx3})の6個の方程式によって決定される.















%
\subsection{線積分とStokesの定理}
ここで,線積分と重要な定理であるStokesの定理について説明しておく.
\subsubsection{ベクトル場の線積分}
ベクトル場$\bm{C}(\vr)=(C_x(\vr),C_y(\vr),C_z(\vr))$が領域$D$で定義されているとする.$D$内に曲線$L$と,$L$上に2点$P$,$Q$がある.$L$上の点は,$t$をパラメータとして位置ベクトル$\vr=\vr(t)=(x(t),y(t),z(t))$のように表されているものとする.また,$t=t_0$のとき点$P$,$t=t_1$のとき点$Q$であるとする.\\
 曲線$L$に沿った,点$P$から点$Q$までのベクトル場$\bm{C}$の線積分は
\begin{align}\label{sen1}
\int_L\bm{C}(\vr)\cdot d\vr=\int_P^Q\bm{C}(\vr)\cdot d\vr=\int_{t_0}^{t_1}\bm{C}(\vr)\cdot \frac{d\vr}{dt}dt
\end{align}
と定義される.\\
 領域$D$で定義されるスカラー場$f=f(\vr)=f(x,y,z)$に対して,
\begin{align}\label{chain}
\frac{df}{dt}=\dfrac{\partial f}{\partial x}\frac{dx}{dt}
+\dfrac{\partial f}{\partial y}\frac{dy}{dt}
+\dfrac{\partial f}{\partial z}\frac{dz}{dt}
\end{align}
が成り立つ.ここで,$t\to\vr\to f$という関係から,$f$の引数として$t$をとって,$f(t)$と表せることに注意したい.もしも,ベクトル場$\bm{C}(\vr)$がスカラー場$f(\vr)$のグラディエント$\nabla f(\vr)$を用いて
\begin{align}\label{nab}
\bm{C}=\nabla f(\vr)
=\left(
\dfrac{\partial f}{\partial x},
\dfrac{\partial f}{\partial y},
\dfrac{\partial f}{\partial z}
\right)
\end{align}
と表されているとき,線積分の定義(\ref{sen1})から
\begin{align}\label{sen2}
\int_L\bm{C}(\vr)\cdot d\vr&=\int_L\nabla f(\vr)\cdot d\vr=\int_P^Q\nabla f(\vr)\cdot d\vr\notag\\[10pt]
&=\int_{t_0}^{t_1}\left(
\dfrac{\partial f}{\partial x},
\dfrac{\partial f}{\partial y},
\dfrac{\partial f}{\partial z}
\right)\cdot \frac{d\vr}{dt}dt\notag\\[10pt]
%
%
&=\int_{t_0}^{t_1}\left(\dfrac{\partial f}{\partial x}\frac{dx}{dt}
+\dfrac{\partial f}{\partial y}\frac{dy}{dt}
+\dfrac{\partial f}{\partial z}\frac{dz}{dt}\right)dt\notag\\[10pt]
%
&=\int_{t_0}^{t_1}\frac{df}{dt}dt=\int_{f(t=t_0)}^{f(t=t_1)}df=\Bigl[f\Bigr]_{f(t_0)}^{f(t_1)}\notag\\[10pt]
&=f(t_1)-f(t_0)
\end{align}
で成り立つ.ここで,$f(P)$を点$P$での$f$の値,$f(Q)$を点$Q$での$f$の値とする.つまり,$f(P)=f(t=t_0)$,$f(Q)=f(t=t_1)$とすると,
\begin{align}\label{sen3}
\int_L\bm{C}(\vr)\cdot d\vr&=\int_L\nabla f(\vr)\cdot d\vr=\int_P^Q\nabla f(\vr)\cdot d\vr=f(Q)-f(P)
\end{align}
が成り立つ.とくに,$L$が$D$内の閉曲線,すなわち,始点$P$と終点$Q$が一致するならば,(\ref{sen3})において,$f(Q)=f(P)$となるから,
\begin{align}\label{sen4}
\oint_L\bm{C}(\vr)\cdot d\vr&=\oint_L\nabla f(\vr)\cdot d\vr=0
\end{align}
が成り立つ.ここで閉曲線$L$を一周するような,ベクトル場$\bm{C}(\vr)$の線積分を$\displaystyle\oint_L\bm{C}(\vr)\cdot\vr$と書き,周回積分という.この結果を定理としてまとめておくと次のようになる.
\begin{theorem}\label{thm1}
もしも領域$D$でベクトル場$\bm{C}(\vr)=\nabla f(\vr)$が成り立つとするならば,領域$D$内の2点$P$,$Q$を結ぶ曲線$L$に沿って,
\begin{align}
\int_L\nabla f(\vr)\cdot d\vr=f(Q)-f(P)
\end{align}
である.とくに$L$が閉曲線ならば,
\begin{align}
\oint_L\nabla f(\vr)\cdot d\vr=0
\end{align}
である.
\end{theorem}
定理\ref{thm1}を簡潔に表すと
\begin{align}
\bm{C}(\vr)=\nabla f(\vr)\Rightarrow \oint_L\nabla f(\vr)\cdot d\vr=0
\end{align}
である.









%
\subsubsection{Stokesの定理}
重要なStokesの定理について説明する.Stokesの定理は,閉曲面におけるベクトル場の回転の面積分を閉曲線上での元のベクトル場の線積分へと変換する手続きである.
\begin{theorem}[Stokesの定理]\label{sto}
$L$を空間内の曲面上の閉曲線,$S$を$L$によって囲まれた閉曲面とする.このとき,ベクトル場$\bm{C}(\vr)$に対して,
\begin{align}
\int_S\Bigl(\nabla\times\bm{C}(\vr)\Bigr)\cdot\bm{n}dS=\oint_L\bm{C}(\vr)\cdot d\vr
\end{align}
が成り立つ.これを$\rm{Stokes}$の定理という.
\end{theorem}
ここで,$\bm{n}$は閉曲面$S$の単位法線ベクトル,$dS$は閉曲面$S$の微小領域の面積である.閉曲線$L$には反時計周りを正とするように向きがつけられており,単位法線ベクトル$\bm{n}$は$L$の向きに関して右ネジの法則をみたすものとする.









%
\subsubsection{スカラー場のグラディエントの存在}
先と同様にベクトル場$\bm{C}(\vr)$とスカラー場$f(\vr)$を考え,定義域を$D$とする.今回は$\bm{C}(\vr)=\nabla f(\vr)$を仮定しないことに注意したい.領域$D$内に2点$P$と$Q$があり,$P$と$Q$を結ぶ$D$内の2つの曲線を$L_1$,$L_2$とする.また,$P$から$L_1$に沿って$Q$へ向かい,次に$L_2$の逆に沿って$P_0$へ戻ってくる閉曲線を$L$とし,$L:=L_1-L_2$と書く.\\
 いま,ベクトル場$\bm{C}(\vr)$の回転に対して,任意の位置$\vr$で$\nabla\times\bm{C}(\vr)=\bm0$が成り立つとき,Stokesの定理\ref{sto}より
\begin{align}
\oint_L\bm{C}(\vr)\cdot d\vr=\int_S\Bigl(\nabla\times\bm{C}(\vr)\Bigr)\cdot\bm{n}dS=\int_S0\cdot dS=0
\end{align}
が成り立つ.よって,
\begin{align}
\oint_L\bm{C}(\vr)\cdot d\vr=\int_{L_1}\bm{C}(\vr)\cdot d\vr-\int_{L_2}\bm{C}(\vr)\cdot d\vr=0
\end{align}
となるから,
\begin{align}
\int_{L_1}\bm{C}(\vr)\cdot d\vr=\int_{L_2}\bm{C}(\vr)\cdot d\vr
\end{align}
が成り立つ.つまり,ベクトル場$\bm{C}(\vr)$の線積分は$P$と$Q$を結ぶ経路に無関係に定まるということが示せた.\\
 次に,$P$を領域$D$内で固定された点,$Q$を領域$D$内の任意の点とする.上での結果より,ベクトル場$C$の線積分$\displaystyle\int_P^Q\bm{C}(\vr)\cdot\vr$は経路によらないものとする.したがって,この線積分の値は点$Q$のみに依存することになり,$Q$の位置を$\vr=(x,y,z)$とすれば,
\begin{align}\label{sf}
f(\vr)=\int_P^Q\bm{C}\cdot d\vr
\end{align}
によって,スカラー場$f(\vr)$を定義できる.\\
 経路$P\to Q$の曲線を考える.この曲線上の点は,位置ベクトル$\vr=\vr(t)=(x(t),y(t),z(t))$のようにパラメータされているものとする.また,$t=0$のとき点$P$,$t=t$のとき点$Q$であるとする.
\begin{align}
f(\vr)=\int_P^Q\bm{C}(\vr)\cdot d\vr
\end{align}
は
\begin{align}\label{sft}
f(t)=\int_0^t\bm{C}\cdot\frac{d\vr}{dt}dt
\end{align}
と書ける.(\ref{sft})の両辺を$t$で微分すれば,
\begin{align}\label{sft1}
\frac{df}{dt}=\bm{C}(\vr)\cdot\frac{d\vr}{dt}
\end{align}
一方,(\ref{chain})より,
\begin{align}\label{sft2}
\frac{df}{dt}=\nabla f(\vr)\cdot\frac{d\vr}{dt}
\end{align}
と書けるから,(\ref{sft1})と(\ref{sft2})より,
\begin{align}\label{sft3}
\bm{C}(\vr)\cdot\frac{d\vr}{dt}=\nabla f(\vr)\cdot\frac{d\vr}{dt}
\end{align}
曲線についての接ベクトル$\dfrac{d\vr}{dt}$は任意に取ることができるので,
\begin{align}\label{sft4}
\bm{C}(\vr)=\nabla f(\vr)
\end{align}
と書ける.この結果を定理としてまとめると次のようになる.
\begin{theorem}\label{thm2}
領域$D$でベクトル場$\bm{C}(\vr)$が定義されているとする.領域$D$内の任意の位置$\vr$で$\rot\bm{C}(\vr)=\bm0$ならば,
\begin{align}
\oint_L\bm{C}(\vr)\cdot d\vr=0
\end{align}
が成り立ち,$\bm{C}(\vr)=\nabla f(\vr)$であるような$D$全体で定義されるスカラー場$f(\vr)$が存在する.
\end{theorem}
定理\ref{thm1}を簡潔に表すと
\begin{align}
\rot\bm{C}(\vr)=\bm0\Rightarrow \oint_L\nabla f(\vr)\cdot d\vr=0\Rightarrow \bm{C}(\vr)=\nabla f(\vr)
\end{align}
である.また,定理\ref{thm1}と定理\ref{thm2}より,
\begin{align}
\rot\bm{C}(\vr)=\bm0\Rightarrow \oint_L\nabla f(\vr)\cdot d\vr=0\Leftrightarrow  \bm{C}(\vr)=\nabla f(\vr)
\end{align}
が成り立つ.\\
 定理\ref{thm2}を電磁気学を考える場合に便利になるような形に直しておく.定理\ref{thm2}の$\bm{C}(\vr)=\nabla f(\vr)$において,$f(\vr)\to -f(\vr)$とすれば,定理\ref{thm2}は次のように書きかえられる.
\begin{theorem}\label{thm3}
領域$D$でベクトル場$\bm{C}(\vr)$が定義されているとする.領域$D$内の任意の位置$\vr$で$\rot\bm{C}(\vr)=\bm0$ならば,
\begin{align}
\oint_L\bm{C}(\vr)\cdot d\vr=0
\end{align}
が成り立ち,$\bm{C}(\vr)=-\nabla f(\vr)$であるような$D$全体で定義されるスカラー場$f(\vr)$が存在する.一般にこの場合の$f(\vr)$をベクトル場$\bm{C}(\vr)$のスカラーポテンシャルという.
\end{theorem}







%
\subsection{ゲージ変換}
(\ref{mx2})は
\begin{align}
\bm{B}(\vr,t)=\rot{\bm{A}}(\vr,t)
\end{align}
とおくと,自動的にみたされる.ここで${\bm{A}}(\vr,t)$は微分可能な任意関数でベクトルポテンシャルという.これを(\ref{mx1})左辺第2項に代入すると
\begin{align}
\label{ga1}
\nabla\times\bm{E}(\vr,t)+\dfrac{\partial}{\partial t}\rot{\bm{A}}(\vr,t)&=\bm 0\notag\\[10pt]
\therefore\rot\left(\bm{E}(\vr,t)+\dfrac{\partial}{\partial t}{\bm{A}}(\vr,t)\right)&=\bm 0
\end{align}
ここで,(\ref{ga1})に対して,定理\ref{thm3}を$\bm{C}(\vr,t)=\bm{E}(\vr,t)+\dfrac{\partial}{\partial t}{\bm{A}}(\vr,t)$として適用すると,(\ref{ga1})ならば,
\begin{align}
\bm{E}(\vr,t)+\dfrac{\partial}{\partial t}{\bm{A}}(\vr,t)=-\nabla\phi(\vr,t)
\end{align}
であるようなスカラー場$\phi(\vr,t)$が存在する.スカラー場$\phi(\vr,t)$をスカラーポテンシャルという.\\
 結局(\ref{mx1})と(\ref{mx2})の方程式は
\begin{align}
\label{bp}
\bm{B}(\vr,t)=\rot{\bm{A}}(\vr,t)\\[5pt]
\label{sp}
\bm{E}(\vr,t)+\dfrac{\partial}{\partial t}{\bm{A}}(\vr,t)=-\nabla\phi(\vr,t)
\end{align}
とおくことによって自動的にみたされる.そこで,電磁ポテンシャルの組$({\bm{A}},\phi)$を求めれば,それを(\ref{bp})と(\ref{sp})のように微分することによって,電場$\bm{E}(\vr,t)$と$\bm{B}(\vr,t)$が求まることになる.\\
%
%
%
%
 残りの2つの方程式(\ref{mx3})と(\ref{mx4})についてみる前に,組$({\bm{A}},\phi)$の取り方にどれくらいの不定性があるか考えてみる.\\
  組$({\bm{A}},\phi)$とはべつの組$({\bm{A}^{\prime}},{\phi^\prime})$もまた,(\ref{bp}),(\ref{sp})をみたすとする.するとまず,(\ref{bp})から
\begin{align}
\label{bp1}
\bm{B}(\vr,t)&=\rot{\bm{A}}(\vr,t)\\[5pt]
\label{bp2}
\bm{B}(\vr,t)&=\rot{\bm{A}^{\prime}}(\vr,t)
\end{align}
が成り立つ.(\ref{bp1})$-$(\ref{bp2})より,
\begin{align}
\label{ga2}
\rot\Bigl({\bm{A}}(\vr,t)-{\bm{A}^{\prime}}(\vr,t)\Bigr)=\bm0
\end{align}
となる.定理\ref{thm3}より,
\begin{align}
\label{ga3}
{\bm{A}}(\vr,t)-{\bm{A}^{\prime}}(\vr,t)=-\nabla u(\vr,t)
\end{align}
であるようなスカラー場$u(\vr,t)$が存在する.\\
 次に(\ref{sp})から
\begin{align}
\label{sp1}
\bm{E}(\vr,t)+\dfrac{\partial}{\partial t}{\bm{A}}(\vr,t)&=-\nabla\phi(\vr,t)\\[5pt]
\label{sp2}
\bm{E}(\vr,t)+\dfrac{\partial}{\partial t}{\bm{A}^{\prime}}(\vr,t)&=-\nabla{\phi^\prime}(\vr,t)
\end{align}
が成り立つ.(\ref{sp1})$-$(\ref{sp2})より,
\begin{align}
\label{ga4}
\dfrac{\partial}{\partial t}\Bigl({\bm{A}}(\vr,t)-{\bm{A}^{\prime}}(\vr,t)\Bigr)=-\nabla\phi(\vr,t)+\nabla{\phi^\prime}(\vr,t)
\end{align}
となる.したがって(\ref{ga3})を(\ref{ga4})左辺へ代入すると
\begin{align}
-\dfrac{\partial}{\partial t}\nabla u(\vr,t)=-\nabla\phi(\vr,t)+\nabla{\phi^\prime}(\vr,t)\notag\\[10pt]
\therefore\nabla {\phi^\prime}(\vr,t)=\nabla\phi(\vr,t)-\dfrac{\partial}{\partial t}\nabla u(\vr,t)
\end{align}
が得られる.よって,
\begin{align}\label{ga5}
{\phi^\prime}(\vr,t)=\phi(\vr,t)-\dfrac{\partial u(\vr,t)}{\partial t}
\end{align}
となる.(\ref{ga3})と(\ref{ga5})をまとめると,変換
%%
%%
\begin{subnumcases}
  {}
  \label{gage1}
{\bm{A}^{\prime}}(\vr,t)={\bm{A}}(\vr,t)+\nabla u(\vr,t)& \\[15pt]
  \label{gage2}
 {\phi^\prime}(\vr,t)=\phi(\vr,t)-\dfrac{\partial u(\vr,t)}{\partial t}&
\end{subnumcases}
が得られる.この変換によって,あるベクトルポテンシャル${\bm{A}}$とスカラーポテンシャル$\phi$から新しいベクトルポテンシャル${\bm{A}^{\prime}}$とスカラーポテンシャル${\phi^\prime}$をつくることができる.(\ref{gage1})と(\ref{gage2})の変換をゲージ変換という.電磁ポテンシャルの組$({\bm{A}},\phi)$と$({\bm{A}^{\prime}},{\phi^\prime})$,どちらを選んでもまったく同じ電場$\bm{E}$と磁場$\bm{B}$が得られるから,電磁ポテンシャル${\bm{A}}(\vr,t)$と$\phi(\vr,t)$には任意関数$u(\vr,t)$の不定性があるといえる.つまり,(\ref{gage1})と(\ref{gage2})のゲージ変換に対して,電場$\bm{E}$と磁場$\bm{B}$とは不変である.




















%
\subsection{ベクトルポテンシャルとスカラーポテンシャルをつかったMaxwell方程式の表現}
次に,ベクトルポテンシャル${\bm{A}}$とスカラーポテンシャル$\phi$を決める方程式をみちびかねばならない.電場$\bm{E}$と磁場$\bm{B}$はスカラーポテンシャル$\phi$とベクトルポテンシャル${\bm{A}}$を用いて,
\begin{numcases}
  {}
    \label{ef}
\bm{E}(\vr,t)=-\nabla\phi(\vr,t)-\dfrac{\partial}{\partial t}{\bm{A}}(\vr,t)&\\[15pt]
  \label{mag}
\bm{B}(\vr,t)=\rot{\bm{A}}(\vr,t)&
\end{numcases}
と書ける.上の2式を用いて,Maxwell方程式(\ref{mx3})と(\ref{mx4})を$\bm{E}$と$\bm{B}$ではなく,$\phi$と${\bm{A}}$を用いて書き直す.\\
 Maxwell方程式(\ref{mx3})より,
\begin{align}
   \nabla\times\bm{B}(\vr,t)-\epsilon_0\mu_0\dfrac{\partial\bm{E}(\vr,t)}{\partial t}
      =\mu_0{\bm{i}}(\vr,t).\notag
\end{align}
これに,(\ref{ef})と(\ref{mag})を代入すると,
\begin{align}\label{w1}
   \rot\Bigl(\rot{\bm{A}}(\vr,t)\Bigr)-\epsilon_0\mu_0\dfrac{\partial}{\partial t}\left(
   -\nabla\phi(\vr,t)-\dfrac{\partial}{\partial t}{\bm{A}}(\vr,t)
   \right)
      =\mu_0{\bm{i}}(\vr,t)
\end{align}
となる.ここで(\ref{w1})左辺第1項にベクトル解析の恒等式
\begin{align}
   \rot\Bigl(\rot{\bm{A}}(\vr,t)\Bigr)=\nabla\Bigl(\dv{\bm{A}}(\vr,t)\Bigr)-\Delta{\bm{A}}(\vr,t)
\end{align}
を用いると,
\begin{align}
\nabla\Bigl(\dv{\bm{A}}(\vr,t)\Bigr)-\Delta{\bm{A}}(\vr,t)-\epsilon_0\mu_0\dfrac{\partial}{\partial t}\left(
   -\nabla\phi(\vr,t)-\dfrac{\partial}{\partial t}{\bm{A}}(\vr,t)
   \right)
      =\mu_0{\bm{i}}(\vr,t)
\end{align}
整理すると,
\begin{align}\label{w2}
\nabla
\left(\dv{\bm{A}}(\vr,t)+\epsilon_0\mu_0\dfrac{\partial\phi(\vr,t)}{\partial t}\right)
+\epsilon_0\mu_0\dfrac{\partial^2{\bm{A}}(\vr,t)}{\partial t^2}
-\Delta{\bm{A}}(\vr,t)=\mu_0{\bm{i}}(\vr,t)
\end{align}
が得られる.また,(\ref{mx4})より,
\begin{align}
   \nabla\cdot\bm{E}(\vr,t)
      =\frac{\rho(\vr,t)}{\epsilon_0}.\notag
\end{align}
これに,(\ref{ef})を代入すると
\begin{align}
   \nabla\cdot\bm{E}(\vr,t)&=\dv\left(-\nabla\phi(\vr,t)-\dfrac{\partial}{\partial t}{\bm{A}}(\vr,t)\right)\notag\\[10pt]
      &=-\Delta\phi(\vr,t)-\dfrac{\partial}{\partial t}\Bigl(\dv{\bm{A}}(\vr,t)\Bigr)=\frac{\rho(\vr,t)}{\epsilon_0}
\end{align}
が得られる.上で得られた結果をまとめると
\begin{align}
    \label{mxa1}
\bm{E}(\vr,t)=-\nabla\phi(\vr,t)-\dfrac{\partial}{\partial t}{\bm{A}}(\vr,t)
\end{align}
\begin{align}
  \label{mxa2}
\bm{B}(\vr,t)=\rot{\bm{A}}(\vr,t)
\end{align}
\begin{align}\label{mxa3}
\epsilon_0\mu_0\dfrac{\partial^2{\bm{A}}(\vr,t)}{\partial t^2}
-\Delta{\bm{A}}(\vr,t)+
\nabla
\left(\dv{\bm{A}}(\vr,t)+\epsilon_0\mu_0\dfrac{\partial\phi(\vr,t)}{\partial t}\right)
=\mu_0{\bm{i}}(\vr,t)
\end{align}
\begin{align}\label{mxa4}
-\dfrac{\partial}{\partial t}\Bigl(\dv{\bm{A}}(\vr,t)\Bigr)-\Delta\phi(\vr,t)=\frac{\rho(\vr,t)}{\epsilon_0}
\end{align}
である.(\ref{mxa1})$\sim$(\ref{mxa4})の方程式系は,(\ref{mx1})$\sim$(\ref{mx4})のMaxwellの方程式系と内容的に全く等価であって,(\ref{mxa3})と(\ref{mxa4})の2式を解いて$\phi$と${\bm{A}}$を求め,それらを(\ref{mxa1})と(\ref{mxa2})へ代入することによって,電場$\bm{E}$と磁場$\bm{B}$を求めることができる.\\



























%
\subsection{Lorentzゲージと波動方程式}
 (\ref{mxa3})と(\ref{mxa4})において,
\begin{align}\label{lor1}
\chi(\vr,t):=\dv{\bm{A}}(\vr,t)+\epsilon_0\mu_0\dfrac{\partial\phi(\vr,t)}{\partial t}
\end{align}
とおくと,(\ref{mxa3})は
\begin{align}\label{mxb3}
\epsilon_0\mu_0\dfrac{\partial^2{\bm{A}}(\vr,t)}{\partial t^2}
-\Delta{\bm{A}}(\vr,t)+
\nabla\chi(\vr,t)
=\mu_0{\bm{i}}(\vr,t).
\end{align}
(\ref{mxa4})は(\ref{mxa4})の左辺第1項に$-\dv{\bm{A}}(\vr,t)=-\chi(\vr,t)+\epsilon_0\mu_0\dfrac{\partial\phi(\vr,t)}{\partial t}$を代入し,整理すれば
\begin{align}\label{mxb4}
\epsilon_0\mu_0\dfrac{\partial^2\phi(\vr,t)}{\partial t^2}-\Delta\phi(\vr,t)-\dfrac{\partial\chi(\vr,t)}{\partial t}=\frac{\rho(\vr,t)}{\epsilon_0}
\end{align}
となる.
(\ref{lor1})の$\chi(\vr,t)$はゲージ変換
\begin{subnumcases}
  {}
  \label{gal1}
{\bm{A}}_L(\vr,t)={\bm{A}}(\vr,t)+\nabla u(\vr,t)& \\[15pt]
  \label{gal2}
 \phi_L(\vr,t)=\phi(\vr,t)-\dfrac{\partial u(\vr,t)}{\partial t}&
\end{subnumcases}
に対して,変換
\begin{align}\label{lor2}
\chi_L(\vr,t)&=\dv{\bm{A}}_L(\vr,t)+\epsilon_0\mu_0\dfrac{\partial\phi_L(\vr,t)}{\partial t}\notag\\[10pt]
%
&=\dv\Bigl({\bm{A}}(\vr,t)-\nabla u(\vr,t)\Bigr)
+\epsilon_0\mu_0\dfrac{\partial}{\partial t}\left(\phi(\vr,t)-\dfrac{\partial u(\vr,t)}{\partial t}\right)\notag\\[10pt]
%
&=\dv{\bm{A}}(\vr,t)+\epsilon_0\mu_0\dfrac{\partial\phi(\vr,t)}{\partial t}
+\Delta u(\vr,t)-\epsilon_0\mu_0\dfrac{\partial^2u(\vr,t)}{\partial t^2}\notag\\[10pt]
%
\therefore\chi_L(\vr,t)&=
\chi(\vr,t)
+\left[\Delta-\epsilon_0\mu_0\dfrac{\partial^2}{\partial t^2}\right]u(\vr,t)
\end{align}
を受ける.つまり,${\bm{A}}$と$\phi$に対して,ゲージ変換(\ref{gal1})と(\ref{gal2})を施すと,(\ref{lor1})の$\chi(\vr,t)$には任意関数$u(\vr,t)$の不定性があるということが(\ref{lor2})からわかる.\\
 これより,(\ref{mxb3})と(\ref{mxb4})の${\bm{A}}$と$\phi$に対して,ゲージ変換(\ref{gal1})と(\ref{gal2})を施すと,
\begin{align}
\epsilon_0\mu_0\dfrac{\partial^2{\bm{A}}_L(\vr,t)}{\partial t^2}
-\Delta{\bm{A}}_L(\vr,t)+
\nabla\chi_L(\vr,t)
=\mu_0{\bm{i}}(\vr,t).
\end{align}
\begin{align}
\epsilon_0\mu_0\dfrac{\partial^2\phi_L(\vr,t)}{\partial t^2}-\Delta\phi_L(\vr,t)-\dfrac{\partial\chi_L(\vr,t)}{\partial t}=\frac{\rho(\vr,t)}{\epsilon_0}
\end{align}
となる.(\ref{lor2})の変換性を利用し,任意関数$u(\vr,t)$を適当に選び,(\ref{lor2})の左辺が$0$になるようにすると,
\begin{align}
\chi_L(\vr,t)&=
\chi(\vr,t)
+\left[\Delta-\epsilon_0\mu_0\dfrac{\partial^2}{\partial t^2}\right]u(\vr,t)=0
\end{align}
であるから,ポテンシャル$\phi_L$と${\bm{A}}_L$に対する条件
\begin{align}\label{lor3}
\chi_L(\vr,t)=\dv{\bm{A}}_L(\vr,t)+\epsilon_0\mu_0\dfrac{\partial\phi_L(\vr,t)}{\partial t}=0
\end{align}
が得られる.この条件をLorentzゲージと呼ぶ.また,(\ref{lor1})より,
\begin{align}\label{lor4}
\left[\Delta-\epsilon_0\mu_0\dfrac{\partial^2}{\partial t^2}\right]u(\vr,t)
=-\dv{\bm{A}}(\vr,t)+\epsilon_0\mu_0\dfrac{\partial\phi(\vr,t)}{\partial t}
\end{align}
を得る.(\ref{lor3})の条件式のおかげで,$\phi_L$と${\bm{A}}_L$のみたす式(\ref{mxb3})と(\ref{mxb4})は,次のように書きかえられる.
\begin{align}\label{mxd3}
\epsilon_0\mu_0\dfrac{\partial^2{\bm{A}}_L(\vr,t)}{\partial t^2}
-\Delta{\bm{A}}_L(\vr,t)
=\mu_0{\bm{i}}(\vr,t).
\end{align}
\begin{align}\label{mxd4}
\epsilon_0\mu_0\dfrac{\partial^2\phi_L(\vr,t)}{\partial t^2}-\Delta\phi_L(\vr,t)
=\frac{\rho(\vr,t)}{\epsilon_0}
\end{align}










\subsection{まとめ}
光速$c$を
\begin{align}
c:=\frac{1}{\sqrt{\epsilon_0\mu_0}}
\end{align}
と定義し,得られた方程式系をまとめると,次のようになる.
\begin{align}
    \label{mw1}
\bm{E}(\vr,t)=-\nabla\phi_L(\vr,t)-\dfrac{\partial}{\partial t}{\bm{A}}_L(\vr,t)
\end{align}
\begin{align}
  \label{mw2}
\bm{B}(\vr,t)=\rot{\bm{A}}_L(\vr,t)
\end{align}
\begin{align}\label{mw3}
\left(\frac{1}{c^2}\dfrac{\partial^2}{\partial t^2}
-\Delta\right){\bm{A}}_L(\vr,t)
=\mu_0{\bm{i}}(\vr,t).
\end{align}
\begin{align}\label{mw4}
\left(\frac{1}{c^2}\dfrac{\partial^2}{\partial t^2}
-\Delta\right)\phi_L(\vr,t)
=\frac{\rho(\vr,t)}{\epsilon_0}
\end{align}
\begin{align}\label{lor}
\dv{\bm{A}}_L(\vr,t)+\frac{1}{c^2}\dfrac{\partial\phi_L(\vr,t)}{\partial t}=0.
\end{align}
これらの方程式系はMaxwellの方程式系と等価である.まず,(\ref{mw3})と(\ref{mw4})の4個の独立な波動方程式を解き,得られた電磁ポテンシャルの組$({\bm{A}}_L,\phi_L)$のうちで条件(\ref{lor})をみたす組だけを選ぶ.この条件(\ref{lor})をLorentz条件といい,この条件をみたす電磁ポテンシャル${\bm{A}}_L$,$\phi_L$をLorentzゲージにおける電磁ポテンシャルという.そして,求めた${\bm{A}}_L$,$\phi_L$を(\ref{mw1}),(\ref{mw2})へ代入すれば,電場$\bm{E}$と磁場$\bm{B}$を求めることができる.

% 
\section{自由空間の電磁場}
 はじめは,$z$軸方向に一様な電場$\bm E_0=(0,0,E_0)$,$E_0=\bm E_0$があったとする.そこに帯電していない半径$a$の導体球を持ち込んだとき,電子は電場からの力によって,$z$軸の負の方向に移動し,$z$軸の正の方向は正に帯電する.したがって,導体表面上には電荷が誘導される.その誘導電荷によってつくられる静電場$\bm E_1$とはじめの静電場$\bm E_0$との重ね合わせによる静電場$\bm E$ができる.\\
 問題は,導体球によって,乱された静電ポテンシャル,すなわち電場$\bm E$による静電ポテンシャルを見出すことである.
\subsection{1次元の波動方程式}
$\psi=\psi(x,t)$とするとき,
\be\label{wd1}
\fbox{
$\left(
\dfrac{\partial^2}{\partial t^2}-v^2\dfrac{\partial^2}{\partial x^2}
\right)
\psi=0$
}
\ee
を1次元の波動方程式という.ただし,$v$は正の実定数である.\\
 (\ref{wd1})の一般解は,(2階の偏微分が可能で,偏微分して連続な)任意関数$f=f(x-vt)$,$g=g(x+vt)$を用いて,
\be
\psi=f(x-vt)+g(x+vt)
\ee
の形にかける.


















%
\begin{proof}
$t$と$x$の偏微分に対して,次の表記法を定義する:
\be
\partial_t\equiv\dfrac{\partial}{\partial t}
,\ \ \ \ 
\partial_x\equiv\dfrac{\partial}{\partial x}
\ee
すると,(\ref{wd1})は,
\be
(\partial_t^2-v^2\partial_x^2)\psi=0
\ee
と書ける.上式の左辺の()の中を因数分解すれば,
\be\label{wd12}
(\partial_t+v\partial_x)(\partial_t-v\partial_x)\psi=0
\ee
となる.\\
 次に変数変換$(x,t)\to[\xi,\eta]$,$\psi(x,t)=\psi[\xi,\eta]$
\begin{subnumcases}
  {}
  \xi=x-vt& \\
  \eta=x+vt&
\end{subnumcases}
を考える.合成関数の偏微分chain-ruleより
\begin{align}
\frac{\partial}{\partial t}
&=\frac{\partial\xi}{\partial t}\frac{\partial}{\partial \xi}+\frac{\partial\eta}{\partial t}\frac{\partial}{\partial\eta}
=-v\frac{\partial}{\partial \xi}+v\frac{\partial}{\partial\eta}
=v\left(-\frac{\partial}{\partial \xi}+\frac{\partial}{\partial\eta}\right)\\[10pt]
%
\frac{\partial}{\partial x}
&=\frac{\partial\xi}{\partial x}\frac{\partial}{\partial \xi}+\frac{\partial\eta}{\partial x}\frac{\partial}{\partial\eta}
=\frac{\partial}{\partial \xi}+\frac{\partial}{\partial\eta}
\end{align}
%%%%%%%%
したがって,
\begin{subnumcases}
  {}
  \partial_t+v\partial_x=2v\frac{\partial}{\partial \eta}& \\[10pt]
  \partial_t-v\partial_x=-2v\frac{\partial}{\partial \xi}&
\end{subnumcases}
となり,(\ref{wd12})は,
\be
(\partial_t+v\partial_x)(\partial_t-v\partial_x)\psi=-4v^2\frac{\partial}{\partial \eta}\frac{\partial}{\partial \xi}\psi=0
\ee
\be
\therefore\frac{\partial}{\partial \eta}\left(\frac{\partial\psi}{\partial \xi}\right)=0
\ee
となる.つまり,$\dfrac{\partial\psi}{\partial \xi}$は$\eta$に依存しないということがわかる.しかし,$\dfrac{\partial\psi}{\partial \xi}$は$\xi$にはいくらでも依存してもかまわないので,$\xi$のみの任意関数
\be
f=f[\xi,\eta]=f(\xi)
\ee
を用いて,
\be
\frac{\partial\psi}{\partial \xi}=\text{$\xi$のみの関数}=\frac{\partial f}{\partial \xi}
\ee
と書ける.ここで,右辺を$f(\xi)$ではなく$\dfrac{\partial\psi}{\partial \xi}$としたのは便宜的な理由による.\\
 よって,
\be
\frac{\partial}{\partial \xi}\left\{
\psi-f(\xi)
\right\}=0
\ee
となる.上式より,$\psi-f(\xi)$は$\xi$に依存しないということがわかる.
\footnote{%
ここで注意したいのは,$\psi$と$f(\xi)$の引き算が$\xi$に依存しないということである.
}
$\eta$にはいくらでも依存していいので,$\eta$のみの任意関数
\be
g=g[\xi,\eta]=g(\eta)
\ee
を用いて,
\be
\psi-f(\xi)=g(\eta)
\ee
と書ける.\\
 したがって,
\be
\psi[\xi,\eta]=f(\xi)+g(\eta)
\ee
となる.変数を$[\xi,\eta]\to(x,t)$に戻せば,
\be
\psi(x,t)=f(x-vt)+g(x+vt)
\ee
と書ける.
\end{proof}























%
\subsection{進行電磁波}
$z$軸方向に進む電磁場を考える.これは${\bm{E}}$,${\bm{B}}$が$z$と$t$のみに依存することを意味する.そこで以下では,
\be
\partial_x{\bm{E}}=\partial_x{\bm{E}}=\bm{0},\ \ \ \ \partial_x{\bm{B}}=\partial_x{\bm{B}}=\bm{0}
\ee
とする.\\
 自由空間でのMaxwell方程式は
\begin{align}
\label{mx1}
\nabla\times\bm{E}(\vr,t)+\dfrac{\partial\bm{B}(\vr,t)}{\partial t}=\bm 0
\end{align}

\begin{align}
\label{mx2}
\nabla\cdot\bm{B}(\vr,t)=0
\end{align}

\begin{align}
\label{mx3}
   \nabla\times\bm{B}(\vr,t)=\epsilon_0\mu_0\dfrac{\partial\bm{E}(\vr,t)}{\partial t}
\end{align}

\begin{align}
\label{mx4}
  \nabla\cdot{\bm{E}}(\vr,t)
      =0 
\end{align}
である.\\
 (\ref{mx2}),(\ref{mx4})より,
\be
\nabla\cdot{\bm{E}}={\partial_x} E_x+{\partial_y} E_y+{\partial_z} E_z=0,\ \ \ \therefore{\partial_z} E_z=0
\ee
%
\be
\nabla\cdot{\bm{B}}={\partial_x} B_x+{\partial_y} B_y+{\partial_z} B_z=0,\ \ \ \therefore{\partial_z} B_z=0
\ee
となる.\\
 (\ref{mx1}),(\ref{mx3})の$z$成分より,
\be
{\partial_x} E_y+{\partial_y} E_x=-{\partial_z} B_z,\ \ \ \therefore{\partial_z} B_z=0
\ee
%
\be
{\partial_x} B_y+{\partial_y} B_x=\epsilon_0\mu_0{\partial_z} E_z,\ \ \ \therefore{\partial_z} E_z=0
\ee
したがって,$E_z$,$B_z$は$x,y,t$に依存しない定数である.そこで,これをゼロとおく.(以下,静電場,静磁場部分は$0$とおく.):
\be
E_z=0,\ \ \ \ B_z=0
\ee
よって,${\bm{k}}=(0,0,1)$とすれば,
\be
{\bm{E}}\perp{\bm{k}},\ \ \ \ {\bm{B}}\perp{\bm{k}}
\ee
となる.これは進行電磁波が横波であることを表す.\\
%
%
 次に,${\bm{E}}$の方向を$x$軸方向にとる.つまり,$E_y=E_z=0$とする.(\ref{mx1})の$x$成分,(\ref{mx3})の$y$成分より,
\be
{\partial_y} E_z+{\partial_z} E_y=-{\partial_z} B_x,\ \ \ \therefore{\partial_z} B_x=0
\ee
%
\be
{\partial_z} B_x+{\partial_x} B_z=\epsilon_0\mu_0{\partial_z} E_y,\ \ \ \therefore{\partial_z} B_x=0
\ee
よって,$B_x$は$x,y,z,t$に依存しない定数であり,$B_x=0$とおく.したがって,
\be
{\bm{E}}=(E_x,0,0),\ \ \ {\bm{B}}=(0,B_y,0)
\ee
となり,${\bm{E}},{\bm{B}},{\bm{k}}$は互いに直交する.
%
%
以下,常に${\bm{E}}=(E_x,0,0)$であるとする.\\
${\bm{E}}$のみたす波動方程式は,$\epsilon_0\mu_0=1/c^2$として,
\be
(\partial_t^2-v^2\partial_x^2)\psi=0
\ee
であった,$0$でない$E_x$については,
\be
\partial_t^2E_x-c^2({\partial_x}^2E_x+{\partial_y}^2E_x+{\partial_z}^2E_x)=0
\ee
であるが,${\partial_x} E_x={\partial_y} E_x=0$なので,
\be\label{wex}
\dfrac{\partial^2E_x}{\partial t^2}-c^2\dfrac{\partial^2E_x}{\partial z^2}=0
\ee
となる.これは1次元の波動方程式なので,$E_x$は任意関数$f$と$g$を用いて,
\be\label{sol1}
E_x=f(z-ct)+g(z+ct)
\ee
(\ref{mx1})の$y$成分は,
\be
{\partial_z} E_x-{\partial_x} E_z=-{\partial_z} B_y
\ee
$E_z=0$であるから,したがって,
\begin{eqnarray*}
  \begin{split}
{\partial_z} B_y=-{\partial_z} E_x&=-\frac{\partial}{\partial z}
\left\{
f(z-ct)+g(z+ct)
\right\}\\[10pt]
%
&=-\left\{
\frac{\partial(z-ct)}{\partial z}\frac{\partial f(z-ct)}{\partial (z-ct)}
+\frac{\partial(z+ct)}{\partial z}\frac{\partial g(z+ct)}{\partial (z+ct)}
\right\}\\[10pt]
%
&=-\left\{
\frac{\partial f(z-ct)}{\partial (z-ct)}
+\frac{\partial g(z+ct)}{\partial (z+ct)}
\right\}\\[10pt]
%
&=\frac{1}{c}\left\{
\frac{\partial f(z-ct)}{\partial (z-ct)}\cdot(-c)
-\frac{\partial g(z+ct)}{\partial (z+ct)}\cdot(c)
\right\}\\[10pt]
%
&=\frac{1}{c}\left\{
\frac{\partial f(z-ct)}{\partial (z-ct)}\cdot(-c)
-\frac{\partial g(z+ct)}{\partial (z+ct)}\cdot(c)
\right\}\\[10pt]
  \end{split}
\end{eqnarray*}
ここで,$\mp c={\partial_z} (z\mp ct)$であるから,
\begin{eqnarray*}
  \begin{split}
{\partial_z} B_y=-{\partial_z} E_x
&=\frac{1}{c}\left\{
\frac{\partial f(z-ct)}{\partial (z-ct)}\frac{\partial (z-ct)}{\partial t}
-\frac{\partial g(z+ct)}{\partial (z+ct)}\frac{\partial (z+ct)}{\partial t}
\right\}\\[10pt]
%
&=\frac{1}{c}\left\{
\frac{\partial f(z-ct)}{\partial t}
-\frac{\partial g(z+ct)}{\partial t}
\right\}\\[10pt]
%
&=\frac{\partial }{\partial t}
\left[
\frac{1}{c}\left\{
f(z-ct)+g(z+ct)
\right\}\right]\\[10pt]
  \end{split}
\end{eqnarray*}
つまり,
\be
\frac{\partial }{\partial t}
\left[
B_y-
\frac{1}{c}\left\{
f(z-ct)+g(z+ct)
\right\}\right]=0
\ee
を得る.$B_y$は$x,y$に依存しないので,
\be
B_y=\frac{1}{c}\left\{
f(z-ct)+g(z+ct)
\right\}+D(z)
\ee
となる.ここで,$D(z)$は$z$のみに依存する任意関数である.\\
 (\ref{mx3})の$x$成分より,
\be
{\partial_y} B_z-{\partial_z} B_y=\epsilon_0\mu_0{\partial_z} E_x=\frac{1}{c^2}{\partial_z} E_x
\ee
$B_z=0$であるから,
\be
-{\partial_z} B_y=\frac{1}{c^2}{\partial_z} E_x
\ee
となる.すなわち,
\be
-\frac{\partial }{\partial z}
\left[
\frac{1}{c}\left\{
f(z-ct)+g(z+ct)
\right\}+D(z)\right]
%
=\frac{1}{c^2}\frac{\partial }{\partial t}
\left\{
f(z-ct)+g(z+ct)
\right\}\\[10pt]
\ee
%%
%%
\be
\therefore
-\frac{1}{c}\left\{
\frac{\partial f(z-ct)}{\partial (z-ct)}+\frac{\partial g(z+ct)}{\partial (z+ct)}
\right\}+\frac{\partial D(z)}{\partial z}
%
=\frac{1}{c^2}\left\{
\frac{\partial f(z-ct)}{\partial (z-ct)}\cdot(-c)+\frac{\partial g(z+ct)}{\partial (z+ct)}\cdot(c)
\right\}
\ee
したがって,${\partial_z} D(z)=0$.これは,$D(z)$が$x,y,z,t$に依存しない定数であることを表すので,$D(z)=0$とおく.\\
 以上から,
\be
E_x=f(z-ct)+g(z+ct)
\ee
\be
B_y=\frac{1}{c}
\left\{
f(z-ct)+g(z+ct)
\right\}
\ee
を得る.これが$z$軸の方向に進行する($x$軸の方向に偏った)電磁波の一般形である.




























%
\subsection{平面波}
$z$軸の方向に進む進行電磁波は,
\be
f_1=f_1(z-ct),\ \ \ f_2=f_2(z-ct)
\ee
\be
g_1=g_1(z+ct),\ \ \ g_2=g_2(z+ct)
\ee
を任意関数として,
\begin{subnumcases}
  {}
  E_x=f_1+g_1,\ \ \ E_y=f_2+g_2,\ \ \ E_z=0& \\[10pt]
  B_x=\frac{1}{c}(f_1+g_1),\ \ \ B_y=\frac{1}{c}(f_2+g_2),\ \ \ B_z=0&
\end{subnumcases}
と書けた.このとき,$f_1,f_2$を考えたもの(進行波)を,${\bm{E}}_{{\uparrow}}$,${\bm{B}}_{{\uparrow}}$.$g_1,g_2$を考えたもの(後退波)を,${\bm{E}}_{{\downarrow}}$,${\bm{B}}_{{\downarrow}}$としよう:
\be
{\bm{E}}_{{\uparrow}}=(f_1,f_2,0),\ \ \ {\bm{B}}_{{\uparrow}}=\left(-\frac{1}{c}f_2,-\frac{1}{c}f_1,0\right)
\ee
\be
{\bm{E}}_{{\downarrow}}=(g_1,g_2,0),\ \ \ {\bm{B}}_{{\downarrow}}=\left(\frac{1}{c}g_2,-\frac{1}{c}g_1,0\right)
\ee

















%
\subsubsection{直交性について}
\paragraph{右手系}
${\bm{E}}$と${\bm{B}}$は必ずしも直交しないが,${\bm{E}}_{{\uparrow}}\perp{\bm{B}}_{{\uparrow}}$($\therefore{\bm{E}}_{{\uparrow}}\cdot{\bm{B}}_{{\uparrow}}=0$),${\bm{E}}_{{\downarrow}}\perp{\bm{B}}_{{\downarrow}}$($\therefore{\bm{E}}_{{\downarrow}}\cdot{\bm{B}}_{{\downarrow}}=0$)であり,波の進む向きを${\bm{k}}$とすると,図\ref{},\ref{}のように右手系を作る.

















%
\paragraph{ポインティングベクトル}
ポインティングベクトル$\bm{S}=\dfrac{1}{\mu_0}{\bm{E}}\times{\bm{B}}$について,
\begin{align}
\bm{S}=\frac{1}{\mu_0}
\left(
    \begin{array}{c}
      f_1+g_1 \\[10pt]
      f_2+g_2 \\[10pt]
      0
    \end{array}
  \right)
  \times
  \left(
    \begin{array}{c}
      \dfrac{1}{c}(f_1+g_1) \\[10pt]
      \dfrac{1}{c}(f_2+g_2) \\[10pt]
      0
    \end{array}
  \right)
  =\frac{1}{\mu_0c}\left(0,0,f_1^2+f_2^2-g_1^2-g_2^2\right)
\end{align}
となるので,波の伝わる向き($z$軸方向)に,エネルギーが流れていることがわかる.



















%
\paragraph{電磁場のエネルギー密度}
電場と磁場のエネルギー密度をそれぞれ$w_e,w_m$とおく.電磁場のエネルギー密度
\be
w=w_e+w_m=\frac{\epsilon_0}{2}|{\bm{E}}|^2+\frac{1}{2\mu_0}|{\bm{B}}|^2
\ee
も${\bm{E}},{\bm{B}}$のままではきれいにならない.しかし,${\bm{E}}={\bm{E}}_{{\uparrow}}$,${\bm{B}}={\bm{B}}_{{\downarrow}}$のときは,
\be
w_e=\frac{\epsilon_0}{2}|{\bm{E}}_{{\uparrow}}|^2=\frac{\epsilon_0}{2}(f_1^2+f_2^2)
\ee
\be
w_m=\frac{1}{2\mu_0}|{\bm{B}}_{{\uparrow}}|^2=\frac{1}{2\mu_0c^2}(f_1^2+f_2^2)=\frac{\epsilon_0}{2}(f_1^2+f_2^2)
\ee
となり,$w_e=w_m$となる.



























\subsection{}
$f$のみ,$g$のみのように波の進む向きを決定したものが,通常は平面波と呼ばれるものである.以下,${\bm{E}}={\bm{E}}_{{\uparrow}}$,${\bm{B}}={\bm{B}}_{{\uparrow}}$として考える.
\be\label{v1}
z-ct=-c\left(
t-\frac{z}{c}
\right)
=-\frac{c}{\omega}\left(
\omega t-\frac{\omega z}{c}
\right)
=-\frac{c}{\omega}\left(
\omega t-kz
\right)
\ee
(ここで,$k=\omega/c$とおいた.)より,(正弦波を考えることにして)例えば
\be
f_1(z-ct)=E_0\sin{(\omega t-kz)}
\ee
とおくことができる.(\ref{v1})より,
\be
\sin(\omega t-kz)=\sin{\left\{
-\frac{\omega}{c}(z-ct)
\right\}}
\ee
さらに,$f_2=0$とすれば,
\begin{subnumcases}
  {}
  {\bm{E}}=E_0\sin{(\omega t-kz)}(1,0,0)& \\[10pt]
  {\bm{B}}=\frac{E_0}{c}\sin{(\omega t-kz)}(0,1,0)&
\end{subnumcases}
となり,このとき,
\be
\bm{S}=\frac{1}{\mu_0}{\bm{E}}\times{\bm{B}}=\frac{E_0^2}{\mu_0c^2}\sin^2(\omega t-kz)(0,0,1)
\ee
となる.





% 
\part{電磁場の量子化}
\section{電磁場の量子化}
\subsection{離散化された場合}
一般化された位置と運動量が次の交換関係を満足するように要請する:
\begin{equation}
    [\hat{q}_{\lambda}, \hat{p}_{\lambda^\prime}] = i\hbar\delta_{\lambda,\lambda^\prime},\ \ \
    [\hat{q}_{\lambda}, \hat{q}_{\lambda^\prime}] = 0,\ \ \ 
    [\hat{p}_{\lambda}, \hat{p}_{\lambda^\prime}] = 0
\end{equation}
ここで,$\lambda=(\bm{k},\sigma)$は波数と偏光の自由度$\sigma$を一つにまとめたものを表す.このとき,電磁場のエネルギーは,第二量子化されたHamiltonianとなる:
\begin{equation}\label{elmg_Hamiltonian}
    \hat{H} = \frac{1}{2}\sum_{\bm{k},\sigma} (\hat{p}_{\lambda}^2 + \omega_{\lambda}^2 \hat{q}_{\lambda}^2).
\end{equation}
ここで,次の生成消滅演算子$\hat{a}^{\dag}_{\lambda}$, $\hat{a}_{\lambda}$を導入する:
\begin{align}
    \hat{a}^{\dag}_{\lambda} &= \frac{1}{\sqrt{2\pi\omega_{\lambda}}}
    (\omega_{\lambda} \hat{q}_{\lambda}-i\hat{p}_{\lambda})\\[10pt]
    \hat{a}_{\lambda} &= \frac{1}{\sqrt{2\pi\omega_{\lambda}}}
    (\omega_{\lambda} \hat{q}_{\lambda}+i\hat{p}_{\lambda})
\end{align}
生成消滅演算子は次の交換関係を満たす:
\begin{equation}
    [\hat{a}_{\lambda}, \hat{a}^{\dag}_{\lambda}] = \delta_{\lambda,\lambda^\prime}\hat{1},\ \ \
    [\hat{a}_{\lambda}, \hat{a}_{\lambda^\prime}] = 0,\ \ \ 
    [\hat{a}^{\dag}_{\lambda}, \hat{a}^{\dag}_{\lambda^\prime}] = 0
\end{equation}

\eqref{}を$\hat{q}_{\lambda}$, $\hat{p}_{\lambda}$について解けば,
\begin{align}
    \hat{q}_{\lambda} &= \sqrt{\frac{\hbar}{2\omega_{\lambda}}}
    (\hat{a}_{\lambda}+\hat{a}^{\dagger}_{\lambda})\\[10pt]
    \hat{p}_{\lambda} &= -i\sqrt{\frac{\hbar\omega_{\lambda}}{2}}
    (\hat{a}_{\lambda}-\hat{a}^{\dagger}_{\lambda})
\end{align}
これらを\eqref{elmg_Hamiltonian}へ代入すると
\begin{equation}
    \hat{H} = \sum_{\bm{k},\sigma} 
    \hbar\omega_{\lambda}\left(
    \hat{a}^{\dagger}_{\lambda}\hat{a}_{\lambda} + \frac{1}{2}
    \right)
\end{equation}
が得られる.電磁場の正の振動成分は
\begin{align}
    \hat{\bm{A}}^{+}_{\lambda}
    &=\frac{1}{\sqrt{4\epsilon_0\omega_{\lambda}^2 V}}
    (\omega_{\lambda} \hat{q}_{\lambda}+i\hat{p}_{\lambda})\vec{e}_{\lambda}\\[10pt]
    &=\frac{1}{\sqrt{4\epsilon_0\omega_{\lambda}^2 V}}
    \left[
    \omega_{\lambda} \sqrt{\frac{\hbar}{2\omega_{\lambda}}}
    (\hat{a}_{\lambda}+\hat{a}^{\dagger}_{\lambda})
    +i(-i)\sqrt{\frac{\hbar\omega_{\lambda}}{2}}
    (\hat{a}_{\lambda}-\hat{a}^{\dagger}_{\lambda})
    \right]\vec{e}_{\lambda}\nn[10pt]
    &=\frac{1}{\sqrt{4\epsilon_0\omega_{\lambda}^2 V}}
    \left[
    \sqrt{\frac{\hbar\omega_{\lambda}}{2}}
    (\hat{a}_{\lambda}+\hat{a}^{\dagger}_{\lambda})
    +\sqrt{\frac{\hbar\omega_{\lambda}}{2}}
    (\hat{a}_{\lambda}-\hat{a}^{\dagger}_{\lambda})
    \right]\vec{e}_{\lambda}\nn[10pt]
    &=\frac{1}{\sqrt{4\epsilon_0\omega_{\lambda}^2 V}}
    \sqrt{2\hbar\omega_{\lambda}}
    \hat{a}_{\lambda}
    \vec{e}_{\lambda}\nn[10pt]
    %
    &=\sqrt{\frac{\hbar}{2\epsilon_0\omega_{\lambda} V}}
    \hat{a}_{\lambda}
    \vec{e}_{\lambda}
\end{align}
となる.同様にして,負の振動成分は,
\begin{equation}
    \hat{\bm{A}}^{-}_{\lambda}
    =\sqrt{\frac{\hbar}{2\epsilon_0\omega_{\lambda} V}}
    \hat{a}^{\dagger}_{\lambda}
    \vec{e}_{\lambda}
\end{equation}
と量子化される.これらの量子化により,ベクトル・ポテンシャル,電場,磁場は以下のように量子化される:
\begin{align}
    \hat{\bm{A}}(\bm{r},t)&=
    \sum_{\bm{k},\sigma}\sqrt{\frac{\hbar}{2\epsilon_0\omega_{\lambda} V}}\vec{e}_{\lambda}
    \left[
    \hat{a}_{\lambda}e^{i(\bm{k}\cdot\bm{r}-\omega_{\lambda})}
    + \hat{a}^{\dagger}_{\lambda}e^{-i(\bm{k}\cdot\bm{r}-\omega_{\lambda})}
    \right],\\[10pt]
    %
    \hat{\bm{E}}(\bm{r},t)&=
    \sum_{\bm{k},\sigma}i\sqrt{\frac{\hbar\omega_{\lambda}}{2\epsilon_0 V}}\vec{e}_{\lambda}
    \left[
    \hat{a}_{\lambda}e^{i(\bm{k}\cdot\bm{r}-\omega_{\lambda})}
    - \hat{a}^{\dagger}_{\lambda}e^{-i(\bm{k}\cdot\bm{r}-\omega_{\lambda})}
    \right],\\[10pt]
    %
    \hat{\bm{B}}(\bm{r},t)&=
    \sum_{\bm{k},\sigma}i\sqrt{\frac{\hbar}{2\epsilon_0\omega_{\lambda} V}}\bm{k}\times\vec{e}_{\lambda}
    \left[
    \hat{a}_{\lambda}e^{i(\bm{k}\cdot\bm{r}-\omega_{\lambda})}
    - \hat{a}^{\dagger}_{\lambda}e^{-i(\bm{k}\cdot\bm{r}-\omega_{\lambda})}
    \right]
\end{align}


\subsection{連続モードの場合}










生成演算子$\hat{a}^{\dagger}$と個数状態$\ket{n}$の間には次のような関係式を満たす:
\begin{align}
    \ha^\dag\ket{n}&=\sqrt{n+1}\ket{n+1}
\end{align}
これを用いて,$\hat{a}^{\dagger}$を行列表示すると
\begin{align}
    \hat{a}^{\dagger}=
    \left(
    \begin{array}{ccccc}
   \braket{0|\hat{a}^{\dagger}|0} & \braket{0|\hat{a}^{\dagger}|1}& \dots& \dots  & \dots\\
  \braket{1|\hat{a}^{\dagger}|0} &\braket{1|\hat{a}^{\dagger}|1}& \dots& \dots  & \dots\\
  \braket{2|\hat{a}^{\dagger}|0} &\braket{2|\hat{a}^{\dagger}|1}& \dots& \dots  & \dots\\
  \vdots & \vdots & \dots& \ddots & \vdots \\
  \dots & \dots& \dots& \dots  & \vdots
    \end{array}
    \right)
\end{align}

ここで,行列要素$\braket{n|\hat{a}^{\dagger}|m}$をあらわに書くと,
\begin{align}
    \braket{n|\hat{a}^{\dagger}|m}
    =\sqrt{m+1}\delta_{n,m+1}
\end{align}
と書けるから,$n=m+1$を満たすとき,有限の値を取り得る.それ以外の要素は0となる.これを行列表示として書くと次のようになる:

\begin{align}
    \hat{a}^{\dagger}=
    \left(
    \begin{array}{ccccccc}
   0& 0& 0&0&0& \dots  & \dots\\
  1 &0& 0&0&0& \dots  & \dots\\
  0&\sqrt{2}& 0&0&0& \dots  & \dots\\
  0&0&\sqrt{3}&0& \dots& \dots  & \dots\\
  \vdots & \vdots &\vdots & \vdots & \dots& \ddots & \vdots \\
  \dots & \dots&\dots & \dots& \dots& \dots  & \vdots
    \end{array}
    \right)
\end{align}
となる.



% \part{量子光学的状態}
\section{コヒーレント状態}
\begin{equation}
    \ha\ket{\alpha}=\alpha\ket{\alpha}
\end{equation}
$\alpha\in\mathbb{C}$である.\\
 1modeの調和振動子の光子数状態に関して,成り立つ以下の式
\begin{align}
    \ha^\dag\ket{n}&=\sqrt{n+1}\ket{n+1},\ \ \ \ \ha\ket{n}=\sqrt{n}\ket{n-1}\\[10pt]
    \bra{n}\ha&=\sqrt{n+1}\bra{n+1},\ \ \ \ \bra{n}\ha^\dag=\sqrt{n}\bra{n-1}
\end{align}
を用いると
\begin{equation}
    \braket{n|\alpha}=\frac{1}{\sqrt{n}}\braket{n-1|\ha|\alpha}
    =\frac{\alpha}{\sqrt{n}}\braket{n-1|\alpha}
\end{equation}
が成り立つ.これを$n$について繰り返すと
\begin{equation}
    \braket{n|\alpha}
    =\frac{\alpha}{\sqrt{n}}\braket{n-1|\alpha}
    =\frac{\alpha^2}{\sqrt{(n)(n-1)}}\braket{n-2|\alpha}
    =\cdots=\frac{\alpha^n}{\sqrt{n!}}\braket{0|\alpha}
\end{equation}

$\ket{\alpha}$を$\{\ket{n}\}$で展開すると
\begin{equation}
    \ket{\alpha}=\sum_{n=0}^{\infty}\ket{n}\braket{n|\alpha}
    =\braket{0|\alpha}\sum_{n=0}^{\infty}\frac{\alpha^n}{\sqrt{n!}}\ket{n}
\end{equation}
となるので,$\ket{\alpha}$のノルムの2乗は
\begin{align}
    \|\ket{\alpha}\|^2
    &=\braket{\alpha|\alpha}=
    \left(\braket{\alpha|0}\sum_{m=0}^{\infty}\frac{(\alpha^\ast)^m}{\sqrt{m!}}\bra{m}\right)
    \left(\braket{0|\alpha}\sum_{n=0}^{\infty}\frac{\alpha^n}{\sqrt{n!}}\ket{n}\right)\nn[10pt]
    &=\braket{\alpha|0}\braket{0|\alpha}\sum_{m=0}^{\infty}\frac{(\alpha^\ast)^m}{\sqrt{m!}}
   \sum_{n=0}^{\infty}\frac{\alpha^n}{\sqrt{n!}}\braket{m|n}\nn[10pt]
   &=|\braket{0|\alpha}|^2\sum_{n=0}^{\infty}\frac{(|\alpha|^2)^n}{n!}
   =|\braket{0|\alpha}|^2e^{|\alpha|^2}
\end{align}
$\braket{\alpha|\alpha}=1$を要請すると,
\begin{equation}
    \braket{0|\alpha}=\exp{(-|\alpha|^2/2)}
\end{equation}
したがって,
\begin{equation}
    \ket{\alpha}
    =\braket{0|\alpha}\sum_{n=0}^{\infty}\frac{\alpha^n}{\sqrt{n!}}\ket{n}
    =\exp{(-|\alpha|^2/2)}\sum_{n=0}^{\infty}\frac{(\alpha\ha^\dag)^n}{{n!}}\ket{0}
    =\exp{(-|\alpha|^2/2)}e^{\alpha\ha^\dag}\ket{0}
\end{equation}
ここで
\begin{equation}
    \ket{n}=\frac{(\ha)^\dag}{\sqrt{n!}}\ket{0}
\end{equation}
を使った.
\begin{equation}
    \ket{0}=\exp{(-\alpha^\ast\ha)}\ket{0}
\end{equation}
であることに注意すると,
\begin{align}
    \ket{\alpha}&=e^{-|\alpha|^2/2}e^{\alpha\ha^\dag}e^{-\alpha^\ast\ha}\ket{0}\nn[10pt]
    &=e^{-|\alpha|^2/2}e^{\alpha\ha^\dag-\alpha^\ast\ha+|\alpha|^2[\ha,\ha^\dag]/2}\ket{0}\nn[10pt]
    &=e^{-|\alpha|^2/2}e^{\alpha\ha^\dag-\alpha^\ast\ha}e^{|\alpha|^2/2}\ket{0}
    =e^{\alpha\ha^\dag-\alpha^\ast\ha}\ket{0}
\end{align}

\subsection{直交性と完全性}
2つのコヒーレント状態$\ket{\alpha}$,$\ket{\beta}$との内積を求めると,
\begin{align}
    \braket{\alpha|\beta}&=\braket{\alpha|0}\braket{0|\beta}
    \sum_{m=0}^{\infty}\frac{(\alpha^\ast)^m}{\sqrt{m!}}
   \sum_{n=0}^{\infty}\frac{\beta^n}{\sqrt{n!}}\braket{m|n}\nn[10pt]
   &=\exp{(-|\alpha|^2/2)}\exp{(-|\beta|^2/2)}
   \sum_{n=0}^{\infty}\frac{(\alpha^\ast\beta)^n}{{n!}}\nn[10pt]
   &=\exp{(-|\alpha|^2/2-|\beta|^2/2+\alpha^\ast\beta)}
\end{align}
固有値の異なるコヒーレント状態は直交しないことがわかる.ここで,$\beta=-\alpha$とおくと,
\begin{align}
    \braket{\alpha|-\alpha}
   &=\exp{(-|\alpha|^2/2-|\alpha|^2/2-|\alpha|^2)}=e^{-2|\alpha|^2}
\end{align}
となる.$\alpha\to\infty$ならば,$\braket{\alpha|-\alpha}\to0$,つまり直交することがわかる.\\
 次に完全性について計算を行う.
\begin{align}
    \int\ket{\alpha}\bra{\alpha}d^2\alpha
    &=\sum_{m,n}\int d^2\alpha\ket{m}\braket{m|\alpha}\braket{\alpha|n}\bra{n}\nn[10pt]
    &=\sum_{m,n}\int d^2\alpha\ket{m}\bra{n}\frac{\alpha^m\alpha^n}{\sqrt{m!n!}}e^{-|\alpha|^2}\nn[10pt]
\end{align}
ここで,$\alpha=re^{i\theta}$,$d^2\alpha=rd\theta dr$と変数変換し,$|\alpha|^2=|r|^2|e^{i\theta}|^2=|r^2|$が成り立つから,
\begin{align}
    \int\ket{\alpha}\bra{\alpha}d^2\alpha
    &=\sum_{m,n}\frac{1}{\sqrt{m!n!}}\ket{m}\bra{n}\int d^2\alpha\alpha^m(alpha^\ast)^n e^{-|\alpha|^2}\nn[10pt]
    &=\sum_{m,n}\frac{1}{\sqrt{m!n!}}\ket{m}\bra{n}\int_0^{\infty}r\cdot dr
    \int_{0}^{2\pi}d\theta\ 
    r^me^{i m\theta} r^n e^{-in\theta} e^{-r^2}\nn[10pt]
    &=\sum_{m,n}\frac{1}{\sqrt{m!n!}}\ket{m}\bra{n}\int_0^{\infty}dr\ e^{-r^2} r^{(m+n+1)}
    \int_{0}^{2\pi}d\theta\ 
    e^{i(m-n)\theta}
\end{align}
まず$\theta$についての積分を実行すると
\begin{align}
    \int_{0}^{2\pi}d\theta\ 
    e^{i(m-n)\theta}=\left[\frac{1}{i(m-n)}e^{i(m-n)\theta}\right]_0^{2\pi}
    =\frac{1}{i(m-n)}(e^{2\pi i(m-n)\theta}-1=\delta_{m,n}
\end{align}
\begin{align}
    \int\ket{\alpha}\bra{\alpha}d^2\alpha
    &=\sum_{m,n}\frac{1}{\sqrt{m!n!}}\ket{m}\bra{n}\int_0^{\infty}dr\ e^{-r^2} r^{(m+n+1)}2\pi\delta_{m,n}\nn[10pt]
    &=2\pi\sum_{n}\frac{1}{n!}\ket{n}\bra{n}\int_0^{\infty}dr\ e^{-r^2} r^{(2n+1)}\nn[10pt]
    &=\sum_{n}\frac{1}{n!}\ket{n}\bra{n}\cdot 2\pi \cdot \frac{n!}{2}
    =\pi
\end{align}
ここで,積分公式
\begin{align}
    \int_0^{\infty}dx\ e^{-ax^2} x^{(2n+1)}&=\frac{n!}{2a^{n+1}}\\[10pt]
\end{align}
を使った.よって,コヒーレント状態の完全性条件は
\begin{equation}
    \frac{1}{\pi}\int d^2\alpha \ket{\alpha}\bra{\alpha}=1
\end{equation}
これを用いるとコヒーレント状態をコヒーレント状態で展開できることがわかる.任意の状態$\ket{\psi}$を
\begin{equation}
    \ket{\psi}=\frac{1}{\pi}\int\ket{\beta}\braket{\beta|\psi}d^2\beta
\end{equation}
と展開し,$\ket{\psi}\to\ket{\alpha}$に置き換えると
\begin{equation}
    \ket{\alpha}=\frac{1}{\pi}\int\ket{\beta}\braket{\beta|\alpha}d^2\beta
    =\frac{1}{\pi}\int\ket{\beta}\exp{(-|\alpha|^2/2-|\beta|^2/2+\alpha\beta^\ast)}d^2\beta
\end{equation}
となる.これを過剰完全性と呼ぶ.


\subsection{光子数期待値}
状態がコヒーレント状態$\ket{\alpha}$を取る場合の平均光子数$\bar{n}$を計算する.まず平均光子数は,
\begin{align}
    \bar{n} =\braket{\alpha|\hat{n}|\alpha} = \braket{\alpha|\hat{a}^{\dagger}\hat{a}|\alpha} = |\alpha|^2
\end{align}
となる.また,平均光子数の分散は
\begin{align}
    \braket{\alpha|\hat{n}^2|\alpha}
    &=\braket{\alpha|\hat{a}^{\dagger}\hat{a}\hat{a}^{\dagger}\hat{a}|\alpha}\nn[10pt]
    &=|\alpha|^2\braket{\alpha|\hat{a}\hat{a}^{\dagger}|\alpha}\nn[10pt]
    &=|\alpha|^2\braket{\alpha|(\hat{a}^{\dagger}\hat{a}+1)|\alpha}\nn[10pt]
    &=|\alpha|^2\left\{
    \braket{\alpha|\hat{a}^{\dagger}\hat{a}|\alpha} + 1
    \right\}\nn[10pt]
    &=|\alpha|^2\cdot|\alpha|^2 + |\alpha|^2
    =|\alpha|^2(|\alpha|^2 + 1 )
\end{align}
であるから,
\begin{align}
    (\Delta n)^2 &= \braket{\alpha|\hat{n}^2|\alpha} - \bar{n}^2\nn[10pt]
    &=|\alpha|^2(|\alpha|^2 + 1 ) - |\alpha|^4 = |\alpha|^2
\end{align}
となる.すなわち,コヒーレント状態は光子数の分散がその期待値に等しい:
\begin{equation}
    (\Delta n)^2 = \bar{n}
\end{equation}

コヒーレント状態の定義から,小悲恋と状態に$n$個の光子が見つかる確率は,
\begin{equation}
    |\braket{n|\alpha}|^2 
    = \left|e^{-|\alpha|^2/2}\sum_{m=0}^{\infty}\frac{\alpha^m}{\sqrt{m!}}\braket{n|m}\right|^2
    =e^{-|\alpha|^2}\frac{(|\alpha|^2)^n}{n!}
\end{equation}
となる.これはポアソン分布になっている.

\subsection{電磁場の期待値と分散}
コヒーレント状態に対する電場演算子と磁場演算子
\begin{align}
    \hat{\bm{E}}(\bm{r},t)&=
    \sum_{\bm{k},\sigma}i\sqrt{\frac{\hbar\omega_{\lambda}}{2\epsilon_0 V}}\vec{e}_{\lambda}
    \left[
    \hat{a}_{\lambda}e^{i(\bm{k}\cdot\bm{r}-\omega_{\lambda})}
    - \hat{a}^{\dagger}_{\lambda}e^{-i(\bm{k}\cdot\bm{r}-\omega_{\lambda})}
    \right],\\[10pt]
    %
    \hat{\bm{B}}(\bm{r},t)&=
    \sum_{\bm{k},\sigma}i\sqrt{\frac{\hbar}{2\epsilon_0\omega_{\lambda} V}}\bm{k}\times\vec{e}_{\lambda}
    \left[
    \hat{a}_{\lambda}e^{i(\bm{k}\cdot\bm{r}-\omega_{\lambda})}
    - \hat{a}^{\dagger}_{\lambda}e^{-i(\bm{k}\cdot\bm{r}-\omega_{\lambda})}
    \right]
\end{align}
の期待値と分散を計算すると,電場については
\begin{align}
    \braket{\alpha|\hat{\bm{E}}(\bm{r},t)|\alpha}
    &=\sqrt{\frac{2\hbar\omega}{\epsilon_0V}}|\alpha|\sin{\omega t-\bm{k}\cdot\bm{r}-\theta}\\[10pt]
    (\Delta E)^2
    & = \frac{\hbar\omega}{2\epsilon_0V}\\[10pt]
    \Delta E
    & = \sqrt{\frac{\hbar\omega}{2\epsilon_0V}}
\end{align}
磁場については
\begin{align}
    \braket{\alpha|\hat{\bm{B}}(\bm{r},t)|\alpha}
\end{align}
となる.

この計算を電場についてだけ証明しておこう:
\begin{equation}
    \hat{E}=
    i\sqrt{\frac{\hbar\omega}{2\epsilon_0V}}
    \left(
    \hat{a}e^{-i(\omega t-\bm{k}\cdot\bm{r})}
    -\hat{a}^{\dagger}e^{i(\omega t-\bm{k}\cdot\bm{r})}
    \right)
\end{equation}

\begin{align}
    \hat{E}^2&=
    -\frac{\hbar\omega}{2\epsilon_0V}
    \left\{
    \left(
    \hat{a}e^{-i(\omega t-\bm{k}\cdot\bm{r})}
    -\hat{a}^{\dagger}e^{i(\omega t-\bm{k}\cdot\bm{r})}
    \right)
    \left(
    \hat{a}e^{-i(\omega t-\bm{k}\cdot\bm{r})}
    -\hat{a}^{\dagger}e^{i(\omega t-\bm{k}\cdot\bm{r})}
    \right)
    \right\}\nn[10pt]
    &=
    -\frac{\hbar\omega}{2\epsilon_0V}
    \left(
    \hat{a}^2e^{-2i(\omega t-\bm{k}\cdot\bm{r})}
    +\hat{a}^{\dagger 2}e^{2i(\omega t-\bm{k}\cdot\bm{r})}
    -\hat{a}\hat{a}^{\dagger}
    -\hat{a}^{\dagger}\hat{a}
    \right)\nn[10pt]
\end{align}


\begin{align}
    \braket{\alpha|\hat{E}^2|\alpha}
    &=
    -\frac{\hbar\omega}{2\epsilon_0V}
    \Braket{\alpha|\left(
    \hat{a}^2e^{-2i(\omega t-\bm{k}\cdot\bm{r})}
    +\hat{a}^{\dagger 2}e^{2i(\omega t-\bm{k}\cdot\bm{r})}
    -\hat{a}\hat{a}^{\dagger}
    -\hat{a}^{\dagger}\hat{a}
    \right)|\alpha}\nn[10pt]
    &=
    -\frac{\hbar\omega}{2\epsilon_0V}
    \braket{\alpha|
    \hat{a}^2|\alpha}e^{-2i(\omega t-\bm{k}\cdot\bm{r})}
    +\braket{\alpha|
    \hat{a}^{\dagger 2}|\alpha}e^{2i(\omega t-\bm{k}\cdot\bm{r})}
    -\braket{\alpha|\hat{a}\hat{a}^{\dagger}|\alpha}
    -\braket{\alpha|\hat{a}^{\dagger}\hat{a}|\alpha}\nn[10pt]
    %
    &=
    -\frac{\hbar\omega}{2\epsilon_0V}
    \alpha^2
    e^{-2i(\omega t-\bm{k}\cdot\bm{r})}
    +\alpha^{\ast 2}
    e^{2i(\omega t-\bm{k}\cdot\bm{r})}
    -\braket{\alpha|\hat{a}^{\dagger}\hat{a}|\alpha}-1
    -\braket{\alpha|\hat{a}^{\dagger}\hat{a}|\alpha}\nn[10pt]
    %
    &=
    -\frac{\hbar\omega}{2\epsilon_0V}
    \left(
    \alpha^2
    e^{-2i(\omega t-\bm{k}\cdot\bm{r})}
    +\alpha^{\ast 2}
    e^{2i(\omega t-\bm{k}\cdot\bm{r})}
    -2|\alpha|^2-1
    \right)\nn[10pt]
\end{align}

we set $\Delta=\omega t-\bm{k}\cdot\bm{r}$
\begin{align}
    &\alpha^2
    e^{-2i\Delta}
    +\alpha^{\ast 2}
    e^{2i\Delta}
    -2|\alpha|^2-1
    \nn[10pt]
    &=|\alpha|^2e^{2i\theta}
    e^{-2i\Delta}
    +|\alpha|^2e^{-2i\theta}
    e^{2i\Delta}
    -2|\alpha|^2-1\nn[10pt]
    &=2|\alpha|^2
    \frac{\left(
    e^{-2i(\Delta-\theta)}
    +
    e^{2i(\Delta-\theta)}
    \right)}
    {2}
    -2|\alpha|^2-1\nn[10pt]
    &=2|\alpha|^2
    \cos{2(\Delta-\theta)}
    -2|\alpha|^2-1\nn[10pt]
    &=2|\alpha|^2
    (1-2\sin^2{(\Delta-\theta)})
    -2|\alpha|^2-1\nn[10pt]
    &=-4|\alpha|^2\sin^2{2(\Delta-\theta)})-1
\end{align}
we use $\alpha=|\alpha|^2e^{i\theta}$. So we obtain
\begin{align}
    \braket{\alpha|\hat{E}^2|\alpha}
    &=
    \frac{\hbar\omega}{2\epsilon_0V}
    \left(
    4|\alpha|^2\sin^2{(\omega t-\bm{k}\cdot\bm{r}-\theta)})+1
    \right)
\end{align}
Thus, 電磁場の分散は
\begin{align}
    (\Delta E)^2 & = \braket{\alpha|\hat{E}^2|\alpha} -(\braket{\alpha|\hat{E}|\alpha})^2\nn[10pt]
    &=\frac{\hbar\omega}{2\epsilon_0V}
    \left(
    4|\alpha|^2\sin^2{(\omega t-\bm{k}\cdot\bm{r}-\theta)})+1
    \right)
    -\frac{2\hbar\omega}{\epsilon_0V}|\alpha|^2\sin^2({\omega t-\bm{k}\cdot\bm{r}-\theta})\nn[10pt]
    &=\frac{\hbar\omega}{2\epsilon_0V}
\end{align}



\subsection{演算子の代数}
\paragraph{公式1}
演算子$\hat{A}$, $\hat{B}$が可換ではないとする.つまり$[\hat{A}, \hat{B}]\neq0$であるとする.このとき,演算子$\hat{A}$, $\hat{B}$, $\xi\in\mathbb{C}$に対して以下の公式が成り立つ:
\begin{equation}
    e^{\xi\hat{A}}\hat{B}e^{-\xi\hat{A}}
    =\hat{B}+\xi[\hat{A}, \hat{B}]
    +\frac{\xi^2}{2!}\bigl[
    \hat{A}, [\hat{A}, \hat{B}]
    \bigr]
    ++\frac{\xi^3}{3!}
    \Bigl[\hat{A},
    \bigl[
    \hat{A}, [\hat{A}, \hat{B}]
    \bigr]
    \Bigr]
    +\cdots
\end{equation}



\paragraph{公式2}
もし,$\bigl[\hat{A}, [\hat{A}, \hat{B}]\bigr]$, $\xi=1$ならば以下の公式が成り立つ:
\begin{equation}
    e^{\hat{A}+\hat{B}+\frac{1}{2}[\hat{A}, \hat{B}]}
    =e^{\hat{A}}e^{\hat{B}}
\end{equation}

\subsection{変位演算子}
変位演算子を
\begin{equation}
    \hat{D}(\alpha)=\exp{(\alpha\hat{a}^\dagger-\alpha^\ast\hat{a})}
\end{equation}
により導入する.ここで$\alpha\in \mathbb{C}$は無次元である.定義より
\begin{equation}
    \hat{D}^\dagger(\alpha)=\exp{(\alpha^\ast\hat{a}-\alpha\hat{a}^\dagger)}
    =\hat{D}(-\alpha)=\hat{D}^{-1}(\alpha)
\end{equation}
である.よって変位演算子$\hat{D}(\alpha)$は$\hat{D}(\alpha)\hat{D}^\dagger(\alpha)=\hat{I}$を満たしおり,ユニタリ演算子であることがわかる.

\subsection{Basic property of boost operator}
ここでは,変位演算子の性質について調べる.まず,変位演算子$\hat{D}(\alpha)$を真空状態$\ket{0}$に作用させる.その前に次の公式
\begin{equation}
    \exp{\left(\alpha\hat{a}^\dagger-\alpha^\ast\hat{a}\right)}
    =e^{\frac{1}{2}|\alpha|^2}e^{\alpha\hat{a}^\dagger}e^{-\alpha^\ast\hat{a}}
\end{equation}
を確認しておく.これは次のように示せる.演算子$\hat{A}=\alpha\hat{a}^\dagger$, $\hat{B}=-\alpha^\ast\hat{a}$に対して公式2を適用すると,
\begin{equation}
    \exp{\left(\alpha\hat{a}^\dagger-\alpha^\ast\hat{a}+\frac{1}{2}[\alpha\hat{a}^\dagger, -\alpha^\ast\hat{a}]\right)}
    =e^{\alpha\hat{a}^\dagger}e^{-\alpha^\ast\hat{a}}
\end{equation}

\begin{equation}
    \exp{\left(\alpha\hat{a}^\dagger-\alpha^\ast\hat{a}+\frac{1}{2}|\alpha|^2\right)}
    =e^{\alpha\hat{a}^\dagger}e^{-\alpha^\ast\hat{a}}
\end{equation}

\begin{align}
    [\alpha\hat{a}^\dagger, -\alpha^\ast\hat{a}]
    =-\alpha\alpha^\ast\hat{a}^\dagger\hat{a}
    +\alpha\alpha^\ast\hat{a}\hat{a}^\dagger
    =|\alpha|^2 [\hat{a}, \hat{a}^\dagger]=|\alpha|^2
\end{align}
である.

\begin{align}
    \hat{D}(\alpha)\ket{0}
    &=e^{\frac{1}{2}|\alpha|^2}e^{\alpha\hat{a}^\dagger}e^{-\alpha^\ast\hat{a}}\ket{0}\nn[10pt]
    &=e^{\frac{1}{2}|\alpha|^2}e^{\alpha\hat{a}^\dagger}
    \left(
    \hat{I}-\alpha^\ast\hat{a}+\frac{1}{2!}(-\alpha^\ast\hat{a})^2+\cdots
    \right)\ket{0}\nn[10pt]
    %
    &=e^{\frac{1}{2}|\alpha|^2}e^{\alpha\hat{a}^\dagger}\ket{0}
\end{align}
ここで,$\hat{a}\ket{0}=0$を使った.
\begin{align}
    \hat{D}(\alpha)\ket{0}
    &=e^{\frac{1}{2}|\alpha|^2}
    \left(
    \hat{I}+\alpha\hat{a}^\dagger+\frac{1}{2!}(\alpha\hat{a}^\dagger)^2+\cdots
    \right)\ket{0}\nn[10pt]
    %
    &=e^{\frac{1}{2}|\alpha|^2}\sum_{n=0}^{\infty}\frac{(\alpha)^n\sqrt{n!}}{n!}\ket{0}\nn[10pt]
    &=e^{\frac{1}{2}|\alpha|^2}\sum_{n=0}^{\infty}\frac{(\alpha)^n}{\sqrt{n!}}\ket{0}
    =\ket{\alpha}
\end{align}


次に変位演算子により$\hat{a}$, $\hat{a}^\dag$にユニタリ変換を施すと,生成(消滅)演算子$\hat{a}^\dagger$, $\hat{a}$を$\alpha$, $\alpha^\ast$だけ平行移動する効果がある:
\begin{align}
    \hat{D}^\dagger(\alpha)\hat{a}\hat{D}(\alpha)&=\hat{a} + \alpha\\[10pt]
    \hat{D}^\dagger(\alpha)\hat{a}^\dagger\hat{D}(\alpha)&=\hat{a}^\dagger + \alpha^\ast
\end{align}

これらの式を導出を行う.
変位演算子による$\hat{a}^{\dagger}$の変換を考える.これは,次の公式
\begin{equation}
    e^{\xi\hat{A}}\hat{B}e^{-\xi\hat{A}}
    =\hat{B}+\xi[\hat{A},\hat{B}]+\frac{\xi^2}{2!}[\hat{A},[\hat{A},\hat{B}]]+\cdots
\end{equation}
において,$\hat{A}=\alpha^\ast\hat{a}-\alpha\hat{a}^{\dagger}$,$\hat{B}=\hat{a}^{\dagger}$, $\xi=1$とおくことにより,次のように計算できる:
\begin{align}
    \hat{D}^{\dagger}(\alpha)\hat{a}^{\dagger}\hat{D}(\alpha)
    &=e^{\alpha^\ast\hat{a}-\alpha\hat{a}^{\dagger}}
    \hat{a}^{\dagger}
    e^{\alpha\hat{a}^{\dagger}-\alpha^\ast\hat{a}}\nn[10pt]
    &=\hat{a}^{\dagger}
    +[\alpha^\ast\hat{a}-\alpha\hat{a}^{\dagger},\hat{a}^{\dagger}]
    +\frac{1}{2}\left[\alpha^\ast\hat{a}-\alpha\hat{a}^{\dagger},[\alpha^\ast\hat{a}-\alpha\hat{a}^{\dagger},\hat{a}^{\dagger}]\right]\nn[10pt]
    &=\hat{a}^{\dagger}+\alpha^{\ast}
\end{align}
ここで,次の関係式
\begin{align}
    [\alpha^\ast\hat{a}-\alpha\hat{a}^{\dagger},\hat{a}^{\dagger}]
    &=(\alpha^\ast\hat{a}-\alpha\hat{a}^{\dagger})\hat{a}^{\dagger}
    -\hat{a}^{\dagger}(\alpha^\ast\hat{a}-\alpha\hat{a}^{\dagger})\nn[10pt]
    &=\alpha^\ast\hat{a}\hat{a}^{\dagger}-\alpha\hat{a}^{\dagger}\hat{a}^{\dagger}
    -\alpha^\ast\hat{a}^{\dagger}\hat{a}+\alpha\hat{a}^{\dagger}\hat{a}^{\dagger}\nn[10pt]
    &=\alpha^\ast\hat{a}\hat{a}^{\dagger}
    -\alpha^\ast\hat{a}^{\dagger}\hat{a}
    \alpha^\ast-\alpha^\ast[\hat{a},\hat{a}^{\dagger}]\nn[10pt]
\end{align}
と第3項の交換子が可換になることを用いた.$\hat{D}^\dagger(\alpha)\hat{a}\hat{D}(\alpha)=\hat{a}+\alpha$を同様に示せる.

$\hat{D}(\alpha)\ket{0}$は$\hat{a}$の固有状態であることも示せる:
\begin{equation}
    \hat{a}\hat{D}(\alpha)\ket{0}
    =\hat{D}(\alpha)\hat{D}^\dagger(\alpha)\hat{a}\hat{D}(\alpha)\ket{0}
    =\hat{D}(\alpha)(\hat{a}+\alpha)\ket{0}
    =\alpha\hat{D}(\alpha)\ket{0}
\end{equation}
ここで,$\hat{a}\ket{0}=0$を使った.

% 
\subsection{直交位相振幅(Quadrature phase amplitude)}



\section{スクイーズド状態 (Single-mode squeezed states)}
\subsection{スクイーズド Hamiltonianの対角化}
We discuss the eigenvalues and eigenvector of Hamiltonian
\begin{equation}
    \hH_{\rm{S}}=\hbar\omega\ha^\dag\ha + \hbar \left(\frac{E^\ast}{2}\ha^{2} + \frac{E}{2}\ha^{\dag2}\right).
\end{equation}
Bogoliubov transformation
\begin{align}
    \left( 
     \begin{array}{c}
     \hb\\[10pt]
     \hb^\dag
     \end{array}
    \right)
    =
    \left( 
     \begin{array}{cc}
     \xi_1&\xi_2 \\[10pt]
     \xi_2^{\ast}&\xi_1^{\ast}
     \end{array}
    \right)
    \left( 
     \begin{array}{c}
     \ha\\[10pt]
     \ha^\dag
     \end{array}
    \right)
\end{align}
この逆変換は
\begin{align}
    \left( 
     \begin{array}{c}
     \ha\\[10pt]
     \ha^\dag
     \end{array}
    \right)
    =
    \left( 
     \begin{array}{cc}
     \xi_1^{\ast}&-\xi_2 \\[10pt]
     -\xi_2^{\ast}&\xi_1
     \end{array}
    \right)
    \left( 
     \begin{array}{c}
     \hb\\[10pt]
     \hb^\dag
     \end{array}
    \right)
\end{align}

\begin{align}
    \ha&=\xi_1^{\ast}\hb-\xi_2\hb^\dag\\[10pt]
     \ha^\dag&=-\xi_2^{\ast}\hb+\xi_1\hb^\dag
\end{align}


\begin{align}
    \hH_{\rm{S}}&=\hbar\omega(-\xi_2^{\ast}\hb+\xi_1\hb^\dag)(\xi_1^{\ast}\hb-\xi_2\hb^\dag)\nn[10pt]
    &+ \hbar\frac{E^\ast}{2}(\xi_1^{\ast}\hb+-\xi_2\hb^\dag)(\xi_1^{\ast}\hb+-\xi_2\hb^\dag)\nn[10pt]
    &+ \hbar\frac{E}{2}(-\xi_2^{\ast}\hb+\xi_1\hb^\dag)(-\xi_2^{\ast}\hb+\xi_1\hb^\dag)
\end{align}

the first term
\begin{align}
    \hH_{\rm{S}1}/(\hbar\omega)&=(-\xi_2^{\ast}\hb+\xi_1\hb^\dag)(\xi_1^{\ast}\hb-\xi_2\hb^\dag)
    =-\xi_1^{\ast}\xi_2^{\ast}\hb^2 +|\xi_2|^2\hb\hb^\dag
    +|\xi_1|^2\hb^\dag\hb-\xi_1\xi_2\hb^{\dag2}\nn[10pt]
    &=|\xi_1|^2\hb^\dag\hb
    +|\xi_2|^2\hb^\dag\hb
    -\xi_1^{\ast}\xi_2^{\ast}\hb^2 -\xi_1\xi_2\hb^{\dag2}+|\xi_2|^2
\end{align}

\begin{align}
    \hH_{\rm{S}2}/(\hbar E^{\ast}/2)&=(\xi_1^{\ast}\hb-\xi_2\hb^\dag)(\xi_1^{\ast}\hb-\xi_2\hb^\dag)
    =-\xi_1^{\ast}\xi_2\hb^\dag\hb-\xi_1^{\ast}\xi_2\hb\hb^\dag
    +\xi_1^{\ast2}\hb^2+\xi_2^{2}\hb^{\dag2}\nn[10pt]
    &=-2\xi_1^{\ast}\xi_2\hb^\dag\hb
    +\xi_1^{\ast2}\hb^2+\xi_2^{2}\hb^{\dag2}-\xi_1^{\ast}\xi_2
\end{align}


\begin{align}
    \hH_{\rm{S}3}/(\hbar E/2)
    &=(-\xi_2^{\ast}\hb+\xi_1\hb^\dag)(-\xi_2^{\ast}\hb+\xi_1\hb^\dag)
    =-\xi_1\xi_2^{\ast}\hb^\dag\hb-\xi_1\xi_2^{\ast}\hb\hb^\dag
    +\xi_2^{\ast2}\hb^2+\xi_1^{2}\hb^{\dag2}\nn[10pt]
    &=-2\xi_1\xi_2^{\ast}\hb^\dag\hb
    +\xi_2^{\ast2}\hb^2+\xi_1^{2}\hb^{\dag2}-\xi_1\xi_2^{\ast}
\end{align}

\begin{align}
    \hH_{\rm{S}}&=\hbar\omega(|\xi_1|^2\hb^\dag\hb
    +|\xi_2|^2\hb^\dag\hb
    -\xi_1^{\ast}\xi_2^{\ast}\hb^2 -\xi_1\xi_2\hb^{\dag2}+|\xi_2|^2)\nn[10pt]
    &+ \hbar\frac{E^\ast}{2}(-2\xi_1^{\ast}\xi_2\hb^\dag\hb
    +\xi_1^{\ast2}\hb^2+\xi_2^{2}\hb^{\dag2}-\xi_1^{\ast}\xi_2)\nn[10pt]
    &+ \hbar\frac{E}{2}(-2\xi_1\xi_2^{\ast}\hb^\dag\hb
    +\xi_2^{\ast2}\hb^2+\xi_1^{2}\hb^{\dag2}-\xi_1\xi_2^{\ast})\nn[15pt]
    %
    &=\hbar\left\{\omega(|\xi_1|^2+|\xi_2|^2)\hb^\dag\hb
    +\frac{E^\ast}{2}(-2\xi_1^{\ast}\xi_2\hb^\dag\hb)
    +\frac{E}{2}(-2\xi_1\xi_2^{\ast}\hb^\dag\hb)
    \right\}
    \nn[10pt]
    &+\hbar\left\{\omega(-\xi_1^{\ast}\xi_2^{\ast}\hb^2)
    +\frac{E^\ast}{2}(\xi_1^{\ast2}\hb^2)
    +\frac{E}{2}(\xi_2^{\ast2}\hb^2)
    \right\}\nn[10pt]
    &+\hbar\left\{\omega(-\xi_1\xi_2\hb^{\dag2})
    +\frac{E^\ast}{2}(\xi_2^{2}\hb^{\dag2})
    +\frac{E}{2}(\xi_1^{2}\hb^{\dag2})
    \right\}\nn[10pt]
    &+\hbar\left\{\omega|\xi_2|^2
    +\frac{E^\ast}{2}(-\xi_1^{\ast}\xi_2)
    +\frac{E}{2}(-\xi_1\xi_2^{\ast})
    \right\}
\end{align}
このHaniltonianを対角化するためには,非対角項が
\begin{align}
\label{squeez_offdiag_1}
    -\omega\xi_1^{\ast}\xi_2^{\ast}
    +\frac{E^\ast}{2}\xi_1^{\ast2}
    +\frac{E}{2}\xi_2^{\ast2}&=0\\[10pt]
    %
\label{squeez_offdiag_2}
    -\omega\xi_1\xi_2
    +\frac{E^\ast}{2}\xi_2^{2}
    +\frac{E}{2}\xi_1^{2}&=0
\end{align}
を満たすような$\xi_1$,$\xi_2$を満たす必要がある.\eqref{squeez_offdiag_1}の両辺に$E^\ast/\xi_1^2$をかけると,$\xi_2/\xi_1$に関する2次方程式となる:
\begin{align}
    E^{\ast2}\left(\frac{\xi_2}{\xi_1}\right)^{2}-2\omega E^{\ast}\frac{\xi_2}{\xi_1}
    +|E|^2
    &=0.
\end{align}
これを解くと
\begin{align}
    \frac{\xi_2}{\xi_1}&=\frac{\omega E^{\ast}\pm\sqrt{\omega^2E^{\ast2}-E^{\ast2}|E|^2}}{E^{\ast2}}\nn[10pt]
    &=\frac{\omega \pm\sqrt{\omega^2-|E|^2}}{E^{\ast}}
    =\frac{\omega \pm\lambda}{E^{\ast}},
\end{align}
where $\lambda=\sqrt{\omega^2-|E|^2}$を得る.2次方程式の解の公式を用いた:
\begin{align}
    ax^2+bx+c=0,\\[10pt]
    x=\frac{-b\pm\sqrt{b^2-4ac}}{2a}.
\end{align}
同様にして,\eqref{squeez_offdiag_2}の両辺に$E/\xi_1^{\ast2}$をかけると
\begin{align}
    E^{2}\left(\frac{\xi_2^{\ast}}{\xi_1^{\ast}}\right)^{2}-2\omega E\frac{\xi_2^{\ast}}{\xi_1^{\ast}}
    +|E|^2
    &=0
\end{align}
\begin{align}
    \frac{\xi_2^{\ast}}{\xi_1^{\ast}}
    &=\frac{\omega \pm\sqrt{\omega^2-|E|^2}}{E}
    =\frac{\omega \pm\lambda}{E}
\end{align}
where $\lambda=\sqrt{\omega^2-|E|^2}$.よって,
\begin{align}
    \frac{|\xi_2|^2}{|\xi_1|^2}
    &=\frac{(\omega \pm\lambda)^2}{|E|^2}
    =\frac{-(\lambda\pm\omega)^2}{(\lambda^2-\omega^2)}
    =\frac{-(\lambda\pm\omega)}{(\lambda\mp\omega)}
\end{align}
\begin{align}
    -\frac{|\xi_2|^2}{|\xi_1|^2}
    =\frac{(\lambda\pm\omega)}{(\lambda\mp\omega)}
\end{align}
を得る.Using $|\xi_1|^2-|\xi_2|^2=1$, we obtain
\begin{align}
    \frac{1}{|\xi_1|^2}&=1+\left(-\frac{|\xi_2|^2}{|\xi_1|^2}\right)
    =1+\frac{(\lambda\pm\omega)}{(\lambda\mp\omega)}=\frac{2\lambda}{\lambda\mp\omega}
\end{align}
Therefore we have
\begin{align}
    |\xi_1|^2&=\frac{\lambda\mp\omega}{2\lambda}\\[10pt]
    |\xi_2|^2&=\frac{\lambda\mp\omega}{2\lambda}-1=\frac{\lambda\mp\omega-2\lambda}{2\lambda}
    =\frac{-\lambda\mp\omega}{2\lambda}.
\end{align}


ここで,各項は,
\begin{align}
    \hH_{\rm{S}}
    &=\hbar\left\{\omega(|\xi_1|^2+|\xi_2|^2)\hb^\dag\hb
    +\frac{E^\ast}{2}(-2\xi_1^{\ast}\xi_2\hb^\dag\hb)
    +\frac{E}{2}(-2\xi_1\xi_2^{\ast}\hb^\dag\hb)
    \right\}
\end{align}




\begin{align}
    {\mathrm{1st\ term}}
    &=|\xi_1|^2+|\xi_2|^2=\frac{\lambda\mp\omega}{2\lambda}+\frac{-\lambda\mp\omega}{2\lambda}
    =\frac{\mp2\omega}{2\lambda}=\mp\frac{\omega}{\lambda}
\end{align}






\begin{align}
    {\mathrm{2nd\ term}}
    %
    &=\frac{E^\ast}{2}(-2\xi_1^{\ast}\xi_2)
    =-E^\ast\xi_1^{\ast}\xi_2
    =-E^\ast|\xi_1|^2\frac{\xi_2}{\xi_1}
    =-E^\ast\frac{\lambda\mp\omega}{2\lambda}\frac{\omega\pm\lambda}{E^\ast}\nn[10pt]
    &=\frac{(\pm\omega-\lambda)(\omega\pm\lambda)}{2\lambda}
    =\frac{\omega^2-\lambda^2}{2\lambda}=\frac{|E|^2}{2\lambda}
\end{align}

\begin{align}
    {\mathrm{3rd\ term}}
    %
    &=\frac{E}{2}(-2\xi_1\xi_2^{\ast})
    =-E|\xi_1|^2\frac{\xi_2^{\ast}}{\xi_1^{\ast}}
    =-E \frac{\lambda\mp\omega}{2\lambda}\frac{\omega\pm\lambda}{E}\nn[10pt]
    &=\frac{(\pm\omega-\lambda)(\omega\pm\lambda)}{2\lambda}
    =\frac{\omega^2-\lambda^2}{2\lambda}=\frac{|E|^2}{2\lambda}
\end{align}
と計算できるので,Squeezed Hamiltonianは次のように対角化できる:
\begin{align}
    \hH_{\rm{S}}
    &=\hbar\left\{\mp\frac{\omega^2}{\lambda}
    \pm\frac{|E|^2}{2\lambda}
    \pm\frac{|E|^2}{2\lambda}
    \right\}\hb^\dag\hb
    %
    =\hbar\left\{\frac{\mp\omega^2\pm|E|^2}{\lambda}
    \right\}\hb^\dag\hb
    =\mp\hbar\lambda\hb^\dag\hb
\end{align}

\subsection{スクイーズド状態の演算子形式}
スクイーズ演算子を
\begin{equation}
    \hat{S}(\zeta)\equiv
    \exp{
    \left\{
    \frac{1}{2}(\zeta^\ast\hat{a}^2-\zeta\hat{a}^{\dagger 2})
    \right\}
    },\ \ \ \zeta = re^{i\varphi}
\end{equation}
により導入する.定義より
\begin{equation}
    \hat{S}^\dagger(\zeta)=\hat{S}(-\zeta)=\hat{S}^{-1}(\zeta)
\end{equation}
が確認できる.つまりスクイーズ演算子はユニタリ演算子であることがわかる.

一般スクイーズ状態はスクイーズ演算子$\hat{S}(\zeta)$を用いて,以下のように定義される:
\begin{equation}
    \ket{\zeta,\alpha}\equiv\hat{S}(\zeta)\ket{\alpha}
    =\hat{S}(\zeta)\hat{D}(\alpha)\ket{0}
\end{equation}


また,スクイーズ演算子による生成消滅演算子$\hat{a}^\dagger$, $\hat{a}$のユニタリ変換は
\begin{align}
    \hat{S}^\dagger(\zeta)\hat{a}\hat{S}(\zeta)&=
    \hat{a}\cosh{r} - \hat{a}^\dagger e^{i\varphi}\sinh{r}\\[10pt]
    \hat{S}^\dagger(\zeta)\hat{a}^{\dagger}\hat{S}(\zeta)&=
    \hat{a}^\dagger\cosh{r} - \hat{a} e^{-i\varphi}\sinh{r}
\end{align}
同様にして,
\begin{align}
    \hat{S}(\zeta)\hat{a}\hat{S}^\dagger(\zeta)&=
    \hat{a}\cosh{r} + \hat{a}^\dagger e^{i\varphi}\sinh{r}\\[10pt]
    \hat{S}(\zeta)\hat{a}^{\dagger}\hat{S}^\dagger(\zeta)&=
    \hat{a}^\dagger\cosh{r} + \hat{a} e^{-i\varphi}\sinh{r}
\end{align}


\begin{align}
    \hat{S}^\dagger(\zeta)\hat{a}\hat{S}(\zeta)
    &=e^{
    (\zeta\hat{a}^{\dagger 2}-\zeta^\ast\hat{a}^2)/2
    }\ 
    \hat{a}\ 
    e^{
    (\zeta^\ast\hat{a}^2-\zeta\hat{a}^{\dagger 2})/2
    }\nn[10pt]
    &=\hat{a}
    +\frac{1}{1!}
    \hat{C}^{(1)}
    +\frac{1}{2!}\hat{C}^{2}+\frac{1}{3!}\hat{C}^{(3)}+\cdots
\end{align}
ここで,入れ子になっている交換子を次のように定義した:
\begin{align}
    \hat{C}^{(1)}&\equiv[(\zeta\hat{a}^{\dagger 2}-\zeta^\ast\hat{a}^2)/2, \hat{a}]\\[10pt]
    \hat{C}^{(2)}&\equiv\Bigl[(\zeta\hat{a}^{\dagger 2}-\zeta^\ast\hat{a}^2)/2, 
    \hat{C}^{(1)}
    \Bigr]
    =\Bigl[(\zeta\hat{a}^{\dagger 2}-\zeta^\ast\hat{a}^2)/2, 
    [(\zeta\hat{a}^{\dagger 2}-\zeta^\ast\hat{a}^2)/2, \hat{a}]
    \Bigr]\\[10pt]
    %
    \hat{C}^{(3)}&\equiv\Bigl[(\zeta\hat{a}^{\dagger 2}-\zeta^\ast\hat{a}^2)/2, 
    \hat{C}^{(2)}
    \Bigr]
    =\biggl[(\zeta\hat{a}^{\dagger 2}-\zeta^\ast\hat{a}^2)/2, 
    \Bigl[(\zeta\hat{a}^{\dagger 2}-\zeta^\ast\hat{a}^2)/2, 
    [(\zeta\hat{a}^{\dagger 2}-\zeta^\ast\hat{a}^2)/2, \hat{a}]
    \Bigr]\biggr]\\[10pt]
    \vdots
\end{align}

次に交換子$\hat{C}^{(n)}$, $n=1,2,3,\ldots$の計算を実行する:
\begin{align}
    \hat{C}^{(1)}&\equiv[(\zeta\hat{a}^{\dagger 2}-\zeta^\ast\hat{a}^2)/2, \hat{a}]
    =\frac{1}{2}(
    \zeta\hat{a}^{\dagger 2}\hat{a}-\zeta^\ast\hat{a}^3
    -\zeta\hat{a}\hat{a}^{\dagger 2}+\zeta^\ast\hat{a}^3
    )
    =\frac{1}{2}(
    \zeta\hat{a}^{\dagger 2}\hat{a}-\zeta\hat{a}\hat{a}^{\dagger}\hat{a}^{\dagger}
    )\nn[10pt]
    &=\frac{\zeta}{2}(
    \hat{a}^{\dagger 2}\hat{a}-(\hat{a}^{\dagger}\hat{a}+1)\hat{a}^{\dagger}
    )
    =\frac{\zeta}{2}(
    \hat{a}^{\dagger 2}\hat{a}-\hat{a}^{\dagger}\hat{a}\hat{a}^{\dagger}-\hat{a}^{\dagger}
    )
    =\frac{\zeta}{2}(
    \hat{a}^{\dagger 2}\hat{a}-\hat{a}^{\dagger}(\hat{a}^{\dagger}\hat{a}+1)-\hat{a}^{\dagger}
    )
    \nn[10pt]
    &=\frac{\zeta}{2}(
    \hat{a}^{\dagger 2}\hat{a}-\hat{a}^{\dagger}\hat{a}^{\dagger}\hat{a}-2\hat{a}^{\dagger}
    )
    =-\zeta\hat{a}^{\dagger}
\end{align}

\begin{align}
    \hat{C}^{(2)}&\equiv\Bigl[(\zeta\hat{a}^{\dagger 2}-\zeta^\ast\hat{a}^2)/2, 
    \hat{C}^{(1)}
    \Bigr]
    =\Bigl[(\zeta\hat{a}^{\dagger 2}-\zeta^\ast\hat{a}^2)/2, 
    -\zeta\hat{a}^{\dagger}
    \Bigr]\nn[10pt]
    &=-\frac{1}{2}(
    \zeta^2\hat{a}^{\dagger 3}-|\zeta|^2\hat{a}^2\hat{a}^{\dagger}
    -\zeta^2\hat{a}^{\dagger 3}+|\zeta|^2\hat{a}^{\dagger}\hat{a}^2
    )
    =-\frac{|\zeta|^2}{2}(
    \hat{a}^{\dagger}\hat{a}^2-\hat{a}^2\hat{a}^{\dagger}
    )\nn[10pt]
    &=-\frac{|\zeta|^2}{2}(
    \hat{a}^{\dagger}\hat{a}^2-\hat{a}^2\hat{a}^{\dagger}
    )
    =-\frac{|\zeta|^2}{2}(
    \hat{a}^{\dagger}\hat{a}^2-\hat{a}^{\dagger}\hat{a}^2-2\hat{a}
    )
    \nn[10pt]
    &=|\zeta|^2\hat{a}
\end{align}


\begin{align}
    \hat{C}^{(3)}&\equiv\Bigl[(\zeta\hat{a}^{\dagger 2}-\zeta^\ast\hat{a}^2)/2, 
    \hat{C}^{(2)}
    \Bigr]
    =\Bigl[(\zeta\hat{a}^{\dagger 2}-\zeta^\ast\hat{a}^2)/2, 
    |\zeta|^2\hat{a}
    \Bigr]\nn[10pt]
    &=\frac{1}{2}(
    \zeta|\zeta|^2\hat{a}^{\dagger 2}\hat{a}-\zeta^\ast|\zeta|^2\hat{a}^3
    -\zeta|\zeta|^2\hat{a}\hat{a}^{\dagger 2}+\zeta^\ast|\zeta|^2\hat{a}^3
    )
    =\frac{\zeta|\zeta|^2}{2}(
    \hat{a}^{\dagger 2}\hat{a}-\hat{a}\hat{a}^{\dagger 2}
    )\nn[10pt]
    &=\frac{\zeta|\zeta|^2}{2}(
    \hat{a}^{\dagger 2}\hat{a}-\hat{a}^{\dagger 2}\hat{a}-2\hat{a}^\dagger
    )
    \nn[10pt]
    &=-\zeta|\zeta|^2\hat{a}^\dagger
\end{align}

\begin{align}
    \hat{C}^{(4)}&\equiv\Bigl[(\zeta\hat{a}^{\dagger 2}-\zeta^\ast\hat{a}^2)/2, 
    \hat{C}^{(3)}
    \Bigr]
    =\Bigl[(\zeta\hat{a}^{\dagger 2}-\zeta^\ast\hat{a}^2)/2, 
    -\zeta|\zeta|^2\hat{a}^\dagger
    \Bigr]\nn[10pt]
    &=-\frac{1}{2}(
    \zeta^2|\zeta|^2\hat{a}^{\dagger 2}\hat{a}^{\dagger}-|\zeta|^4\hat{a}^2\hat{a}^{\dagger}
    -\zeta^2|\zeta|^2\hat{a}^{\dagger}\hat{a}^{\dagger 2}+|\zeta|^4\hat{a}^{\dagger}\hat{a}^2
    )
    =-\frac{|\zeta|^4}{2}(
    \hat{a}^{\dagger}\hat{a}^2-\hat{a}^2\hat{a}^{\dagger}
    )\nn[10pt]
    &=-\frac{|\zeta|^4}{2}(
    \hat{a}^{\dagger}\hat{a}^2-\hat{a}^{\dagger}\hat{a}^2-2\hat{a}
    )
    \nn[10pt]
    &=|\zeta|^4\hat{a}
\end{align}


\begin{align}
    \hat{C}^{(5)}&\equiv\Bigl[(\zeta\hat{a}^{\dagger 2}-\zeta^\ast\hat{a}^2)/2, 
    \hat{C}^{(4)}
    \Bigr]
    =\Bigl[(\zeta\hat{a}^{\dagger 2}-\zeta^\ast\hat{a}^2)/2, 
    |\zeta|^4\hat{a}
    \Bigr]\nn[10pt]
    &=\frac{1}{2}(
    \zeta|\zeta|^4\hat{a}^{\dagger 2}\hat{a}-\zeta^\ast|\zeta|^4\hat{a}^3
    -\zeta|\zeta|^4\hat{a}\hat{a}^{\dagger 2}+\zeta^\ast|\zeta|^4\hat{a}^3
    )
    =\frac{\zeta|\zeta|^4}{2}(
    \hat{a}^{\dagger 2}\hat{a}-\hat{a}\hat{a}^{\dagger 2}
    )\nn[10pt]
    &=\frac{\zeta|\zeta|^4}{2}(
    \hat{a}^{\dagger 2}\hat{a}-\hat{a}^{\dagger 2}\hat{a}-2\hat{a}^\dagger
    )
    \nn[10pt]
    &=-\zeta|\zeta|^4\hat{a}^\dagger
\end{align}

したがって,一般形について
\begin{align}
    \hat{C}^{(n)}
    =\left\{
    \begin{array}{l}
    -\zeta|\zeta|^{n-1}\hat{a}^{\dagger},\ \ n=\rm{odd} \\[10pt]
    |\zeta|^{n}\hat{a},\ \ n=\rm{even}
    \end{array}
    \right.
\end{align}
が成り立つことがわかる.


\begin{align}
    \hat{S}^\dagger(\zeta)\hat{a}\hat{S}(\zeta)
    &=e^{
    (\zeta\hat{a}^{\dagger 2}-\zeta^\ast\hat{a}^2)/2
    }\ 
    \hat{a}\ 
    e^{
    (\zeta^\ast\hat{a}^2-\zeta\hat{a}^{\dagger 2})/2
    }\nn[10pt]
    &=\hat{a}
    -\frac{1}{1!}
    \zeta\hat{a}^{\dagger}
    +\frac{1}{2!}
    |\zeta|^{2}\hat{a}
    -\frac{1}{3!}
    \zeta|\zeta|^2\hat{a}^{\dagger}
    +\frac{1}{4!}
    |\zeta|^{4}\hat{a}
    -\frac{1}{5!}
    \zeta|\zeta|^4\hat{a}^{\dagger}
    +\cdots\nn[10pt]
    &=\hat{a}
    \left(
    1+\frac{1}{2!}
    |\zeta|^{2}+\frac{1}{4!}
    |\zeta|^{4}+\cdots
    \right)
    -\zeta\hat{a}^{\dagger}
    \left(
    \frac{1}{1!}
    |\zeta|^{0}
    +\frac{1}{3!}
    |\zeta|^2
    +\frac{1}{5!}
    |\zeta|^4
    +\cdots
    \right)\nn[10pt]
    %
    &=\hat{a}
    \left(
    1+\frac{1}{2!}r^2
    +\frac{1}{4!}
    r^{4}+\cdots
    \right)
    -\hat{a}^{\dagger}e^{i\varphi}
    \left(
    \frac{1}{1!}r
    +\frac{1}{3!}
    r^3
    +\frac{1}{5!}
    r^5
    +\cdots
    \right)\nn[10pt]
    &=\hat{a}\ 
    {\rm{cosh}}r
    -\hat{a}^{\dagger}e^{i\varphi}
    {\rm{sinh}}r
\end{align}

$\zeta=re^{i\varphi}$

\begin{align}
    {\rm{sinh}}x &= \frac{x}{1!}+\frac{x^3}{3!}+\frac{x^5}{5!}+\cdots\\[10pt]
    {\rm{cosh}}x &= 1+\frac{x^2}{2!}+\frac{x^4}{4!}+\cdots
\end{align}


% 
\section{平衡ホモダイン測定 (Balanced Homodyne detection)}
\subsection{Beam spliterの一般論}
今,被測定系Mとプローブ系Pを考える.また,被測定系の入力光,出力光を表す演算子をそれぞれ$\hat{a}_{\rm{M}}$, $\hat{a}_1$と表す.同様にして,プローブ系についても,$\hat{a}_{\rm{P}}$, $\hat{a}_2$と表す.

ビームスプリッターとは,入力光ビームが2つ$(\hat{a}_{\rm{P}},\ \hat{a}_{\rm{M}})$,出力光ビームが2つ$(\hat{a}_{1},\ \hat{a}_{2})$の2入力,2出力デバイスである.その出力関係は次の線形変換で記述される:
\begin{equation}
    \left(
        \begin{array}{c}
       \hat{a}_1\\[10pt]
       \hat{a}_2\\
        \end{array}
        \right)
        =B\left(
        \begin{array}{c}
       \hat{a}_{\rm{P}}\\[10pt]
       \hat{a}_{\rm{M}}\\
        \end{array}
        \right)
        =
        \left(
        \begin{array}{cc}
       B_{11}&B_{12}\\[10pt]
        B_{21}&B_{22} \\
        \end{array}
        \right)
        \left(
        \begin{array}{c}
       \hat{a}_{\rm{P}}\\[10pt]
       \hat{a}_{\rm{M}}\\
        \end{array}
        \right)
\end{equation}
$B$は$2\times2$行列であり,エネルギー保存の要請,つまり入出力の間で,光子数の変化はないことを示し,次式を満たす必要がある:
\begin{equation}\label{law_of_conservation}
    \hat{a}^{\dag}_1\hat{a}_1 + \hat{a}^{\dag}_2\hat{a}_2
    =\hat{a}^{\dag}_{\rm{P}}\hat{a}_{\rm{P}} + \hat{a}^{\dag}_{\rm{M}}\hat{a}_{\rm{M}}
\end{equation}

\begin{align}
    \hat{a}_1 &= B_{11}\hat{a}_{\rm{P}} + B_{12}\hat{a}_{\rm{M}}\\[10pt]
    \hat{a}_2 &= B_{21}\hat{a}_{\rm{P}} + B_{22}\hat{a}_{\rm{M}}
\end{align}

\begin{align}
    \hat{a}^{\dag}_1\hat{a}_1 + \hat{a}^{\dag}_2\hat{a}_2 
    &=
    (B_{11}^{\ast}\hat{a}^{\dag}_{\rm{P}} +B_{12}^{\ast}\hat{a}^{\dagger}_{\rm{M}})
    (B_{11}\hat{a}_{\rm{P}} + B_{12}\hat{a}_{\rm{M}})
    +(B_{21}^{\ast}\hat{a}^{\dag}_{\rm{P}} +B_{22}^{\ast}\hat{a}^{\dagger}_{\rm{M}})
    (B_{21}\hat{a}_{\rm{P}} + B_{22}\hat{a}_{\rm{M}})\nn[10pt]
    &=(|B_{11}|^2+|B_{21}|^2)\hat{a}^{\dagger}_{\rm{P}}\hat{a}_{\rm{P}}
    +(|B_{12}|^2+|B_{22}|^2)\hat{a}^{\dagger}_{\rm{M}}\hat{a}_{\rm{M}}\nn[10pt]
    &+(B^{\ast}_{11}B_{12}+B^{\ast}_{21}B_{22})\hat{a}^{\dagger}_{\rm{P}}\hat{a}_{\rm{M}}
    +(B_{11}B^{\ast}_{12}+B_{21}B^{\ast}_{22})\hat{a}^{\dagger}_{\rm{M}}\hat{a}_{\rm{P}}
\end{align}
よって,光子数の保存要請\eqref{law_of_conservation}から,行列$B$が満たすべき条件は次のようになる:
\begin{align}
    |B_{11}|^2+|B_{21}|^2 &= |B_{12}|^2+|B_{22}|^2 = 1\\[10pt]
    B^{\ast}_{11}B_{12}+B^{\ast}_{21}B_{22} &= B_{11}B^{\ast}_{12}+B_{21}B^{\ast}_{22}) = 0
\end{align}
この条件は,行列$B$がユニタリ行列$(B^{\dagger}B=I_2)$であることと等価である.これは損失のない過程を考えているので当然ともいえる.
任意の2次ユニタリ行列は,$\Lambda$, $\Psi$, $\Theta$, $\Phi$を任意の実数として,次のように書ける:
\begin{equation}
    B=e^{i\Lambda/2}
        \left(
        \begin{array}{cc}
       e^{i\Phi/2}&0\\[10pt]
        0&e^{-i\Phi/2} \\
        \end{array}
        \right)
        \left(
        \begin{array}{cc}
       \cos{\Theta/2}&\sin{\Theta/2}\\[10pt]
        -\sin{\Theta/2}&\cos{\Theta/2} \\
        \end{array}
        \right)
        \left(
        \begin{array}{cc}
       B_{11}&B_{12}\\[10pt]
        B_{21}&B_{22} \\
        \end{array}
        \right)
        \left(
        \begin{array}{cc}
       e^{i\Phi/2}&0\\[10pt]
        0&e^{-i\Phi/2} \\
        \end{array}
        \right)
\end{equation}
ここで,$\Lambda$, $\Psi$, $\Phi$は位相因子であるから,光学実験において適当に調整できるパラメータである.したがって,$\Lambda=\Psi=\Phi=0$とセットし,
\begin{equation}
    B=
        \left(
        \begin{array}{cc}
       \cos{\Theta/2}&\sin{\Theta/2}\\[10pt]
        -\sin{\Theta/2}&\cos{\Theta/2} \\
        \end{array}
        \right)
\end{equation}
とできる.透過率$T$, 反射率$R$をそれぞれ
\begin{align}
    T + R =1
\end{align}
て定義すると,行列$B$は
\begin{equation}
    B=
        \left(
        \begin{array}{cc}
       \sqrt{T}&-\sqrt{R}\\[10pt]
        \sqrt{R}&\sqrt{T}\\
        \end{array}
        \right)
\end{equation}

\subsection{バランス型ホモダイン検出}
ビームスプリッターとして,$50/50$ビームスプリッターを用いる.これは,$T=R=1/2$という意味である.つまり,ビームスプリッターによる変換を表す行列$B$は
\begin{equation}
    B=
        \left(
        \begin{array}{cc}
       \sqrt{1/2}&-\sqrt{1/2}\\[10pt]
        \sqrt{1/2}&\sqrt{1/2}\\
        \end{array}
        \right)
\end{equation}
となる.また,被測定系には単一モードの電磁場を利用し,$\hat{a}_{\rm{S}} = \hat{a}_{\rm{M}}$とする.(これは,シグナル光と呼ばれている.)プローブ系にも同様に単一モードの電磁場を利用し,$\hat{a}_{\rm{LO}} = \hat{a}_{\rm{P}}$とする.(こちらはLocal Oscillatorとよぶ.)このとき,入出力の関係は,
\begin{align}
    \hat{a}_1 &= \sqrt{T}\hat{a}_{\rm{LO}} - \sqrt{R}\hat{a}_{\rm{S}}\\[10pt]
    \hat{a}_2 &= \sqrt{T}\hat{a}_{\rm{S}} + \sqrt{R}\hat{a}_{\rm{LO}}
\end{align}
となる.ここで,測定するのは光の強さ(フォトン数)であり,測定の出力は光子検出器1,2でそれぞれ測定したフォトン数の差$\hat{R}\equiv\hat{n_2}-\hat{n_1}$で表される.ここで,$\hat{n}_1=\hat{a}^{\dag}_1\hat{a}_1$, $\hat{n}_2=\hat{a}^{\dag}_2\hat{a}_2$である.

\begin{align}
    \hat{R}&=\hat{n}_2-\hat{n}_1
    =\hat{a}^{\dag}_2\hat{a}_2-\hat{a}^{\dag}_1\hat{a}_1\nn[10pt]
    &=(\sqrt{T}\hat{a}^{\dagger}_{\rm{S}} + \sqrt{R}\hat{a}^{\dagger}_{\rm{LO}})
    (\sqrt{T}\hat{a}_{\rm{S}} + \sqrt{R}\hat{a}_{\rm{LO}})
    -(\sqrt{T}\hat{a}^{\dagger}_{\rm{LO}} - \sqrt{R}\hat{a}^{\dagger}_{\rm{S}})
    (\sqrt{T}\hat{a}_{\rm{LO}} - \sqrt{R}\hat{a}_{\rm{S}})\nn[10pt]
    &=(T-R)\hat{a}^{\dagger}_{\rm{S}}\hat{a}_{\rm{S}} 
    - (T-R)\hat{a}^{\dagger}_{\rm{LO}}\hat{a}_{\rm{LO}}
    +2\sqrt{TR}
    (\hat{a}^{\dagger}_{\rm{LO}}\hat{a}_{\rm{S}} + \hat{a}^{\dagger}_{\rm{S}}\hat{a}_{\rm{LO}})
\end{align}

ここで, $T=R=1/2$のとき,
\begin{align}
    \hat{R}=\hat{n}_2-\hat{n}_1
    =\hat{a}^{\dagger}_{\rm{LO}}\hat{a}_{\rm{S}} + \hat{a}^{\dagger}_{\rm{S}}\hat{a}_{\rm{LO}}
\end{align}

ビームスプリッターの入力状態,すなわち,全系の初期状態を$\ket{\Psi}\equiv\ket{\psi}_{\rm{LO}}\ket{\varphi}_{\rm{S}}$とすると,平均光子数は
\begin{align}
    \braket{\Psi|\hat{R}|\Psi}
    &=\bra{\varphi}_{\rm{S}}\bra{\psi}_{\rm{LO}}\hat{R}\ket{\varphi}_{\rm{S}}\ket{\psi}_{\rm{LO}}\nn[10pt]
    &=\bra{\varphi}_{\rm{S}}\bra{\psi}_{\rm{LO}}
    (\hat{a}^{\dagger}_{\rm{LO}}\hat{a}_{\rm{S}}
    +\hat{a}^{\dagger}_{\rm{S}}\hat{a}_{\rm{LO}})
    \ket{\varphi}_{\rm{S}}\ket{\psi}_{\rm{LO}}\nn[10pt]
    &=\braket{\psi|\hat{a}^{\dagger}_{\rm{LO}}|\psi}_{\rm{LO}}
    \cdot\braket{\varphi|\hat{a}_{\rm{S}}|\varphi}_{\rm{S}}
    +\braket{\psi|\hat{a}_{\rm{LO}}|\psi}_{\rm{LO}}
    \cdot\braket{\varphi|\hat{a}^{\dagger}_{\rm{S}}|\varphi}_{\rm{S}}
\end{align}
シグナル光$\ket{\varphi}_{\rm{S}}$では,スクイズド状態などさまざまな状態の光が入力として利用される.LOでは,コヒーレント状態$\ket{\alpha}$を用いる.すると,$\hat{a}_{\rm{LO}}\ket{\alpha}=\alpha\ket{\alpha}$より,
\begin{align}
    \braket{\Psi|\hat{R}|\Psi}
    &=\alpha^{\ast}\braket{\varphi|\hat{a}_{\rm{S}}|\varphi}_{\rm{S}}
    +\alpha\braket{\varphi|\hat{a}^{\dagger}_{\rm{S}}|\varphi}_{\rm{S}}
\end{align}
ここで,$\alpha=|\alpha|e^{i\theta_{\rm{LO}}}$とすると,$\braket{\hat{R}}$は,
\begin{align}
    \braket{\Psi|\hat{R}|\Psi}
    &=|\alpha|e^{-i\theta_{\rm{LO}}}\braket{\varphi|\hat{a}_{\rm{S}}|\varphi}_{\rm{S}}
    +|\alpha|e^{i\theta_{\rm{LO}}}\braket{\varphi|\hat{a}^{\dagger}_{\rm{S}}|\varphi}_{\rm{S}}\nn[10pt]
    &=2|\alpha|
    \Braket{\varphi|
    \frac{\hat{a}^{\dagger}_{\rm{S}}e^{i\theta_{\rm{LO}}}+\hat{a}_{\rm{S}}e^{-i\theta_{\rm{LO}}}}{2}
    |\varphi}_{\rm{S}}\nn[10pt]
    &=2|\alpha|\braket{\varphi|\hat{x}(\theta_{\rm{LO}})|\varphi}_{\rm{S}}
\end{align}
を得る.すなわち,$\hat{R}$の測定から,$\hat{x}(\theta_{\rm{LO}})$の期待値を知ることができる.例えば,$\hat{x}(\theta_{\rm{LO}}=0)$のとき,

% 
\section{コヒーレント状態による展開と準確率分布関数}
\subsection{$P$表現}
\begin{equation}
    \hrho=\int\int\braket{\alpha'|\hrho|\alpha''}\ket{\alpha'}\bra{\alpha'}\frac{d^2\alpha'd^2\alpha''}{\pi^2}
\end{equation}
一般的には,
\begin{equation}
    \hrho=\int P(\alpha)\ket{\alpha}\bra{\alpha}
\end{equation}
と展開できる.ある密度行列がこのように$\ket{\alpha}$による対角表現できるとき,これを密度行列の$P$表現とよぶ.$P(\alpha)$をGlauber-Sudarshan(グラウバー・スダーシャン)の$P$関数という.


\subsection{$Q$関数}
密度演算子のコヒーレント状態の対角要素$\braket{\alpha|\hrho|\alpha}$は正の確率分布関数として知られる.これを$\pi$で割ったものを$Q$関数と呼ぶ.もし,$\hrho$に対して,$\hrho=\int P(\beta)\ket{\beta}\bra{\beta}d^2\beta$という$P(\beta)$が存在すれば
\begin{align}
    Q(\alpha)&=\frac{\braket{\alpha|\hrho|\alpha}}{\pi}=\frac{1}{\pi}\int P(\beta)\|\braket{\alpha|\beta}\|^2 d^2\beta\nn[10pt]
    &=\frac{1}{\pi}\int P(\beta)e^{-|\alpha-\beta|^2}d^2\beta
\end{align}
であって,$Q$関数は$P(\beta)$のガウス型関数による畳み込みである.

\section{量子力学のc数表現}

% 
\section{Wigner関数}
Wigner分布関数(Wigner distribution function)は古典極限で古典的な分布関数に一致し,結果の解釈の助けになる.Wigner分布関数は次のWigner変換で定義される量である:
\begin{equation}
    W(x,p;t) = \frac{1}{2\pi\hbar}\int_{-\infty}^{\infty}d\xi \exp{\left(-\frac{ip\xi}{\hbar}\right)}
    \rho\left(x+\frac{\xi}{2}, x+\frac{\xi}{2};t\right).
\end{equation}
ここで,$\xi=()$である.この分布関数は$p$について積分を行うと座標に対する分布関数$W(x)$,$x$について積分すると運動量に対する分布関数$W(p)$を与えるハイブリット型の分布関数で,$W(x,p;t)$は古典統計力学でよく用いられる古典的な位相空間に類似した量である.しかし,Wigner分布関数は粒子の存在確率を与える古典的な意味の分布関数ではない.実際この関数は負の値を取ることもあり,「確率分布」という意味を持たないことに注意が必要である.\\
定義式より,

これを用いるとQuantum Liouville equation
\begin{equation}
    \frac{\partial}{\partial t}\hrho(t) =-\frac{i}{\hbar}[\hH, \hrho(t)]
\end{equation}
はWigner分布関数に対するLiouville方程式に変換することができる:
\begin{equation}
    \frac{\partial}{\partial t}W(r,p;t) =-\hat{\mathcal{L}}W(r,p;t)
\end{equation}
where
\begin{equation}
    -\hat{\mathcal{L}}\equiv
    \left\{
    -\frac{p}{m}\cdot\frac{\partial}{\partial r}
    +\frac{1}{i\hbar}\left[U\left(r-\frac{\hbar}{2i}\cdot\frac{\partial}{\partial p}\right)
    -U\left(r+\frac{\hbar}{2i}\cdot\frac{\partial}{\partial p}\right)
    \right]
    \right\}
\end{equation}
はWigner表示のLiouville演算子である.




\part{量子系のダイナミクス}
\section{2状態系の状態間の遷移}
2状態系の状態間の遷移を議論する際,厳密に解くか,摂動論を使うかで結果が異なる.ここでは2種類の方法の適用を試みる.


外部から何らかの作用が印加されたとき,古典力学では,系の初期状態から終状態への遷移の途中過程を時々刻々と追跡することできる.しかし,量子力学の場合,重ね合わせの原理によってそのような追跡は不可能であり,知ることができるのはそれらの状態間の遷移確率である.
\begin{equation}
    \hH=\hH_0+\hat{V}
\end{equation}
Hamiltonian$\hH$による状態ベクトル$\ket{\psi(t)}$の時間発展はSchrodinger方程式によって記述される.$\ket{\psi(t)}$を$\hH_0$の固有状態$\{\ket{n}\}$で展開する:
\begin{equation}
    \ket{\psi(t)}=\sum_{n}\ket{n}a_n.
\end{equation}
このように展開する理由は,状態$\ket{\psi(t)}$において,系を$\hH_0$の固有状態$\ket{m}$に発見する確率$|a_m(t)|^2$を求めることが目的だからである.これを\eqref{}に代入し,$\hH_0\ket{n}=E_n\ket{n}$を用いて,左から$\bra{m}$を掛けると,we obtain $a_m(t)$ of the equation
\begin{equation}
    i\hbar\frac{d}{dt}a_m(t)=E_m^{(0)}a_m(t)+\sum_n\braket{m|\hat{V}|n}a_n(t).
\end{equation}
\eqref{}で$\braket{n|\hat{V}|m}=v\delta_{n,m}$,すなわち,$\hat{V}$が対角的であれば,\eqref{}はそれぞれの係数$a_m(t)$
に関する独立な方程式に分離できるため,初期状態が他の状態に遷移することはない.状態間の遷移を起こすのは,外からの作用$\hat{V}$が非対角項を持つときである.\\

\subsection{2状態系}
ここで,$\hH_0$の固有状態が基底状態$\ket{g}$と第一励起状態$\ket{e}$の2個しかない2状態系を考える.すなわち,
\begin{equation}
    \hH_0\ket{g}=E_g^{(0)}\ket{g},\ \ \hH_0\ket{e}=E_e^{(0)}\ket{e},\ \ E_e^{(0)}>E_g^{(0)}
\end{equation}
とする.このとき,$\ket{\psi(t)}$は
\begin{equation}
    \ket{\psi(t)}=\ket{g}a_g(t)+\ket{e}a_e(t)
\end{equation}
となり,また相互作用は
\begin{equation}
    \hat{V} = \left(
        \begin{array}{cc}
       \braket{g|\hat{V}|g}&\braket{g|\hat{V}|e} \\[10pt]
        \braket{g|\hat{V}|e}&\braket{e|\hat{V}|e} \\
        \end{array}
        \right)
        =
        \left(
        \begin{array}{cc}
       0&v\\[10pt]
        v^{\ast}&0 \\
        \end{array}
        \right)
\end{equation}
で与えられるとする.するとこのとき,\eqref{}は
\begin{align}
    i\hbar\frac{d}{dt}a_g(t)&=E_g^{(0)}a_g(t)+v a_e(t)\\[10pt]
    i\hbar\frac{d}{dt}a_e(t)&=E_e^{(0)}a_e(t)+v^{\ast} a_g(t)
\end{align}
となる.ここで,$a_i(t)=C_i(t)\exp{[-iE_i^{(0)}t/\hbar]}$, $i=1,2$とおくと,\eqref{}は
\begin{align}
    i\hbar\frac{d}{dt}C_g(t)&=ve^{-\omega_0t}C_e(t)\\[10pt]
    i\hbar\frac{d}{dt}C_e(t)&=v^{\ast}e^{-\omega_0t} C_g(t)
\end{align}
ここで,$\omega_0\equiv(E_e^{(0)}-E_g^{(0)})/\hbar$である.さて,時刻$t=0$において,系が低エネルギー順位$E_g^{(0)}$の固有状態$\ket{g}$にあったとする.このとき,微分方程式\eqref{}の初期条件は,$C_g(0)=1$, $C_e(0)=0$,また\eqref{}より
\begin{equation}
    \frac{dC_1}{dt}(0)=0,\ \ \frac{dC_2}{dt}(0)=\frac{v^{\ast}}{i\hbar}
\end{equation}
となる.また,\eqref{}から$C_g(t)$を消去すると,
\begin{equation}
    \frac{d^2C_2(t)}{dt^2}-i\omega_0\frac{dC_2(t)}{dt}+\frac{|v|^2}{\hbar^2}C_2(t)=0.
\end{equation}
この方程式の一般解を求めるために,$C_2(t)=A\exp[i\omega t]$とおき,これを\eqref{}へ代入すると,we obtain
\begin{equation}
    \omega^2-\omega_0\omega-|v|^2/\hbar^2=0.
\end{equation}
$\omega$について解くと,$\omega=(1/2)[\omega_0\pm\sqrt{\omega_0^2+4|v\|^2/\hbar^2}]$である.したがって,\eqref{}の一般解は
\begin{equation}
    C_2(t)=e^{i\omega t/2}[k_1e^{i\Tilde{\omega} t/2}+k_2e^{i\Tilde{\omega} t/2}]
\end{equation}
で与えられる.ここで$\Tilde{\omega}\equiv\sqrt{\omega_0^2+2|v|^2/\hbar^2}$で,$k_1$と$k_2$は積分定数である.これらは,初期条件を使うと,$k_1=-v^{\ast}/\hbar\Tilde{\omega}$, $k_2=+v^{\ast}/\hbar\Tilde{\omega}$
と決まる.これより特殊解として次を得る:
\begin{equation}
    C_2(t)=(-2iv^{\ast}/\hbar\Tilde{\omega})e^{i\omega t/2}\frac{e^{i\Tilde{\omega} t/2}-e^{i\Tilde{\omega} t/2}}{2i}
    =(-2iv^{\ast}/\hbar\Tilde{\omega})e^{i\omega t/2}\sin{\Tilde{\omega}t/2}
\end{equation}
したがって,時刻$t$に系を状態$\ket{e}$に発見する確率は
\begin{equation}
    |a_2(t)|^2=|C_2(t)|^2
    =\frac{4|v|^2}{(\hbar\Tilde{\omega})^2}\sin^2{\left(\frac{\Tilde{\omega}t}{2}\right)}
\end{equation}
で与えられ,系が基底状態$\ket{g}$に残留している確率は
\begin{equation}
    |a_1(t)|^2=1-|a_2(t)|^2=1-\frac{4|v|^2}{(\hbar\Tilde{\omega})^2}\sin^2{\left(\frac{\Tilde{\omega}t}{2}\right)}
\end{equation}




\section{rabi振動}

\begin{equation}
    \hat{H}=
    \frac{\Delta}{2} \hat{\sigma}_{z} + \frac{\lambda}{2} \hat{\sigma}_{x}
    =\frac{\Omega}{2}
    \left(
    \frac{\Delta}{\Omega} \hat{\sigma}_{z} + \frac{\lambda}{\Omega} \hat{\sigma}_{x}
    \right)
\end{equation}
ここで,
\begin{equation}
    \Omega\equiv \sqrt{\Delta^2 + \lambda^2}
\end{equation}


\begin{align}
     \ket{\psi(t)}&=\left(
        \begin{array}{c}
       e^{i\omega t/2}
       \Bigl(
       \cos{\frac{\Omega }{2}t}
       +i\frac{\Delta}{\Omega}
       \sin{\frac{\Omega}{2}t}
       \Bigr)\\[15pt]
       %
       -ie^{-i\omega t/2}
       \frac{\lambda}{\Omega}
       \sin{\frac{\Omega}{2}t}\\
        \end{array}
        \right)\\[10pt]
        &=e^{i\omega t/2}
       \Bigl(
       \cos{\frac{\Omega }{2}t}
       +i\frac{\Delta}{\Omega}
       \sin{\frac{\Omega}{2}t}
       \Bigr)
       \ket{e}
       -ie^{-i\omega t/2}
       \frac{\lambda}{\Omega}
       \sin{\frac{\Omega}{2}t}\ket{g}
\end{align}
よって,状態が$\ket{e}$, $\ket{g}$に残っている確率はそれぞれ以下のように与えられる:
\begin{align}
    P_e &= |\hat{P}_e\ket{\psi(t)}|^2
    =  \cos^2{\frac{\Omega }{2}t}
       +\frac{\Delta^2}{\Omega^2}
       \sin^2{\frac{\Omega}{2}t}\\[10pt]
    P_g &= |\hat{P}_g\ket{\psi(t)}|^2
    = \frac{\lambda^2}{\Omega^2}
       \sin^2{\frac{\Omega}{2}t}
\end{align}



\subsection{Simulation結果とパラメータの対応関係}

\part{原子と光の相互作用}
\section{原子-光相互作用の一般論}
原子系と電磁場の相互作用を考える.原子系,電磁場そして相互作用Hamiltonianをそれぞれ$\hat{H}_{\rm{A}}$,$\hat{H}_{\rm{F}}$,$\hat{H}^{'}$としたとき,全系のHamiltonianは,電気双極子近似のもとで,
\begin{align}
    \hat{H} &= \hat{H}_{\rm{A}}+\hat{H}_{\rm{F}}+\hat{H}^{'}\\[10pt]
    \hat{H}_{\rm{A}}&=\sum_{i}E_{i}\ket{i}\bra{i}\\[10pt]
    \hat{H}_{\rm{F}}&=\sum_{k}\hbar\omega_k\left(\hat{a}_k^{\dag}\hat{a}_k+\frac{1}{2}\right)\\[10pt]
    \hat{H}^{'}&=-e\hat{\bm{r}}\cdot\hat{\bm{E}}
\end{align}
と表される.ここで,$\ket{i}$は原子系の$i$番目のエネルギー固有値$E_i$に対応する固有状態,$\hat{\bm{r}}$はポテンシャルの中心を原点とした時の電子の位置座標演算子であり,電場演算子$\hat{\bm{E}}$は原子内で一様であるとする.\\

相互作用$\hat{H}^{'}$については,原子系に演算する部分$e\hat{\bm{r}}$と電磁場に演算する部分$\hat{\bm{E}}$に分けられる.まず,演算子$e\hat{\bm{r}}$を固有関数系$\{\ket{i}\}$で展開すると,
\begin{equation}
    e\hat{\bm{r}}=e\hat{I}\hat{\bm{r}}\hat{I}
    =\sum_{i,j}\ket{i}\braket{i|\hat{\bm{r}}|j}\bra{j}
    =\sum_{i,j}P_{i,j}\hsig_{i,j}
\end{equation}
となる.ここで,$P_{i,j}\equiv e\braket{i|\hat{\bm{r}}|j}$は電気双極子遷移の遷移行列要素であり,$\hsig_{i,j}=\ket{i}\bra{j}$は,原子系の$\ket{j}$から$\ket{i}$へ変換する状態変換演算子である.一方,電場$\hat{\bm{E}}$は,原子が固定されていることにより,(電磁場の量子化を参照)を使い,
\begin{equation}
    \hat{\bm{E}}=\sum_{k}{\Tilde{\bm{e}}}\mathcal{E}_k(\hat{a}_k+\hat{a}_k^{\dag})
\end{equation}
と表される.ここで簡単のために,電磁場は直線偏光(ここよくわからない)としてこれらをまとめると,
\begin{equation}
    \hat{H}^{'}=-e\hat{\bm{r}}\cdot\hat{\bm{E}}
    =-(\sum_{i,j}P_{i,j}\hsig_{i,j})(\sum_{k}{\Tilde{\bm{e}}}\mathcal{E}_k(\hat{a}_k+\hat{a}_k^{\dag}))
    =\hbar\sum_{i,j}\sum_{k}g_k^{ij}\hsig_{i,j}(\hat{a}_k+\hat{a}_k^{\dag})
\end{equation}
where $g_k^{ij}=-P_{i,j}\cdot\Tilde{\bm{e}}\mathcal{E}_k$.
ここまでが一般論である.



\section{Jaynes-Cummings model}
2準位系を考える.第一励起状態を$\ket{e}$, 基底状態を$\ket{g}$とし,次のベクトル表示をとる:
\begin{equation}
     \ket{e}=\left(
        \begin{array}{c}
       1\\[10pt]
       0\\
        \end{array}
        \right),\ \ \ 
        \ket{g}=\left(
        \begin{array}{c}
       0\\[10pt]
       1\\
        \end{array}
        \right)
\end{equation}
とする.

\begin{equation}
     \hat{H}_{\rm{A}}=\left(
        \begin{array}{cc}
       \frac{1}{2}\hbar\omega&0\\[15pt]
       0&-\frac{1}{2}\hbar\omega\\
        \end{array}
        \right)
        =\frac{1}{2}\hbar\omega\hsig_z
\end{equation}
上昇演算子,下降演算子はそれぞれ次のように表される:
\begin{equation}
     \hsig_+=\ket{e}\bra{g}=\left(
        \begin{array}{cc}
       0&1\\[10pt]
       0&0\\
        \end{array}
        \right),\ \ \ 
        \hsig_-=\ket{g}\bra{e}=\left(
        \begin{array}{cc}
       0&0\\[10pt]
       1&0\\
        \end{array}
        \right)
\end{equation}

またこれらは,
\begin{equation}
    \hsig_{\pm}=\frac{\hsig_x\pm i\hsig_y}{2}
\end{equation}

\subsection{問題設定}
2準位原子が,周波数$\omega_0$の単一モードの電磁場と相互作用する状況を考える.このとき,2準位原子と単一モードの電磁場との相互作用を記述するHamiltonianは
\begin{equation}
    \hat{H}_{\rm{JC}}
    =\frac{1}{2}\hbar\omega\hsig_z
    +\hbar\omega_0\hat{a}^\dagger\ha
    +\hbar g(\hsig_+\hat{a}+\hsig_-\hat{a}^\dagger)
\end{equation}
と書ける.これはJaynes-Cummings modelと呼ばれている.


\subsection{Jaynes-Cummings modelの固有値問題}
ここでは,Jaynes-Cummings modelの固有状態および固有値を求める.そこで次の固有方程式を考える:
\begin{equation}
    \hat{H}_{\rm{JC}}\ket{\Psi}=E\ket{\Psi}
\end{equation}

ここで,$\ket{u,n}=\ket{u}\otimes\ket{n}$, $\ket{g,n}=\ket{g}\otimes\ket{n}$の基底で展開すると
\begin{align}
    \hsig_z\otimes\hat{1}_n
    &=(\ket{e}\bra{e}-\ket{g}\bra{g})\otimes
    \sum_{n=0}^{\infty}\ket{n}\bra{n}\nn[10pt]
    &=\sum_{n=0}^{\infty}(\ket{e,n}\bra{e,n}-\ket{g,n}\bra{g,n})
    \nn[10pt]
    &=-\ket{g,0}\bra{g,0}+\ket{e,0}\bra{e,0}
    -\ket{g,1}\bra{g,1}+\ket{e,1}\bra{e,1}+\cdots
    \nn[10pt]
    &=\left(
        \begin{array}{cc}
       -1&0\\[10pt]
       0&1\\
        \end{array}
        \right)
        \otimes
        \left(
        \begin{array}{ccccc}
        1                                                \\
         & 1          &        & \text{\huge{0}}   \\
         &                 & \ddots                     \\
         & \text{\huge{0}} &        & \ddots            \\
         &                 &        &           & \ddots
        \end{array}
        \right)\nn[10pt]
    &=  \left(
        \begin{array}{cccccc}
        -1                                                \\
         & 1          &        & &\text{\huge{0}}   \\
         &                 &-1                     \\
         & &        & 1           \\
         &                  \text{\huge{0}}&        &           & \ddots\\
         &                 &        &           &        &\ddots
        \end{array}
        \right)\nn[10pt]
\end{align}



\begin{align}
    \hat{1}_2\otimes\hat{a}^\dag\hat{a}
    &=(\ket{g}\bra{g}+\ket{e}\bra{e})\otimes
    \sum_{n=0}^{\infty}n\ket{n}\bra{n}\nn[10pt]
    &=\sum_{n=0}^{\infty}n(\ket{g,n}\bra{g,n}+\ket{e,n}\bra{e,n})
    \nn[10pt]
    &=0(\ket{g,0}\bra{g,0}+\ket{e,0}\bra{e,0})
    +1(\ket{g,1}\bra{g,1}+\ket{e,1}\bra{e,1})+\cdots
    \nn[10pt]
    &=\left(
        \begin{array}{cc}
       1&0\\[10pt]
       0&1\\
        \end{array}
        \right)
        \otimes
        \left(
        \begin{array}{cccccc}
        0                                                \\
         & 1          &        & &\text{\huge{0}}   \\
         &                 &2                    \\
         & &        & 3           \\
         &                  \text{\huge{0}}&        &           & \ddots\\
         &                 &        &           &        &\ddots
        \end{array}
        \right)\nn[10pt]
    &=  \left(
        \begin{array}{cccccc}
        0                                               \\
         & 1          &        & &\text{\huge{0}}   \\
         &                 &2                     \\
         & &        & 3           \\
         &                  \text{\huge{0}}&        &           & \ddots\\
         &                 &        &           &        &\ddots
        \end{array}
        \right)\nn[10pt]
\end{align}

\begin{align}
    \hsig_+\hat{a}
    &=(\ket{g}\bra{g}+\ket{e}\bra{e})\otimes
    \sum_{n=0}^{\infty}n\ket{n}\bra{n}\nn[10pt]
    &=\sum_{n=0}^{\infty}n(\ket{g,n}\bra{g,n}+\ket{e,n}\bra{e,n})
    \nn[10pt]
    &=0(\ket{g,0}\bra{g,0}+\ket{e,0}\bra{e,0})
    +1(\ket{g,1}\bra{g,1}+\ket{e,1}\bra{e,1})+\cdots
    \nn[10pt]
    &=\left(
        \begin{array}{cc}
       0&1\\[10pt]
       0&0\\
        \end{array}
        \right)
        \otimes
    \left(
        \begin{array}{ccccccc}
       0& 0& 0&0&0& \dots  & \dots\\
      1 &0& 0&0&0& \dots  & \dots\\
      0&\sqrt{2}& 0&0&0& \dots  & \dots\\
      0&0&\sqrt{3}&0& \dots& \dots  & \dots\\
      \vdots & \vdots &\vdots & \vdots & \ddots& \dots & \vdots \\
      \vdots & \vdots &\vdots & \vdots & \dots& \ddots & \vdots \\
      \dots & \dots&\dots & \dots& \dots& \dots  & \ddots
        \end{array}
        \right)\nn[10pt]
    &=  \left(
        \begin{array}{cccccc}
        0                                               \\
         & 1          &        & &\text{\huge{0}}   \\
         &                 &2                     \\
         & &        & 3           \\
         &                  \text{\huge{0}}&        &           & \ddots\\
         &                 &        &           &        &\ddots
        \end{array}
        \right)\nn[10pt]
\end{align}


\begin{align}
    \hsig_-\hat{a}^\dagger
    &=(\ket{g}\bra{g}+\ket{e}\bra{e})\otimes
    \sum_{n=0}^{\infty}n\ket{n}\bra{n}\nn[10pt]
    &=\sum_{n=0}^{\infty}n(\ket{g,n}\bra{g,n}+\ket{e,n}\bra{e,n})
    \nn[10pt]
    &=0(\ket{g,0}\bra{g,0}+\ket{e,0}\bra{e,0})
    +1(\ket{g,1}\bra{g,1}+\ket{e,1}\bra{e,1})+\cdots
    \nn[10pt]
    &=\left(
        \begin{array}{ccccccc}
       0& 0& 0&0&0& \dots  & \dots\\
      1 &0& 0&0&0& \dots  & \dots\\
      0&\sqrt{2}& 0&0&0& \dots  & \dots\\
      0&0&\sqrt{3}&0& \dots& \dots  & \dots\\
      \vdots & \vdots &\vdots & \vdots & \ddots& \dots & \vdots \\
      \vdots & \vdots &\vdots & \vdots & \dots& \ddots & \vdots \\
      \dots & \dots&\dots & \dots& \dots& \dots  & \ddots
        \end{array}
        \right)\nn[10pt]
    &=  \left(
        \begin{array}{cccccc}
        0                                               \\
         & 1          &        & &\text{\huge{0}}   \\
         &                 &2                     \\
         & &        & 3           \\
         &                  \text{\huge{0}}&        &           & \ddots\\
         &                 &        &           &        &\ddots
        \end{array}
        \right)\nn[10pt]
\end{align}


したがって,Hamiltonian $\hat{H}_{\rm{JC}}$を行列表示すると,
\begin{align}
    \hat{H}_{\rm{JC}}
    &=\frac{1}{2}\hbar\omega(\hsig_z\otimes\hat{1}_N)
    +\hbar\omega_0(\hat{1}_2\otimes\hat{a}^\dagger\hat{a})
    +\hbar g(\hsig_+\hat{a}+\hsig_-\hat{a}^\dagger)\nn[10pt]
    &=  \left(
        \begin{array}{cccccc}
        - \frac{1}{2}\hbar\omega                                              \\
         & \frac{1}{2}\hbar\omega         &        & &\text{\huge{0}}   \\
         &                 &-\frac{1}{2}\hbar\omega                  \\
         & &        & \frac{1}{2}\hbar\omega         \\
         &                  \text{\huge{0}}&        &           & \ddots\\
         &                 &        &           &        &\ddots
        \end{array}
        \right)
    +  \left(
        \begin{array}{cccccc}
        0                                               \\
         & \hbar\omega_0          &        & &\text{\huge{0}}   \\
         &                 &2\hbar\omega_0                     \\
         & &        & 3\hbar\omega_0           \\
         &                  \text{\huge{0}}&        &           & \ddots\\
         &                 &        &           &        &\ddots
        \end{array}
        \right)\nn[10pt]
    &+  \left(
        \begin{array}{cccccccc}
       0& 0& 0&0&0&0& \dots  & \dots\\
      0&0& g&0&0&0& \dots  & \dots\\
      0&g& 0&0&0&0& \dots  & \dots\\
      0&0&0&0&\sqrt{2}g& \dots& \dots  & \dots\\
      0&0&0&\sqrt{2}g&0& \dots& \dots  & \dots\\
      \vdots & \vdots &\vdots&\vdots & \vdots & \ddots& \dots & \vdots \\
      \vdots & \vdots &\vdots&\vdots & \vdots & \dots& \ddots & \vdots \\
      \dots & \dots&\dots&\dots & \dots& \dots& \dots  & \ddots
        \end{array}
        \right)\nn[10pt]
    &=  \left(
        \begin{array}{cccccccc}
       - \frac{1}{2}\hbar\omega& 0& 0&0&0&0& \dots  & \dots\\
      0&\frac{1}{2}\hbar\omega& g&0&0&0& \dots  & \dots\\
      0&g& - \frac{1}{2}\hbar\omega+\hbar\omega_0&0&0&0& \dots  & \dots\\
      0&0&0&\frac{1}{2}\hbar\omega+\hbar\omega_0&\sqrt{2}g& \dots& \dots  & \dots\\
      0&0&0&\sqrt{2}g&- \frac{1}{2}\hbar\omega+2\hbar\omega_0& \dots& \dots  & \dots\\
      \vdots & \vdots &\vdots&\vdots & \vdots & \ddots& \dots & \vdots \\
      \vdots & \vdots &\vdots&\vdots & \vdots & \dots& \ddots & \vdots \\
      \dots & \dots&\dots&\dots & \dots& \dots& \dots  & \ddots
        \end{array}
        \right)
\end{align}
よって,状態をベクトル$\ket{e,n}$と$\ket{g,n+1}$で張られる部分空間のブロック行列を得る:
\begin{align}
    \hat{H}_{\rm{JC}}^{(n)}=\left(
        \begin{array}{cc}
      \frac{1}{2}\hbar\omega+n\hbar\omega_0& \hbar\sqrt{n+1}g\\[10pt]
      \hbar\sqrt{n+1}g& - \frac{1}{2}\hbar\omega+(n+1)\hbar\omega_0\\
        \end{array}
        \right)
\end{align}
ただし,$n=0,1,2,\ldots$である.
行列$\hat{H}_{\rm{JC}}^{(n)}$から得られる固有方程式は
\begin{align}
    \left(
        \begin{array}{cc}
      E-(\hbar\omega/2+n\hbar\omega_0)& -\hbar\sqrt{n+1}g\\[10pt]
      -\hbar\sqrt{n+1}g& E-(-\hbar\omega/2+(n+1)\hbar\omega_0)\\
        \end{array}
        \right)
        \left(
        \begin{array}{c}
      a\\[10pt]
      b\\
        \end{array}
        \right)=0
\end{align}

\begin{align}
    \left|
    E\hat{1}_2-\hat{H}_{\rm{JC}}^{(n)}
    \right|
    &=\left|\left(
        \begin{array}{cc}
      E-(\hbar\omega/2+n\hbar\omega_0)& -\hbar\sqrt{n+1}g\\[10pt]
      -\hbar\sqrt{n+1}g& E-(-\hbar\omega/2+(n+1)\hbar\omega_0)\\
        \end{array}
        \right)\right|\nn[10pt]
    &=\left[
    E-(\hbar\omega/2+n\hbar\omega_0)
    \right]
    \left[
    E-(-\hbar\omega/2+(n+1)\hbar\omega_0)
    \right]
    -(\hbar g)^2(n+1)=0
\end{align}


これを解くと
\begin{equation}
    E_{\pm}=\left(
    n+\frac{1}{2}
    \right)\hbar\omega_0\pm\frac{\hbar\Delta_n}{2}
\end{equation}
を得る.

$E=E_+$のとき
\begin{align}
    \left(
        \begin{array}{cc}
      \left(n+\frac{1}{2}
    \right)\hbar\omega_0+\frac{\hbar\Delta_n}{2}-(\hbar\omega/2+n\hbar\omega_0)& -\hbar\sqrt{n+1}g\\[10pt]
      -\hbar\sqrt{n+1}g& \left(n+\frac{1}{2}
    \right)\hbar\omega_0+\frac{\hbar\Delta_n}{2}-(-\hbar\omega/2+(n+1)\hbar\omega_0)\\
        \end{array}
        \right)
        \left(
        \begin{array}{c}
      a\\[10pt]
      b\\
        \end{array}
        \right)&=0\nn[10pt]
        \left(
        \begin{array}{cc}
      \frac{\hbar}{2}(\omega_0-\omega+\Delta_n)& -\hbar\sqrt{n+1}g\\[10pt]
      -\hbar\sqrt{n+1}g& -\frac{\hbar}{2}(\omega_0-\omega+\Delta_n)\\
        \end{array}
        \right)
        \left(
        \begin{array}{c}
      a\\[10pt]
      b\\
        \end{array}
        \right)&=0
\end{align}

この連立方程式を解くと
\begin{align}
    \frac{\hbar}{2}(\omega_0-\omega+\Delta_n)a-\hbar\sqrt{n+1}gb&=0\\[10pt]
    a&=\frac{\hbar\sqrt{n+1}g}{\frac{\hbar}{2}(\omega_0-\omega+\Delta_n)}b
\end{align}

規格化条件より$a^2+b^2=1$

\begin{align}
    1&=a^2+b^2=\left(
    1+\frac{\hbar^2(n+1)g^2}{\frac{\hbar^2}{4}(\omega_0-\omega+\Delta_n)^2}
    \right)b^2\nn[10pt]
    &=
    \frac{\frac{\hbar^2}{4}(\omega_0-\omega+\Delta_n)^2+\hbar^2(n+1)g^2}{\frac{\hbar^2}{4}(\omega_0-\omega+\Delta_n)^2}
    b^2
    =\frac{(\omega_0-\omega+\Delta_n)^2+4(n+1)g^2}{(\omega_0-\omega+\Delta_n)^2}
    b^2
    \end{align}
    
    \begin{align}
    \therefore 
    b^2&=\frac{(\Delta_n-(\omega-\omega_0))^2}{(\Delta_n-(\omega-\omega_0))^2+4(n+1)g^2}\nn[10pt]
    \therefore 
    b&=\frac{(\Delta_n-\delta)}{\sqrt{(\Delta_n-\delta)^2+4(n+1)g^2}}
    \equiv\cos\theta_n
\end{align}
ここで,$\delta=\omega-\omega_0$である.
\begin{align}
    a=\frac{2\sqrt{n+1}g}{(\omega_0-\omega+\Delta_n)}b
    =\frac{2\sqrt{n+1}g}
    {\sqrt{(\Delta_n-\delta)^2+4(n+1)g^2}}
    \equiv\sin\theta_n
\end{align}
ここで
\begin{equation}
    \tan\theta_n
    =\frac{\sin{\theta_n}}{\cos{\theta_n}}
    =\frac{2\sqrt{n+1}g}{(\Delta_n-\delta)}
\end{equation}
である.

よって,固有状態$\ket{\Psi_{\pm}}$についても
\begin{align}\label{JC_eigen_state1}
    \ket{\Psi_+}&=\sin{\theta_n}\ket{e,n}+\cos\theta_n\ket{g,n+1}\\[10pt]
    \label{JC_eigen_state2}
    \ket{\Psi_-}&=\cos{\theta_n}\ket{e,n}-\sin\theta_n\ket{g,n+1}
\end{align}

得られたエネルギー固有値から,電磁場と結合した原子のエネルギー準位が2つに分裂し,その差が
\begin{equation}
    \Delta E\equiv E_+-E_-=\hbar\Delta_n=\hbar\sqrt{4g^2(n+1)+\delta}
\end{equation}
であることがわかる.これをラビ分裂という.特に,電磁場が真空状態$(n=0)$の場合でさえも,原子のエネルギー準位が$\hbar\sqrt{4g^2+\delta}$だけ分裂を起こす.これを真空ラビ分裂(Vacume Labi Splitting)と呼ぶ.

\eqref{JC_eigen_state1},\eqref{JC_eigen_state2}のような電磁場と結合した状態にある原子は,あたかも電磁場のドレスをまとった原子のように見えるので,ドレストアトムと呼ばれる.




\subsection{状態の時間発展}
次にJC modelにおける状態の時間発展について論じる.今,初期時刻$t=0$では原子は第一励起状態$\ket{e}$にあり,電磁場は光子数状態$\ket{n}$にあるとする.このとき初期状態は
\begin{equation}
    \ket{\Psi(t=0)}=\ket{e}\ket{n}\equiv\ket{u,n}
\end{equation}
となる.系の時間発展は,Schr\"{o}dinger方程式
\begin{equation}\label{Sch.eq_JC}
    i\hbar\frac{\partial}{\partial t}\ket{\Psi(t)}=\hat{H}_{\rm{JC}}\ket{\Psi(t)}
\end{equation}
で記述される.

\eqref{Sch.eq_JC}の形式的な解は以下で与えられる:
\begin{equation}\label{Solve_Sch.eq_JC}
    \ket{\Psi(t)}=e^{-i\hat{H}_{\rm{JC}}t/\hbar}\ket{\Psi(0)}=e^{-i\hat{H}_{\rm{JC}}t/\hbar}\ket{e,n}
\end{equation}
ここで,\eqref{JC_eigen_state1},\eqref{JC_eigen_state2}を逆に解くことで,初期状態$\ket{e,n}$をJC-modelの固有状態$\ket{\Psi_+}$,$\ket{\Psi_-}$での展開した形を具体的に記述することができる:
\begin{align}
    \sin\theta_n\ket{\Psi_+}&=\sin^2{\theta_n}\ket{e,n}+\sin{\theta_n}\cos\theta_n\ket{g,n+1}\\[10pt]
    \cos{\theta_n}\ket{\Psi_-}&=\cos^2{\theta_n}\ket{e,n}-\sin\theta_n\cos\theta_n\ket{g,n+1}
\end{align}

\begin{equation}\label{initial_state}
    \ket{u,n} = \sin{\theta_n}\ket{\Psi_+}+\cos{\theta_n}\ket{\Psi_-}
\end{equation}

\eqref{initial_state}を\eqref{Solve_Sch.eq_JC}へ代入すると
\begin{align}\label{Solve_Sch.eq_JC2}
    \ket{\Psi(t)}&=e^{-i\hat{H}_{\rm{JC}}t/\hbar}
    (\sin{\theta_n}\ket{\Psi_+}+\cos{\theta_n}\ket{\Psi_-}\nn[10pt]
    &=
    \sin{\theta_n}e^{-iE_+t/\hbar}\ket{\Psi_+}
    +\cos{\theta_n}e^{-iE_-t/\hbar}\ket{\Psi_-}\nn[10pt]
    &=
    \sin{\theta_n}e^{-iE_+t/\hbar}\Bigl(\sin{\theta_n}\ket{e,n}+\cos\theta_n\ket{g,n+1}\Bigr)
    +\cos{\theta_n}e^{-iE_-t/\hbar}\Bigl(\cos{\theta_n}\ket{e,n}-\sin\theta_n\ket{g,n+1}\Bigr)
    \nn[10pt]
    %
    &=
    \Bigl(e^{-iE_+t/\hbar}\sin^2{\theta_n}
    +e^{-iE_-t/\hbar}\cos^2{\theta_n}\Bigr)\ket{e,n}
    +
    \sin{\theta_n}\Bigl(e^{-iE_+t/\hbar}-e^{-iE_-t/\hbar}\Bigr)\sin{\theta_n}\cos\theta_n\ket{g,n+1}
\end{align}
\eqref{Solve_Sch.eq_JC2}にJC modelのエネルギー固有値\eqref{},\eqref{}を代入すると,
\begin{align}
    e^{-iE_{\pm}t/\hbar}
    &=\exp{
    \left\{
    -i
    \left(
    \left(
    n+\frac{1}{2}
    \right)\hbar\omega_0\pm\frac{\hbar\Delta_n}{2}
    \right)
    t/\hbar
    \right\}
    }\nn[10pt]
    &=\exp{
    \left\{
    -i
    \left(
    n+\frac{1}{2}
    \right)\omega_0t
    \right\}
    }
    \exp{
    \left\{
    \mp i\frac{\hbar\Delta_n}{2}t
    \right\}
    }\nn[10pt]
    &=\exp{
    \left\{
    -i
    \left(
    n+\frac{1}{2}
    \right)\omega_0t
    \right\}
    }
    \left\{
    \cos\frac{\Delta_n t}{2}
    \mp i\sin\frac{\Delta_n t}{2}
    \right\}
\end{align}
より,まず
\begin{align}
    &e^{-iE_+t/\hbar}\sin^2{\theta_n}
    +e^{-iE_-t/\hbar}\cos^2{\theta_n}\nn[10pt]
    &=\exp{
    \left\{
    -i
    \left(
    n+\frac{1}{2}
    \right)\omega_0t
    \right\}
    }
    \left\{
    \left(
    \cos\frac{\Delta_n t}{2}
    -i\sin\frac{\Delta_n t}{2}
    \right)\sin^2{\theta_n}
    +\left(
    \cos\frac{\Delta_n t}{2}
    +i\sin\frac{\Delta_n t}{2}
    \right)\cos^2{\theta_n}
    \right\}\nn[10pt]
    %%%%%
    &=e^{
    -i
    \left(
    n+\frac{1}{2}
    \right)\omega_0t
    }
    \left\{
    \cos\frac{\Delta_n t}{2}
    \left(\sin^2{\theta_n}
    +\cos^2{\theta_n}
    \right)
    -\sin\frac{\Delta_n t}{2}
    \left(
    \cos^2{\theta_n}
    -i\sin^2{\theta_n}
    \right)
    \right\}
    \nn[10pt]
    %%%%%
    &=e^{
    -i
    \left(
    n+\frac{1}{2}
    \right)\omega_0t
    }
    \left\{
    \cos\frac{\Delta_n t}{2}
    -i\sin\frac{\Delta_n t}{2}
    \cos{2\theta_n}
    \right\}
\end{align}
次に
\begin{align}
    e^{-iE_+t/\hbar}
    -e^{-iE_-t/\hbar}
    &=\exp{
    \left\{
    -i
    \left(
    n+\frac{1}{2}
    \right)\omega_0t
    \right\}
    }
    \left(
    \cos\frac{\Delta_n t}{2}
    -i\sin\frac{\Delta_n t}{2}
    -
    \cos\frac{\Delta_n t}{2}
    -i\sin\frac{\Delta_n t}{2}
    \right)\nn[10pt]
    &=\exp{
    \left\{
    -i
    \left(
    n+\frac{1}{2}
    \right)\omega_0t
    \right\}
    }
    \left(
    -2i\sin\frac{\Delta_n t}{2}
    \right)
\end{align}
したがって,JC modelの解として次式を得る:
\begin{align}\label{Solve_Sch.eq_JC2}
    \ket{\Psi(t)}&=
    e^{
    -i
    \left(
    n+\frac{1}{2}
    \right)\omega_0t
    }
    \left[
    \left\{
    \cos\frac{\Delta_n t}{2}
    -i\sin\frac{\Delta_n t}{2}
    \cos{2\theta_n}
    \right\}\ket{e,n}
    +
    \left(
    -i\sin\frac{\Delta_n t}{2}
    \right)2\sin{\theta_n}\cos\theta_n\ket{g,n+1}
    \right]\nn[10pt]
    &=
    e^{
    -i
    \left(
    n+\frac{1}{2}
    \right)\omega_0t
    }
    \left[
    \left\{
    \cos\frac{\Delta_n t}{2}
    -i\sin\frac{\Delta_n t}{2}
    \cos{2\theta_n}
    \right\}\ket{e,n}
    -i\sin\frac{\Delta_n t}{2}
    \sin{2\theta_n}\ket{g,n+1}
    \right]
\end{align}
この解について考察を行う.まず,$\ket{g,n+1}$の展開係数に含まれる
\begin{equation}
    \sin{2\theta_n}=2\sin\theta_n\cos\theta_n
    =\frac{4g\sqrt{n+1}(\Delta_n-\delta)}{(\Delta_n-\delta)^2+4(n+1)g^2}
\end{equation}
に注目すると,Detuning$\delta$が十分大きいとき$(\delta\to\infty)$,
\begin{equation}
    \sin{2\theta_n}
    =\frac{4g\sqrt{n+1}(\Delta_n\delta-1)}{\delta(\Delta_n/\delta-1)^2+4(n+1)g^2/\delta}
    \to\mathcal{O}{(1/\delta)}
\end{equation}
となる.
つまり,$\delta=\omega-\omega_0$が大きくなると,原子は基底状態$\ket{g,n+1}$の状態へ遷移しにくくなることがわかる.


次にOn resonant$(\delta=0),\therefore\omega=\omega_0$の場合,
\begin{equation}
    \Delta_n=2g\sqrt{n+1},\ \ \ \sin{2\theta_n}=1,\ \ \ \cos{2\theta_n}=0
\end{equation}
となるので,系の時間発展は次のようになる:
\begin{align}\label{Solve_Sch.eq_JC2}
    \ket{\Psi(t)}
    &=
    e^{
    -i
    \left(
    n+\frac{1}{2}
    \right)\omega_0t
    }
    \left[
    \cos{(g\sqrt{n+1} t)}
    \ket{e,n}
    -i\sin{(g\sqrt{n+1} t)}\ket{g,n+1}
    \right]
\end{align}
ここから,時刻$t$に原子が第一励起状態に見いだされる確率$P_e(t)$および基底状態に見いだされる確率$P_g(t)$は次のように与えられる:
\begin{align}
    P_{e}(t)&=|\Braket{e,n|\Psi(t)}|^2
    =\cos^2{(g\sqrt{n+1} t)}\\[10pt]
    P_{g}(t)&=|\Braket{g,n+1|\Psi(t)}|^2
    =\sin^2{(g\sqrt{n+1} t)}.
\end{align}
また,時刻$t$における平均光子数は次のように与えられる:
\begin{align}
    \Braket{\hat{a}^\dagger\hat{a}}(t)
    &=\braket{\Psi(t)|\hat{a}^{\dagger}\hat{a}|\Psi(t)}\nn[10pt]
    &=
    (\cos{(g\sqrt{n+1} t)}
    \bra{e,n}
    +i\sin{(g\sqrt{n+1} t)}\bra{g,n+1})
    (\hat{a}^\dagger\hat{a})
    (\cos{(g\sqrt{n+1} t)}
    \ket{e,n}
    -i\sin{(g\sqrt{n+1} t)}\ket{g,n+1})\nn[10pt]
    &=
    (\cos^2{(g\sqrt{n+1} t)}
    \braket{e,n|\hat{a}^\dagger\hat{a}|e,n}
    -i\sin{(g\sqrt{n+1} t)}\cos{(g\sqrt{n+1} t)}
    \braket{e,n|\hat{a}^\dagger\hat{a}|g,n+1}\nn[10pt]
    %%%%%
    &-i\sin{(g\sqrt{n+1} t)}\cos{(g\sqrt{n+1} t)}
    \braket{g,n+1|\hat{a}^\dagger\hat{a}|e,n}
    +\sin^2{(g\sqrt{n+1} t)}\braket{g,n+1|\hat{a}^\dagger\hat{a}|g,n+1})\nn[10pt]
    &=
    n\cos^2{(g\sqrt{n+1} t)}
    +(n+1)\sin^2{(g\sqrt{n+1} t)}\nn[10pt]
    &=
    n
    +\sin^2{(g\sqrt{n+1} t)}
    =n+P_{g}(t)
\end{align}

このように,原子は決まった周波数
\begin{equation}
    \Omega_{\rm{Rabi}}=2g\sqrt{n+1}
\end{equation}
で光子の吸収と放出を繰り返す.これをRabi振動という.また,$\Delta_n$や$\Omega_{\rm{Rabi}}$はRabi周波数と呼ばれる.特に,電磁場の初期状態が真空状態の場合,すなわち,$n=0$の場合も原子の初期状態が第一励起状態であればRabi振動が起こることがわかる.







\bibliographystyle{unsrt}%参考文bibliographystyle献出力スタイル
\bibliography{myrefs}
\end{document}





