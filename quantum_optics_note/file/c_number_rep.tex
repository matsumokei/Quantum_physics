
\section{コヒーレント状態による展開と準確率分布関数}
\subsection{$P$表現}
\begin{equation}
    \hrho=\int\int\braket{\alpha'|\hrho|\alpha''}\ket{\alpha'}\bra{\alpha'}\frac{d^2\alpha'd^2\alpha''}{\pi^2}
\end{equation}
一般的には,
\begin{equation}
    \hrho=\int P(\alpha)\ket{\alpha}\bra{\alpha}
\end{equation}
と展開できる.ある密度行列がこのように$\ket{\alpha}$による対角表現できるとき,これを密度行列の$P$表現とよぶ.$P(\alpha)$をGlauber-Sudarshan(グラウバー・スダーシャン)の$P$関数という.


\subsection{$Q$関数}
密度演算子のコヒーレント状態の対角要素$\braket{\alpha|\hrho|\alpha}$は正の確率分布関数として知られる.これを$\pi$で割ったものを$Q$関数と呼ぶ.もし,$\hrho$に対して,$\hrho=\int P(\beta)\ket{\beta}\bra{\beta}d^2\beta$という$P(\beta)$が存在すれば
\begin{align}
    Q(\alpha)&=\frac{\braket{\alpha|\hrho|\alpha}}{\pi}=\frac{1}{\pi}\int P(\beta)\|\braket{\alpha|\beta}\|^2 d^2\beta\nn[10pt]
    &=\frac{1}{\pi}\int P(\beta)e^{-|\alpha-\beta|^2}d^2\beta
\end{align}
であって,$Q$関数は$P(\beta)$のガウス型関数による畳み込みである.

\section{量子力学のc数表現}
